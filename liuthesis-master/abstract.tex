The work in this thesis covers the design, growth and characterisation of neutron multilayers. The achieved reflectivity performance of a neutron multilayer depends on the achieved optical contrast between the layers as well as the achieved interface width between the layers. Because the reflectivity of a neutron multilayer depends exponentially on the square of the interface width, even a modest improvement can substantially increase the achieved reflectivity performance. It is for this reason that a large part of this work has been focused in making smoother and more abrupt multilayers, descreasing the total interface width. As multilayers are such an integral component of most neutron optical instruments, any improvement in terms of reflectivity performance has broad implications for all conducted neutron scattering experiments. \\
\\
The conventional material system of choice for neutron optical components is Ni/Ti, owing to the high contrast in scattering length density (SLD). The reflected intensity of such components is largely dependent on the interface width, primarily caused by the  formation of nanocrystallites, interdiffusion, and/or intermixing. Apart from hampering the reflectivity performance, the finite interface width between the layers also limits the minimum usable layer thickness in the mirror stack. In this work, Ni/Ti based multilayers are grown using ion-assisted magnetron sputtering. By co-depositing \natBC in the multilayer stack, the formation of nanocrystallites as well as intermetallics between the interfaces were succesfully prevented. The co-deposition of \natBC has been combined with a modulated ion assistance scheme, where an initial buffer layer is grown at a low ion energy creating abrupt interfaces, while the remainder of the layer is grown at a higher ion energy, smoothening the interfaces. \\ X-ray reflectivity (XRR) measurements show significant improvements in terms of reflectivity when the multilayers are co-deposited with \natBC. This has further been investigated using low neutron-absorbing isotope-enriched \BC. The deposited \BC containing multilayers have been characterized using neutron reflectometry, X-ray reflectivity, transmission electron microscopy, elastic recoil detection analysis, X-ray photoelectron spectroscopy and grazing incidence small angle scattering (GISAXS). Structural parameters in the growth direction such as interface width and thickness variations have been determined by combined fits on X-ray and neutron reflectivity measurements, while the interface morphology has been investigated using GISAXS. The GISAXS measurements show that the co-deposition of \BC leads to mounded interfaces with more strongly vertically correlated interface profiles, this can be attributed to a decreased adatom mobility when \BC is incorporated. The coupled fits to specular X-ray and neutron reflectivity measurements suggest a significant improvement in interface width for the samples that have been co-deposited with \BC using a modulated on assistance scheme during deposition, where an interface width of 2.7 Å has been achieved in a \BC containing multilayer. The reflectivity for such \BC containing multilayers have been simulated for a neutron supermirror (N = 5000) using the IMD software. The predicted reflectivity performance at the critical angle of an m = 6 supermirror for the \BC containing samples amounts to about 76$\%$, which is a significant increase compared to the current state-of-the-art supermirrors made with pure Ni/Ti, which have a predicted reflectivity of 65$\%$. This results in a reflectivity increase from 1.3$\%$ to 6.6$\%$ after a total of 10 reflections from this critical angle, translating to an increase of over 500\% for neutrons reflected at this angle. At lower incidence angles, corresponding to thicker periods, the current state-of-the-art is expected to perform better due the higher optical contrast. By combining a material system with a higher optical contrast in the thicker layers with \BC containing multilayers in the thinner layers, a high reflectivity performance can be obtained over all reflected incidence angles. This has been demonstrated by a proof of concept, where thicker Ni/Ti multilayers with thin (0.15 nm) \BC interlayers between the interfaces show the best reflectivity at a period of 84 Å, while \BC deposited in both multilayer stacks has shown the best reflectivity performance at thinner periods at 48 Å and 30 Å. \\
\\ Chemically  homogeneous  \natBC interference mirrors with  $^\textrm{11}$B/$^\textrm{10}$B  isotope modulation have been investigated as well to seek new possibilities for future neutron optical components. Preliminary neutron reflectometry were performed, showing very promising results with 10\% absolute reflectivity for a 128 nm thick multilayer consisting of only 20 bilayer periods with a periodicity of 33.1 Å, showing how only few layers are needed for a high reflectivity. 