\chapter{Introduction}\label{introduction}

\section{Background}
Neutron scattering is a versatile non-destructive experimental technique to study the structure and dynamics of materials. The technique is used by over 5000 researchers over the world \cite{lefmann2017neutron} spanning over an increasingly large range of disciplines, including physics, chemistry, biology, ceramics and metallurgy \cite{applications}. Neutron scattering differentiates itself from X-ray scattering due to the interaction with the investigated material. While X-rays interact with the electron cloud, neutrons interact with the nucleus of the atom. This gives several distinct advantages for neutron scattering. As the nucleus of an atom is only a tiny portion of the atom, most of the material will be empty space to a neutron. Because of this, neutrons have a very large penetration depth \cite{neutronapplicationspaper}, making it possible to study bulk materials. Moreover, the wavelength of neutrons is similar to the atomic spacing in solids, making it ideal for structural studies \cite{neutronscattering_book}. Another advantage of neutron scattering arises from the fact that neutrons carry a magnetic dipole moment, the interaction of the neutron's spin with unpaired electrons in ferro- and paramagnetic materials gives rise to magnetic scattering \cite{magneticscattering}. The scattering properties for neutrons vary seemingly randomly across the periodic table, even different isotopes of the same element can have completely different properties for neutrons. This makes the combination X-ray and neutron scattering a very powerful technique, as the scattering properties for X-rays are dependent on the atomic number. By combining these two techniques, it is possible to obtain two completely independent data sets with the same structural information. This latter advantage is exploited in this report, where a combination of neutron and X-ray scattering is used to find the structural properties of deposited multilayers. \\
The dawn of neutron scattering began in the first half of the twentieth century. The first experiments on Bragg reflection using neutrons were performed as early as 1936, however it wasn't until the second half of the 1940s that the invention of the nuclear reactor made the first proper neutron experiments possible \cite{belushkin1999neutron}. As free neutrons have a  mean lifetime of 15 minutes \cite{neutronscattering_book}, they need to be produced while running the experiment. Traditionally, free neutrons for scientific experiments are produced at fission reactors. In this process, a neutron collides with uranium-235, forming the following nuclear reaction:
\begin{equation}
	^{235}\textrm{U} + \textrm{ } ^1_0\textrm{n}  \rightarrow \textrm{fission fragments} + 2.52 \textrm{ } ^1_0 \textrm{n} + 180 \textrm{MeV},
\end{equation}
releasing two or three free neutrons carrying an energy of 1.29 MeV \cite{neutronscattering_book}, averaging to 2.52 free neutrons after each collision. Each emitted neutron can undergo a fission reaction with another $^{235}$U particle forming a chain reaction where even more free neutrons are emitted. This technique has however reached its technological limits in terms of neutron flux due to power density problems in the reactor core \cite{spallationbauer}\cite{sourcesbauer}, but also the risk of nuclear proliferation has made the enrichment of $^{235}$U a politically difficult matter \cite{sourcesbauer}. A slightly newer technique that does not have this disadvantage is the use of spallation sources. In these sources, proton pulses are accelerated to high energies, typically in the range of GeV, and directed onto a target. The resulting spallation reaction could be described as 
\begin{equation}
	^1_1\textrm{H}  + \textrm{ST} \rightarrow \textrm{ST fragments} + k \textrm{} ^1_0 \textrm{n}.
\end{equation}
In this reaction, ST describes the spallation target, which releases several different spallation fragments as well as a total of $k$ neutrons per spallation event. Depending on the target material, the total amount of neutrons for each spallation event could be as high as 50 \cite{neutronscattering_book}, making an extremely high peak flux of neutrons possible. For both of these techniques, the energy of the neutrons is too high to be used for experiments, with energies that correspond to a wavelength in the order of $\lambda = 10^{-5}$ Å \cite{neutronscattering_book}. In order to reduce the energy of the neutron beam, the neutrons are transferred through a moderator material such as H$_\textrm{2}$O. This slows the neutrons down, and thereby reduces their energy such that they can be used for neutron scattering experiments. Finally, neutrons need to be brought from the source to experiments. As neutrons do not have any charge, they cannot be bend by any electromagnetic field. Instead, neutron mirrors are used that enclose the flight path of neutrons \cite{neutronbookmatrac}. 
\section{European Spallation Source}
As this project is part of the SwedNess neutron graduate school, it should be seen in context of the construction of the European Spallation Source (ESS), which is being built in Lund. An overarching goal of SwedNess in general is to develop an understanding of neutron scattering in particular and thereby increasing the expertise that is needed with planned operations of ESS in the future. The spallation source is designed to deliver 5 MW of 2.5 GeV protons to a single target, which corresponds to an increase in average and peak neutron flux by a factor of 30 compared to Europe's currently most powerful pulsed spallation source at ISIS in the UK \cite{ESSstudy}. The aim of ESS is to deliver a time average flux of neutrons that is comparable to the brightest continuous source in existence at ILL, using a low enough pulse repetition rate such in order to avoid loss of efficiency at high flux even for cold neutrons applications \cite{ESSdesign}. Construction started at 2014, while full operation was planned in 2025 \cite{ESSdesign}, this has now been moved to 2027 \cite{ESS_operational}. 
\section{Neutron multilayers}
Neutron multilayers are used in many different neutron optical instruments. A common example is that of neutron guides, which are necessary to guide a neutron beam from source to experiment. In a neutron guide, different neutron supermirrors are used to enclose the beam trajectory, these neutron supermirrors consist of multilayers with many different periods. Multilayers are also used for monochromators, which are used to select specific wavelengths of a beam. By choosing a certain period at a specified angle, only one wavelength fulfills the Bragg condition at reflection. Therefore, only a certain wavelength will be reflected at the specified angle. This technique is commonly used to filter specific wavelengths. 
Multilayers with one magnetic material can be used to create neutron polarisers, filtering a certain spin direction of the neutron beam. The total neutron scattering potential has a contribution from the nuclear and magnetic scattering length, whether the magnetic scattering length subtracts or adds to the total scattering length depends on the polarisation state of the incoming neutron beam \cite{magneticscattering}. Using the right set of materials, the scattering contrast will disappear for one spin-state only, making the multilayer transparent for that spin-state while the other spin-state is reflected. Polarized neutron beams can be used to investigate magnetic properties of materials. Cold neutron beams may also be polarised using $^3$He filters, but a better performance can be achieved when the required angular acceptance of the beam is narrow \cite{depolarization_thierry}. 
The performance of multilayer components in terms of reflected intensity is highly dependent on both the scattering contrast and the interface width between the layers. The scattering contrast is limited by the materials of choice, while the interface width can be reduced using the correct deposition techniques. Reducing this interface width offers great possibilities in terms of performance for multilayer components. The obtained reflectivity depends exponentially on the achieved interface width. Meaning even a small improvement will lead to a drastic increase in reflectivity performance. This means that the total flux at experiment can be increased very significantly by improving the instrumentation, without the need of more power-intensive neutron sources. Another advantage of a reduced interface width is that the minimum thickness that can be achieved is reduced as well. The total thickness of a layer cannot meaningfully be less than the width of the interfaces. If the interfaces can be made flatter and more abrupt, the layers can be made thinner as well. This makes it possible for instance to deposit monochromators that can be applied for higher energies, making it possible to reflect neutrons in the hot and epithermal regime corresponding to shorter wavelengths.
\section{Research aims}
Neutron scattering is limited by the attainable neutron flux, even modern neutron sources still have a flux that is orders of magnitude lower than the attainable flux for X-ray synchrotron sources \cite{neutron_flux}. The biggest improvement in terms of neutron flux at experiment is not expected to come from more brilliant neutron sources, but instead from the improvement of different neutron optical components \cite{improvement}. The essential goal of this work is to improve the performance of neutron multilayers used in such sources by reducing the interface width between the layers.  As multilayers are such essential and crucial elements in any of the intended neutron instrumentation, even the slightest improvements of the performance will have an immediate and large impact on all research conducted using those instruments. The work presented is mainly focused on the simulation and modeling of neutron multilayers. The underlying goal here is to develop an understanding of the interface evolution of the samples, and how their performance is affected by different parameters. 
\section{Outline of the thesis}
This thesis will start with a theoretical background behind this work before moving slowly towards the more practical concepts such as multilayer growth and characterization. Starting with chapter \ref{scatteringtheory}, we will take a look at the general scattering principles and what physical principles actually lead to neutron scattering itself. The optical theory that describes specular scattering for multilayers is covered in chapter \ref{multilayeroptics}. How to characterize roughness and interface morphology is covered in chapter \ref{off_specular_section}, which covers off-specular scattering. In this chapter we will go into correlation functions and how to interpret an off-specular measurement of a multilayer. The growth of multilayers in this work is covered in chapter \ref{multilayerdepositions}. This chapter will also describe the sputtering techniques, and how these are used to minimize the interface width between the individual layers. The instrumental aspects for the reflectivity are described in chapter \ref{instrumentalaspects}, where both neutron and X-ray reflectometry will be described. The theory behind the reflectivity simulations that are used for the characterization of structural parameters is described in chapter \ref{reflectivitysimulations}. The other techniques that have been used for multilayer characterizations are covered in chapter \ref{multilayercharacterization}. A summary of the results will be given in chapter \ref{summaryresults}, followed by an outlook in chapter \ref{outlook}. The summary of the papers in this work is covered in chapter \ref{summarypapers}, while the papers themselves can be found after the bibliography.