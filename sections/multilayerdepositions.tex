\chapter{Multilayer depositions}\label{multilayerdepositions}
In order to create high quality multilayers a proper understanding of the growth technique is necessary.  In this chapter we will explain the techniques used to grow the samples discussed in this work, starting by an introduction on magnetron sputter deposition itself.

\section{Magnetron sputter deposition}
One of the most popular techniques to grow multilayers is physical vapor deposition (PVD) \cite{alvarez_garcia-martin_lopez-santos_rico_ferrer_cotrino_gonzalez-elipe_palmero_2014}.  Magnetron sputter deposition is a common type of PVD, which is used to grow the samples in this work. The basic working principle behind sputtering is intuitively relatively simple. A steady flow of a noble gas such as Argon is led into a vacuum chamber. Meanwhile, a bias is applied to the sputtering target, which contains the material that will be deposited upon the substrate. Stray electrons near the target are accelerated towards the substrate due to the electric field that is present, colliding with neutral atoms from inlet sputtering gas on the way. If these electrons have gained sufficient energy upon collision, it may knock off an electron from the sputtering gas, converting it in to a positively charged ion  \cite{ohring}:
\begin{equation}
	\text{e}^{-} + \text{Ar} \rightarrow 2\text{e}^{-} + \text{Ar}^{+}.
\end{equation}
Note how two additional electrons are released during this ionization process, these too are accelerated by the electric field and can bombard additional gas atoms generating even more free electrons. Meanwhile the ionized atoms, which are positively charged, will be attracted to the negatively charged sputtering target. Upon collision, both target atoms as well as secondary electrons are ejected. The secondary electrons will be accelerated and ionize even more neutral gas atoms while part the ejected target atoms will land on the substrate surface, slowly creating a uniform layer of the target material. 
\begin{figure}
	\centering
	\def\svgwidth{\textwidth}
	\input{magnetronsputtering.pdf_tex}
	\caption{A simplified sketch of magnetron sputtering. Incoming Argon ions collide upon the sputtering target, knocking out target atoms which in their turn form a new layer of the material on the substrate.}
\label{magnetronsputtering}
\end{figure}
This state of ionized gas is known as a plasma, and is generally considered as an own state of matter. Upon the electron excitation of the sputtering gas, photons will be emitted which gives the visible plasma a visible glow. In order for the plasma to be sustained, there has to be a sufficient amount of collisions between electrons and sputtering atoms. The secondary electrons are therefore important to sustain the plasma \cite{ohring}.


In order to maximize the amount of secondary electrons that can be used for the ionization process magnets are placed behind the target material. The force that acts upon charged particles can be described using the well-known Lorentz force:
\begin{equation}
	\vb{F} = q(\vb{E} + \vb{v} \cross \vb{B}),
\end{equation}
where q is the charge of the particle, $\vb{E}$ the electric field, $\vb{v}$ the velocity of the particle and $\vb{B}$ the magnetic field. The path that the electrons will follow therefore depends on the magnetic and electric fields, a depiction of a typical magnetic configuration in the used deposition systems is shown in Figure \ref{magnetronsputtering}, where the magnetic field is orientated radially above the target surface. The resulting path will follow a trajectory according to the cross-product  $\vb{E} \cross \vb{B}$, which gives rise to a circular trajectory above the target surface. This preferential trajectory for the electrons results that more Argon ions will be ionized in this region, leading to a denser plasma. It is for this reason that a circular erosion zone appears on used targets \cite{sputterprocess}. 

\section{Ion assistance}\label{ion_assistance}
In order to grow as smooth and abrupt layers as possible, it is important to consider the energy of the incoming adatoms on the multilayer that is being grown. Limited adatom mobility is known to lead to rough interfaces \cite{rough_morphology}, when incoming adatoms have insufficient energy to migrate from their landing sites, they are statistically likely to be at a position that does not contribute to a smoother surface \cite{ERIKSSON200684}, leading to a rougher interface over time. Furthermore, roughness that is already present at the surface will not be smoothened out, but instead the existing interface profile will be replicated throughout the multilayer. Such growth with low adatom mobility will therefore result in rough surfaces with accumulated roughness \cite{thesis_fredrik} where  the total interface width increases over time. In order to grow smooth layers, it is therefore important to have enough adatom mobility to allow for surface migration such that adatoms can move to a local energy minimum with many bonds to surrounding atoms that smoothens the interfaces \cite{ERIKSSON200684}. \\
One technique that can be applied to increase adatom mobility it to make use of the ions that are available in the plasma during film growth \cite{thesis_kenneth}. By applying a negative bias on the substrate, there will be a significant potential drop towards the substrate such that the ions will be accelerated towards the film with an energy proportional to the applied substrate bias voltage. When an ion reaches the medium, it interacts with it resulting into different mechanisms of energy and momentum transfer. Given enough energy, this bombardment of ions on the film surface will lead to surface displacement of the adatoms that are present on the surface. This is further illustrated in Figure \ref{layergrowth}, where the ion assistance allows adatoms on the target to move from their landing sites leading to smoother layers. If the ion energy becomes to large however, it can lead to bulk displacement where the adatoms get knocked into the bulk, leading to intermixing between the interfaces. This gives us a clear energy window, where the energy has to be high enough to allow for surface displacements, but low enough to prevent bulk displacement \cite{ERIKSSON200684}. Ideally, all adatoms should be displaced from their landing site, requiring a relatively high ion flux. A magnetic coil is therefore used in order to increase the ion density near the substrate. The magnetic field allows for secondary electrons near the target to be guided from the magnetron source to the sample, surface further ionizing neutral atoms along the way, which therefore leads to a higher incident flux of ions near the sample substrate. 
\begin{figure}
	\centering
	\def\svgwidth{\textwidth}
	\input{layer_growth.pdf_tex}
	\caption{a) Film growth without ion assistance. Due to the lower adatom mobility, adatoms stick directly to their landing sites, forming a rough surface. b) Ion-assistance during layer growth increases adatom mobility, making surface migration possible. Note how the ions are not part of deposited layer, instead they recoil in the form of neutral gas atoms \cite{thesis_naureen}.}
	\label{layergrowth}
\end{figure}
Even in the case of mere surface displacements however, a certain intermixing is still present at the interfaces, as incoming adatoms are still allowed to be displaced into the surface of the top layer leading to a minor intermixing. While this intermixing effect is mainly limited to the first atomic layer, the multilayers in this work have an interface width that is roughly in the order of the atomic spacing. Such small displacements can therefore lead to a significant increase in the total interface width and should therefore be avoided.
\\
In order grow smooth and abrupt interfaces, a modulated ion assistance scheme is used during the growth of the deposited multilayers. In this set-up, an initial layer is grown with a low bias at the substrate. This layer initial layer is therefore grown with a very low adatom energy, leading to relative rough growth at a lower density, but without any intermixing. After an initial part of the layer is grown using this low-bias scheme, the bias is increased in order to allow for surface migration when the adatom reaches the surface, smoothening the rest of the layer. While there will be intermixing at this higher bias, the intermixing will be limited to the initially grown layer which is the same material. The initially grown layer therefore acts as a buffer layer, to protect the lower layer against intermixing. This process with an alternating bias scheme is repeated throughout the deposition of the entire multilayer, leading to smooth and abrupt surfaces \cite{thesis_fredrik}. This scheme is illustrated in Figure \ref{splitbias}
\begin{figure}
	\centering
	\def\svgwidth{\textwidth}
	\input{splitbias.pdf_tex}
	\caption{a) An initial buffer layer is grown using a grounded substrate, providing a low adatom-mobility. This results in a rough initial layer, but without intermixing. b) The rest of the layer is grown with a higher substrate bias, giving the adatoms enough energy to migrate. the buffer layer prevents intermixing into the layer beneath. The different colours for the initial and final layers are for the purpose of clarity, they are the same material.}
\label{splitbias}
\end{figure}
\section{$^{11}$B$_4$C co-deposition}
When growing multilayers using magnetron sputtering, nanocrystallites tend to form throughout the sample. These crystallites lead to faceted interfaces between the layers which  can contribute to an increased interface width. Another contributing factor to the interface width is the formation of intermetallics between the layers, decreasing the abruptness between the layers. It would therefore be beneficial to eliminate the formation of crystallites by growing an amorphous multilayer instead, which is achieved using the incorporation of \BC during growth. \\
The addition of boron to the layer inhibits the formation of crystallites, which has shown to succesfully amorphize the entire multilayer stack \cite{characterization_paper}, \cite{morphology_paper}. Using a continuous deposition of $^{11}$B$_4$C during the  growth process, boron is added through the entire multilayer. For neutron multilayers it is important to consider that the the boron-10 isotope, which is present in natural boron, absorbs neutrons.  The boron-11 isotope however is transparent to neutrons, it is for this reason that an isotope enriched $^{11}$B$_4$C target is used during the deposition. The power to the $^{11}$B$_4$C magnetron is constant during the entire deposition process, resulting in a stable $^{11}$B$_4$C-flux during film growth.  \\
\\
The incorporation of \BC also has a few detrimental side-effects on the multilayer stack. In terms of layer growth, the incorporation of \natBC has shown to reduce adatom mobility \cite{b4c_effects}, which can lead to accumulated roughness as described in previous section. Investigations performed in this work has also shown that the incorporation of \BC leads to mounded interfaces with a stronger vertical correlation between the interfaces \cite{GISAXS_paper}, which in turn can contribute to an accumulated roughness. Both these mounded features as well as the correlation between the interfaces can mostly be eliminated using a higher ion energy during growth, but this can come at the expense of intermixing. The formation of interface mounds as a result of lower adatom mobility is consistent with literature \cite{Pelliccione}, and is mostly caused by a shadowing effect. As the incoming adatom flux during deposition has an angular distribution, taller features at the interfaces can block the ion flux at the other side of the magnetron allowing these features to grow leading to such mounds. It is for this reason that multilayer is rotated during deposition.  Another detrimental effect of \BC incorporation in the multilayer, is that this leads to a dilution of the SLD-contrast between the interfaces, reducing the ultimate reflectivity performance. The power that is used for the  $^{11}$B$_4$C-target is therefore tweaked such that it is high enough to make the entire sample is amorphous, but not increased any further to prevent unnecessary dilution in SLD-contrast between the interfaces. 
\section{Growing high-performance neutron multilayers}
The essential goal in this project is to increase the reflectivity performance of neutron multilayers, which is done using a combination of the techniques described above. The ultimate reflectivity performance of multilayers depends on the achieved interface width, and on the optical contrast in terms of SLD between the layers \cite{contrast_interface_width}. 
\subsection{Reducing interface width}
In order to grow high-performance neutron multilayers, it is crucial to consider the interface width that can be achieved in the multilayer stack. Since the reflectivity performance shows an exponentially squared dependence on the interface width, even a minor improvement can have a large impact on the total reflectivity. The biggest contributors to the interface width are crystallites and intermetallics \cite{characterization_paper}. The co-deposition of \BC gets rid of both these factors by preventing the the formation of intermetallics at the interfaces, and amorphizing the multilayer stack. However, this co-deposition of \BC also significantly reduces adatom mobility during the growth process, which leads to correlated interfaces and can result in a strong accumulation of interface width over time. This makes it essential to combine the \BC incorporation with ion assistance in order increase the adatom mobility. A too high ion energy will result in bulk diffusion however, which results in intermixing between interfaces. It is for this reason that the described modulated ion assistance scheme is used. \\
This therefore results in a deposition scheme where \BC incorporation is combined with a modulated ion assistance scheme to get a maximum effect.
\subsection{Optimizing optical contrast}
Apart from the interface width, the optical contrast between the layers in the multilayer is important as well. In particular for thicker layers the optical contrast will be the biggest factor for a good reflectivity performance, as the interface width becomes less relevant for thicker layers. One major disadvantage of \BC incorporation is that this dillutes the optical contrast, and for thicker layers the improvement in interface width is therefore not sufficient to compensate for the negative effects of the contrast dillution. When applying the technique of \BC incorporation it is therefore important to take the thickness of the layers into account. For thicker periods it may be more beneficial to opt for a material system where the optical contrast is preserved. A popular technique for this is the use of ultra-thin interlayers at the interfaces \cite{interlayers1} \cite{interlayers2}, \cite{interlayers3}, which would prevent diffusion between the layers. This technique has shown work to lead to significant improvement for multilayers at a period of $\Lambda$ = 83 Å in this work, while at a period of $\Lambda$ = 48 Å and below \BC incorporation showed the best results \cite{article4}. The reason for the difference in performance for this periodicity is because the importance of interface width depends on the multilayer period. Another approach that has shown to be effective is to incorporate \BC in the Ni layer only. Due to close match in neutron SLD of Ni and \BC ($9.1\cdot 10^{-6}$ Å$^{-2}$ and $9.4 \cdot 10^{-6} Å^{-2}$ for \BC and Ni respectively \cite{periodictable}), the optical contrast for neutrons is preserved for this system while the Ti layer is allowed to grow amorphously. Crystallites are still present however in the Ni layer, and the total achieved interface width is significantly higher than that of a multilayer with \BC deposited throughout the entire multilayer stack, making this system less effective at thinner periodicities.
\clearpage
\section{Deposition systems used in this work}\label{petraIII}
The samples that are used early during this work have been deposited at at a ultra-high vacuum (UHV) magnetron sputtering chamber at PETRA III at the Deutsches Elektronen-Synchrotron in Hamburg, Germany \cite{petraiii}.  The system is mounted on a 1-tonne ultra-high load, high resolution hexapod in the beamline allowing for fine alignment of the sample with respect to the synchrotron beam. In this setup, it is possible to do in-situ X-ray scattering measurements during film growth. The cylindrical deposition chamber has a diameter of 600 mm and four 75-mm diameter sputter sources tilted with an angle of 35$\degree$ towards the substrate normal. Between the sputter sources, a $\mu$-metal shielding is placed in order to extend the magnetic field closer to the substrate and minimise cross contamination. In front of the sputter sources, fast-acting shutters are mounted to control the sputtered flux. Enabling the growth of single layers, multilayers as well as the co-deposited samples grown in this work. The substrate is rotated during deposition, during this work at a constant rate of 7 rpm. The substrate table is electrically isolated in order to enable a negative substrate bias during film growth. The deposition chamber also allows for substrate heating during deposition, which has not been used during the depositions performed in this work. \\
Most samples in this work have been deposited locally at Linköping University. The system used for this work is an ultra-high vacuum (UHV) magnetron sputtering system. Similar to the system at PETRA III, in this setup, two 3-inch magnetron targets are mounted at an angle of 35$\degree$ towards the substrate normal, while a $\mu$-metal shielding is used to extend the magnetic field closer to the substrate and minimise cross-contamination. All magnetrons have a fast-acting shutter, which can be controlled using home-made software written in LabView, allowing for novel designs of multilayer systems. The substrate is rotated at a rate of 7 rpm. Unlike the system at PETRA III, the system at Linköping University also uses a magnetic coil to allow for a dense plasma at near the substrate, allowing for a high ion-flux at a lower ion-energy. 