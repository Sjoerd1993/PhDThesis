\chapter{Summary of the papers}\label{summarypapers}
\section{Paper I}
\subsubsection*{Layer Morphology Control in Ni/Ti Multilayer Mirrors by  Ion-assisted Interface Engineering and  B$_\textrm{4}$C Doping}
Ni/Ti is the materials system of choice for broadband neutron multilayer supermirrors. The reflected absolute intensity as well as neutron energy range from state-of-the-art mirrors are hampered by a Ni/Ti interface width, typically 0.7 nm (caused by nanocrystallites, interdiffusion, and/or intermixing), limiting the optical contrast across the interface as well as limiting the minimum usable layer thickness in the mirror stack. In this work we explore elimination of nanocrystallites by amorphization through boron-carbon doping in combination with interface smoothening by modulation of ion-assistance during magnetron sputter deposition of individual Ni and Ti layers, ranging from 0.8 nm to 6.4 nm in thickness. In situ high-energy synchrotron wide angle X-ray scattering (WAXS) revealed an effective hindering of Ni/Ti crystallization through a minute concurrent \natBC flux during growth. Post-growth X-ray reflectivity (XRR) confirmed the incorporation of \natBC but also showed that interface widths deteriorated when a constant substrate bias of -30 V was used. However, XRR showed that interface widths in \natBC-doped multilayers improved significantly, compared to those in pure Ni/Ti multilayers, by employing a two-stage substrate bias, where the initial 1 nm of each layer was grown with -30 V substrate bias, followed by -100 V bias for the remaining part of the layer. The present results shows that B4 C doping of Ni/Ti multilayers leads to significantly smaller interface widths when combined with engineered interfaces through temporal control of the substrate bias. Consequently, a significant improvement of neutron supermirror performance can be expected by emplying this technique using $^{11}$B isotope-enriched $^{11}$B$_4$C source material.
\clearpage
\section{Paper II}
\subsubsection*{Synthesis of $^{\textrm{11}}$B$_\textrm{4}$C containing Ni/Ti multilayers and characterization using combined X-ray and neutron reflectometry}
In this work we investigated a novel magnetron sputter growth technique where \BC co-deposition has been combined with a split-bias scheme in order to reduce interface width at Ni/Ti multilayers. Combined fits to experimental XRR and NR data indicate average interface widths of 4.5 Å without any accumulated roughness when the sample is co-deposited with \BC . This is a significant reduction from the deposited multilayer with pure Ni/Ti, which shows an average initial interface width of 6.3 Å at the first period with an accumulation of 0.04 Å per period. This improved interface width offers very good prospects
for high-performance supermirrors with high-m values, as well as for neutron instruments for thermal neutrons where thinner interface widths are required. Since neutron multilayers are such essential and crucial elements in any intended instrumentation, even the slightest improvement of the performance will have an immediate and large impact on all research using those instruments.

\section{Paper III}
\subsubsection*{Morphology of buried interfaces in ion-assisted magnetron sputter deposited \BC-containing Ni/Ti multilayer neutron optics investigated by grazing incidence small angle scattering}
The effect of \BC co-deposition and ion assistance on the morphology of buried interface of Ni/Ti neutron multilayers has been investigated using Grazing Incidence Small Angle X-ray Scattering (GISAXS) measurements. The co-deposition of \BC is known to amorphize the multilayers, eliminating crystallites and intermetallics at the interfaces which otherwise contribute to the interface width. The GISAXS analysis revealed that the layers become more strongly correlated and interface mounds form for increasing amounts of $^{\textrm{11}}$B$_\textrm{4}$C. As the adatom mobility decreases, the characteristic separation between the mounds decreases, indicating an increase in the density of mounds for increasing amounts of $^{\textrm{11}}$B+C.
By applying high flux ion assistance during growth, the adatom mobility can be increased, reducing mound formation. However, this comes at the expense of a forward ion knock-on effect, which can lead to interface mixing. To prevent intermixing, a high flux modulated ion assistance scheme was used, where an initial buffer layer was grown with low ion energy and the top of the layer with higher ion energy. X-ray reflectivity measurements showed that intermixing is still possible if the applied ion energy is too high. Therefore, a careful balance between the different growth parameters is necessary to maximize the reflectivity potential.
The optimal condition was found to be adding 26.0 at.\% $^{11}$B+C combined with high flux modulated ion assistance. In each layer, the applied substrate bias voltage was initially kept grounded for 3 Å and then increased to -30 V. Such a multilayer, with a period of 48.2 Å and 100 periods was grown, and the resulting structure was investigated by coupled fitting to neutron and X-ray reflectivity data. The average interface width was found to be only 2.7 Å, which is a significant improvement over the current state-of-the-art commercial Ni/Ti multilayers. These findings provide very promising prospects for high reflectivity neutron optics, including periodic multilayers as well as broadband supermirrors for hot and epithermal neutrons.

\section{Paper IV}
\subsubsection*{Material design optimization for large-m  \BC-based Ni/Ti supermirror neutron optics}
This work investigated several approaches in order to optimize the material design for \BC containing Ni/Ti supermirrors. At low scattering vectors, corresponding to thicker periods, the ultimate reflectivity performance is strongly dependent on the optical contrast. Conversely, at higher scattering vectors, corresponding to thinner multilayer periods, the reflectivity performance is mostly dependent on the achieved interface width. This trade off has clear implications for neutron supermirrors, where multiple layer thicknesses are present in the entire stack. In this work, strategies that focus on the optimization of the achieved optical contrast show very good result at the low-q regime, but at high-q values the interface width becomes more important. Conversely, minimizing the interface width has diminishing returns at higher periodicities, where the achieved optical contrast is more important.\\
Different designs that focus on scattering length density contrast or on a reduction of the interface width for multilayer reflectivity have been investigated, for multilayer periods of 30 Å, 48 Å, and 84 Å, for designs involving pure Ni/Ti multilayers, multilayers with \BC co-deposited in both Ni and Ti layers, multilayers with \BC co-deposited only in Ni layers, and multilayers with \BC deposited as thin interlayers between Ni and Ti layers. Grazing incidence small angle X-ray scattering, neutron and X-ray reflectivity measurements, reflectivity fitting, and transmission electron microscopy has been used to study the structure and morphology to find an optimal layer structure for supermirrors. \\
The results show that Ni/Ti based multilayers where thin (0.15 nm) interlayers of \BC has been co-deposited between the interfaces has the best reflectivity performance at the investigated period of 84 Å, while the multilayers where \BC has been co-deposited throughout the entire multilayer stack shows the best performance at the investigated period of 48 Å and 30 Å. This suggests that a higher reflectivity than that of state-of-the-art Ni/Ti multilayers can be achieved over the entire scattering vector range by applying a depth-graded material design, as demonstrated using simulated neutron reflectivity performance for a supermirror. Specifically, this can be accomplished by incorporating \BC in the Ni and Ti layers of thinner layers below approximately 26 Å and introducing interlayers of \BC between the Ni and Ti layers in the thicker layers. 

\section{Paper V}
\subsubsection*{Chemically homogeneous boron carbide $^{10}$B/$^{11}$B isotope modulated neutron interference mirrors}
In this article, we introduce a novel type of neutron based on chemically homogenous \natBC mirrors with $^\textrm{10}$B/$^\textrm{11}$B  isotope modulation. Simulations predict that these mirrors exhibit very high neutron reflectivities for a small number of bilayer periods. This is experimentally confirmed by neutron reflectivity measurements of mirrors synthesized by ion-assisted magnetron sputter deposition. For example, a 120 nm thick multilayer consisting of just 20 bilayer periods of $^\textrm{10}$B$_\textrm{5.7}$C/$^\textrm{11}$B$_\textrm{5.7}$C, with a periodicity of 61.5 Å, exhibits a neutron reflectivity of 13\% at an incidence angle of 4.7$\degree$ for neutrons with a wavelength of 4.825 Å. This is attributed to a high scattering length density contrast between the layers with an interface width <5Å. Structural analyses by X-ray diffraction, X-ray reflectivity, and transmission electron microscopy demonstrate that the $^\textrm{10}$B$_\textrm{5.7}$C/$^\textrm{11}$B$_\textrm{5.7}$C multilayer mirrors are composed of amorphous, chemically homogenous B$_\textrm{5.7}$, without any internal chemical modulation. The data show that $^\textrm{10}$B$_\textrm{x}$C/$^\textrm{11}$B$_\textrm{x}$C multilayer mirrors have the potential for higher neutron reflectivities at higher q-values using fewer and thinner layers, compared to today’s state-of-the-art chemically modulated neutron multilayer mirrors.