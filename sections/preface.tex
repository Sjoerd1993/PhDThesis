\chapter{Acknowledgments}

As I am writing this section of the thesis, not long before it will finally go to print, it is difficult to believe that it is already time to reflect over my period as PhD student. Unlike the multilayers that have been grown in this work, it has not always been a perfectly smooth experience. Instead, as it usually goes for research, the ride represents more that of a rough and correlated structures with many bumps and valleys along the road. Many of such challenges were expected, others were not. Who would have thought five years ago that half of my PhD would be covered by a global pandemic. Nevertheless, it's exactly these struggles that make it such a rewarding experience. Overall it often feels like trying to solve a puzzle, with perpetually missing pieces. It has exactly been this puzzle solving that has attracted me to academia however, and I count myself lucky that I have stumbled into an occupation where I can really dig down in a subject just for the sake of finding things out. Whatever the future may hold in the long term, as long as I am able to continue on this track of solving my own little mysteries, I count myself a happy person.
\\
\\
The results that were obtained in this work would not have been possible with everyone that has helped me along the way. In no particular order, there's some specific people that I would like to mention. \textbf{Kenneth Järrendahl} was the person who initially responded to my email asking for a research project within the European Erasmus program back in 2016, I cannot understate the influence that this little project has had on my future. This was a time in my life where I was uncertain where I was heading to, and I would probably have skipped academia all together if it wasn't for the work at the time that helped me fall back in love with science as a whole. Similarly, I should mention  \textbf{Jens Birch} who has been my supervisor for my PhD project, and is the one who granted me this position in the first place. Needless to say I am thankful for the trust that has been put into me by giving me this opportunity in the first place. It's been a pleasure to work with Jens. It's rare to stumble upon somebody that has such a strong intuitive feeling as well as an extensive knowledge about pretty much everything in this field. All meetings with Jens have truly been extremely insightful one way or another. I'd also like to thank \textbf{Fredrik Eriksson} who acted as my main supervisor for most of this project. I have been impressed with the extensive help from day one.  Both his patience and pedagogical skills have been a huge help for me during my time as a PhD student. I have learned a lot in particular in the field of X-ray scattering, and therefore subsequently neutron scattering as well. But over all I would like to thank Fredrik for his kindness and understanding. When it feels like everything is going wrong and nothing ever works, it's a huge asset to have a supervisor that has always been positive and understanding. I am extremely grateful to have had Fredrik as my supervisor in these situations, where his interpersonal skills have been an extremely important factor in helping me continue forwards. I also want to thank \textbf{Naureen Ghafoor} for her involvement in this project. Naureen has been a very valuable contributor during project meetings and has been vital to foundational work in this project s well. Naureen's knowledge of multilayer growth, TEM imaging and research planning has been a great help over my period in general. I would like to thank all Jens, Fredrik and Naureen in particular for the help in the period leading up to the end of my thesis, where they've been helpful and contributing far outside of regular working hours to help me out. Furthermore, I should give a shout out to everybody working together on the Adam deposition system. Specifically my fellow PhD students \textbf{Anton Zubayer}, \textbf{Samira Dorri} and \textbf{Marcus Lorentzon}. Your help and contributions throughout this period has been meaningful to me and I look forward to keep in touch in future endeavors as well. Someone outside of LiU that I should mention is \textbf{Alexei Vorobiev}, who has been our contact person at ILL. Alexei has always been very helpful every time we've had neutron measurements, and has always been very hard-working and constructive during our beamtimes. SuperADAM is both an impressive and important, and the work that Alexei and the rest of the SuperADAM team has put into the instrument cannot be understated. I also want to thank Alexei and the SuperADAM team for the warm welcome during my extended stay in 2019. It's been a very informative period where I have learned a lot. I should also mention SwedNess as a whole and all people involved for setting up these projects and opening up a plethora of opportunities. Finally I would like to say a word about the people that supported outside of my academic environment First my family back home both in the Netherlands and elsewhere. I know I have not been around as much as I should, and more frequents trips back home are definitely on the agenda in the future. Finally of course I need to mention my wife \textbf{Karin Stendahl}. I can't express how lucky I am to have you around. The last few weeks in particular have been a hectic period, and I could not have done this all without your love and support. I am forever grateful for making me a better version of myself. My chaotic nature is well known to all that are close to me, having you in my life helped me tremendously in keeping me on track and follow my much needed routines. I cannot express how much your love and companionship means to me, having you around when I come home after a long day means the world to me. I know you are smarter than me, but it remains a mystery to me how I haven't bored you to death with my eternal chatter about pretty much everything that is usually only of interest to myself. 
\\
\\
In the end it would be nearly impossible to mention every single person individually and I'm certainly selling people short by not naming them, but I am very thankful for everybody at the department. Both for the meaningful discussions, but especially for the meaningless ones as well. I like to think that we should never underestimate the importance of non-important matters, a friendly working environment is vital to a satisfying output.
%\vskip\onelineskip
%\begin{flushleft}
%    \sffamily
%    \uiocolon\textbf{\theauthor}
%    \\
%    Norrköping,\MONTH\the\year
%\end{flushleft}