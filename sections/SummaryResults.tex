\chapter{Summary of the results}\label{summaryresults}
\section{Multilayers grown at PETRA III}
The multilayers that were covered over the first half of this project were primarily grown at a sputter deposition system at the PETRA III synchrotron in Hamburg, further described in subsection \ref{petraIII}, these are the samples that are used when writing my licentiate thesis. The multilayers grown at this system were deposited using an initial bias at -30 V to grow a buffer layer followed by a high bias at -100 V for the rest of the layer. Both multilayers with natural \natBC as well as multilayers with isotope-enriched \BC have been deposited at Petra III. The magnetron powers that were used are 20 W, 60 W and 35 W for the Ni, Ti, and \natBC target respectively.\\
\subsection{Initial investigation with \natBC containing Ni/Ti multilayers}
As a proof of concept, Ni/Ti layers were co-deposited with natural \natBC at Petra III in order to eliminate the formation of nanocrystallites, this has been combined with a with a modulated ion assistance scheme to smoothen the interfaces. In this scheme, an initial layer with a thickness of 1 nm is grown using a substrate bias of -30 V, the remainder of the layer is grown with a substrate bias of -100 V. In situ high-energy synchrotron wide angle X-ray scattering (WAXS) showed an effective hindering of nanocrystallites during growth as a result of the \BC co-deposition. After deposition, the \natBC containing films have been characterized using X-ray reflectivity, which showed a significant improvement compared to the pure Ni/Ti multilayers.

\subsection{The effects of \BC co-deposition on reflectivity performance}
As the primary goal of this work has been to improve the performance of neutron multilayers, the study of specular reflectivity has been a major focus during this project. In order to investigate how the co-deposition of \BC affects the total reflectivity, both multilayers with and without \BC co-deposition have been deposited in this work. To obtain information about the elemental composition for the multilayer ERDA measurements were performed. The atomic percentage of $^{\textrm{11}}$B + C was measured to be about 3.6\% C and 21\% $^{\textrm{11}}$B using ERDA. The measured multilayer also contains about slightly less than 1\% Ar, showing that a small percentage of the sputtering gas that is used during deposition is present in the deposited multilayer. The B to C ratio is slightly higher than expected from a pure \BC target. The reason behind this remains unknown, but these findings have been confirmed using Rutherford Backscattering Spectroscopy (RBS) as well. A possible explanation could be an unexpected elemental composition in the target that was used.
\\
\\
Structural parameters of the samples have been obtained by a fitting the measured neutron reflectivity and X-ray reflectivity data to a modelled simulation of the multilayer. In this model, any accumulation of interface width is assumed to be linear and equal for both multilayers. Such characterization showed a good fit on the pure Ni/Ti multilayers with an averaged initial width of 6.4 Å at the interfaces, with an increase of 0.03 Å per bilayer, meaning the total interface width has grown to 9.4 Å after 100 bilayers. The specular diffraction pattern shows a slight broadening of the Bragg peaks, indicating an increase in layer thickness throughout the layer. The simulated fit to the experimental data shows an increase of 0.002 Å per bilayer. The obtained period was 45.8 Å. The sample has been measured with TEM microscropy as well. The TEM measurement confirms the findings from the model simulation, with abrupt interfaces and relatively limited roughness accumulation.\\
Structural parameters of the samples that were co-deposited with \BC have been obtained similarly by a combined fit of NR and XRR data on a model simulation of this multilayer. The same model has been used for these samples as for the pure Ni/Ti samples. The SLD has been calculated using the elemental composition obtained by ERDA where the density for each layer has been calculated using a simple rule of mixtures for the bulk densities for each element. The resulting fit to this model does not show any roughness accumulation throughout the sample. The obtained interface width has an average value of 4.5 Å for the multilayer, which is a clear improvement when compared to the pure Ni/Ti samples. The specular diffraction pattern shows a slight broadening of the Bragg peaks, indicating an increase in layer thickness throughout the layer. The simulated fit to the experimental data shows an increase of 0.002 Å per bilayer. The obtained period was 47.0 Å. These samples were measured with TEM as well, the TEM micrographs for this sample are shown earlier in Figure \ref{TEMfig}. These measurements are in line with the obtained fit to the simulated model, with very abrupt layers and no obvious sign of an increase in interface width throughout the layer. The resulting neutron
and X-ray reflectivity measurements are shown in figure \ref{neutronimprovment} for a Ni/Ti multilayer with and without \BC incorporation. The inset shows the off-specular rocking curves at the first Bragg peak for both multilayers. An interesting observation is how the \BC containing multilayer has a higher diffuse signal than the pure Ni/Ti multilayer, despite having a lower interface width. This phenomenon has later been investigated using GISAXS, where it has been shown that the \BC co-deposition leads to the formation of mounds at the interfaces with vertically correlated interface profiles.
\begin{figure}[b]
	\centering
	\def\svgwidth{\textwidth}
	\input{petra_results.pdf_tex}
	\caption{a) The obtained neutron reflectivity signal for a Ni/Ti multilayer with and without \BC incorporation. b) The obtained X-ray reflectivity signal for a Ni/Ti multilayer with and without \BC incorporation. A clear increase in reflectivity can be observed for the \BC containing multilayers, indicating a significantly lower interface widths. The insets show the rocking curves for both samples at the first Bragg peak, where the background level is indicated by a dashed line.}
	\label{neutronimprovment}
\end{figure}
\clearpage
\section{Samples grown at Linköping University}
The deposition system at Linköping University has a magnetic coil to attract electrons towards the substrate, creating a denser plasma near the substrate. This allows to increase the amount of ions that reach the sample during deposition, which makes it possible to use a lower substrate bias during growth. The The samples that were grown at Linköping University were generally grown with a grounded bias at 0 V for the initial stage, while the remainder of the layer was grown at a substrate bias of -30 V unless otherwise reported. The magnetrons at this deposition system where run in current mode using a current of 80 mA 160 mA for Ni and Ti respectively, while the \BC magnetron was run in power mode at a power of 30 W unless otherwise reported.
\\
\subsection{Reflectivity performance}
To obtain the optimal conditions, multilayers have been deposited at different magnetron powers ranging from P$^{\textrm{11B4C}}$ = 0 W to P$^{\textrm{11B4C}}$ = 70 W, corresponding to a total $^{\textrm{11}}$B + C content up to 41.2 at.\% as confirmed by ERDA. The substrate bias during the second stage of the modulated ion assistance scheme has furthermore been varied from 0V to -50 V on multilayers usign a magnetron power of P$^{\textrm{11B4C}}$ = 40 W. The best performing multilayers have been found to be around a magnetron power of P$^{\textrm{11B4C}}$ = 30 W, corresponding to a $^{\textrm{11}}$B + C content of 34.4 at.\%. Using a combined fitting of X-ray and neutron reflectivity measurements as shown in Figure \ref{genx_fits}, a total interface width as low as 2.7 Å has been observed at the optimal conditions. The high quality of the interfaces in these multilayers can also be seen from the high amount of Bragg peaks in both neutron and X-ray reflectivity measurements, with a total of 6 measured Bragg peaks in X-ray reflectivity, and four distinct Bragg peaks within the measured range in neutron reflectivity. The measured period was 48.2 Å with a total of N = 100 bilayers. No signs of accumulated roughness could be observed in these multilayers. The improvement over the multilayers that have been deposited at PETRA III show that the higher ion flux as a result of the magnetic coil has a positive effect on the quality of the interfaces.  
\begin{figure}
	\centering
	\def\svgwidth{\textwidth}
	\input{genx_fits.pdf_tex}
	\caption{Coupled fits to X-ray and neutron reflectivity measurements of \BC containing multilayers reveal extremely abrupt interfaces with an interface width of $\sigma$ = 2.7 Å.}
	\label{genx_fits}
\end{figure}
\\
\clearpage
\begin{figure}[b]
	\centering
	\def\svgwidth{\textwidth}
	\input{GISAXS_summaries.pdf_tex}
	\caption{a) The distance between the interface mounds decreases with increasing \BC contents, showing the addition of \BC leads to more mounded interfaces. b) Increasing the substrate bias during deposition increase the ion energy and therefore adatom mobility, which leads to fewer mounds at the interfaces.}
	\label{GISAXS_summaries}
\end{figure}
\section{Interface morphology of \BC containing multilayers}
The effects of both the \BC co-deposition and the modulated ion assistance has been investigated for the samples that were grown at Linköping University as well. The structural parameters in the growth-direction of the sample, such as periodicity, thickness drift and interface widths, have been investigated using combined fitting of X-ray reflectivity and neutron reflectivity data. These characterizations showed the highest achieved interface width of about 2.7 Å without any signs of interface width accumulation. \\
The morphology of the buried interfaces in the multilayers have been investigated using grazing incidence small angle X-ray scattering (GISAXS). The GISAXS analysis revealed that the layers become more strongly correlated and interface mounds form for increasing amounts of $^{\textrm{11}}$B$_\textrm{4}$C, caused by a reduction of adatom mobility when \BC is incorporated into the multilayer. As the adatom mobility decreases, the characteristic separation between the mounds decreases, indicating an increase in the density of mounds for increasing amounts of $^{\textrm{11}}$B+C, as shown in Figure \ref{GISAXS_summaries} a). Using ion-assistance during growth leads to fewer mounds at the interfaces, shown in Figure \ref{GISAXS_summaries} b). This indicates that a higher adatom mobility prevents the formation of mounds at the interfaces. The usage of the modulated ion assistance scheme further promotes the formation of interface mounds as well as vertical correlation between the interfaces when compared to a constant bias. This indicates that the second stage in the modulated ion assistance scheme does not completely repair the rough layer that is grown during the initial stage at grounded bias. Nevertheless, the modulated ion assistance scheme does show a higher reflectivity, which can be explained to a lower intermixing using this scheme. Using a higher ion assistance during the second stage further reduces the density of mounds and reduces the vertical correlation, but does not improve overall reflectivity.
\section{Thickness dependent material designs}
Simulations on a neutron supermirror using the obtained optical and structural parameters for a \BC containing multilayer indicate a signficant improvement in terms of reflectivity around the critical edge of the simulated supermirror. At lower angles of incidence however, a conventional Ni/Ti supermirror is expected to perform better due to the higher optical contrast that is achieved. \\
Since the importance of a low interface width decreases at thicker periods, it follows that the optimal material design depends on the multilayer period. Within this thesis, different approaches for Ni/Ti based neutron multilayers were investigated at different multilayer periods. It has been shown that at thicker period of $\Lambda = 84$ Å, multilayers where a thin layer of \BC has been deposited between the interfaces has the best reflectivity performance. At thinner periods of $\Lambda = 48$ Å and $\Lambda = 30$ Å the best result has been obtained by incorporating \BC throughout the entire multilayer stack. By combining these different approaches, where multilayers with \BC interlayers are grown for thicker layers corresponding to $\Lambda > 52$ Å, and \BC is co-deposited at thinner periods at $\Lambda < 52$ Å, a potential supermirror could be grown that shows a better reflectivity performance over the entire reflected range in $q_z$.
\section{Isotope modulated interference mirrors}
Apart from the work on Ni/Ti based multilayers, a series of chemically homogenous \natBC interference mirrors have been investigated as well. Similar to a regular multilayer, a periodic contrast in terms of scattering length density along the growth direction is obtained by growing subsequent layers of \10BC and \BC in a single stack, giving rise to a periodicity as a result of $^\textrm{11}$B/$^\textrm{10}$B  isotope modulation. The different mirrors have been deposited at a total thickness of 128 nm with $^\textrm{10}$B$_\textrm{4}$C/\BC periods $\Lambda$ ranging from 16 Å to 128 Å, with a thickness ratio of $\Gamma$ ranging from 0.125 to 0.875. Both sputter magnetrons were run in power mode using a constant 100 W power for each target during deposition. Similarly to the Ni/Ti based multilayers, a modulated ion assistance scheme has been used where the initial 3 Å of each layer was grown using a grounded substrate bias while the remainder of the layer was grown using a substrate bias of -30 V. 
\\
\\ Both XRD and TEM measurements showed no signs of crystallites, nor of any chemical modulation which has been further confirmed with XRD measurements. Given the high optical contrast between $^\textrm{10}$B  and $^\textrm{11}$B, a high reflectivity can be obtained for these multilayers with only few bilayers.  While the $^\textrm{10}$B isotope is absorbing, simulations have shown that this only forms a significant problem near grazing incidence angles where the optical path length is longer. Preliminary neutron reflectometry measurements were performed as well, showing a total of 10\% absolute reflectivity for a 128 nm thick multilayer consisting of only 20 bilayer periods with a periodicity of 33.1 Å, showing how only few layers are needed for a high reflectivity. Simulations to these measurements indicate very thin interface widths below 5 Å, showing a good promise for neutron optical components that are not used at grazing incidence.