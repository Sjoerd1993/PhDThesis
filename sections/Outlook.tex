\chapter{Outlook}\label{outlook}
\section{Ni/Ti based multilayers}
The results described in chapter \ref{summaryresults} show how a clear improvement has been achieved in terms of interface width by incorporating \BC into the multilayers. As is usual for research projects, there are still aspects where future research can be beneficial. A major challenge of the \BC incorporation into both multilayers for instance is that this dillutes the high optical contrast that is present in pure Ni/Ti. While the improvement in interface width leads to a signficantly higher reflectivity nonetheless, the contrast dillution makes this technique unsuitable for multilayers with thicker periods where the achieved optical contrast is more important than the interface width. This is in particular relevant for components where multiple layer thicknesses are present such as supermirrors used in neutron guides. Initial investigations have been performed to mitigate these effects, such as using a different material design in the thicker periods where the optical contrast is preserved. Further investigations would be needed however to expand this further, allowing for a more fine-tuned approach for multiple multilayer periods. \\
\\
A possible area where this project could be expanded is the use of different \BC contents in either layers. For the samples deposited in this work, a constant flux of \BC has been used during the entire multilayer growth. However, it has been observed how the Ti-layer is significantly easier to amorphize than the Ni layer, while the \BC incorporation in the Ti layer specifically reduces the contrast in neutron SLD. It would therefore be beneficial if as little \BC incorporation could be used in the Ti layer as possible. While Ni\BCs/Ti multilayers were shown to be significantly rougher than the multilayers where \BC has been co-deposited throughout the entire multilayer stacks, no investigations have been done using compound targets where a lower amount of \BC is used in the Ti layer than in the Ni layer. This could further improve the optical contrast of the \BC containing multilayers, while keeping the low interface width that is achieved in the material system with \BC in both interfaces. Such an increase in optical contrast would lead to further improvement regardless of the multilayer period.\\
\\
Nevertheless, the work that is done during this project so far already shows great promise for neutron instruments. In particular at higher incidence angles, corresponding to thinner periods, a great improvement in terms of reflectivity performance has been achieved. For supermirrors the high performance of conventional Ni/Ti at low incidence angles can be combined with the high performance of \BC containing Ni/Ti at higher incidence angles using a hybrid multilayer design where either material system is applied depending on the layer thickness. This is further illustrated in Figure \ref{outlook} where a simulated supermirror for such a combined design is compared to that of pure Ni/Ti in that is used today. The simulated interface width of 3 Å is comparable to that of the best multilayers that have been deposited in this work \cite{GISAXS_paper}. This shows a lot of potential specifically at higher incidence angles, which is further made possible by the ability to grow thinner layers. The general trend is towards neutron mirrors with larger m-values where thinner layers are needed \cite{large_m_value1}, \cite{large_m_value2}. The strong potential for \BC containing multilayers at thinner periods would further aid the development of high-m supermirrors where neutrons are reflected at higher scattering vectors. This would therefore not only increase the reflectivity compared to existing wave guide, but would also allow neutrons with higher energies to be transmitted, thereby extending the applicability of neutron waveguides into the regime of hot and epithermal neutrons \cite{morphology_paper}. \\ Most neutron optical components utilize multilayers in one way or another, and the achieved results so far clearly demonstrate that a large improvement of neutron flux at experiment can be achieved without building bigger and brighter sources. The improvement of such components that has been established in the latest research is severely underutilized at actual neutron sources. Given the sheer size of existing neutron sources both in terms of scientific impact and in terms of economy, industrial development of such improved multilayers would be a worthwile investment. In the realization of that step, further collaboration with industrial partners would be beneficial. Getting these technologies to scale at actual neutron sources benefits all users of neutron scattering, and thereby science and our collective knowledge as a whole. 
\begin{figure}
	\centering
	\def\svgwidth{\textwidth}
	\input{outlook.pdf_tex}
	\caption{The \BC containing multilayers show great promise for supermirror devices. When thicker layers are grown with a material system where optical contrast is maximized, a high reflectivity can be obtained over the entire reflected range. In this simulation, the thicker layers corresponding to a multilayer period of $\Lambda =$ 60 Å in the red curve are grown with pure Ni/Ti while the thinner layers are incorporated with \BC.}
	\label{outlook}
\end{figure}
\clearpage
\section{Isotope modulated interference mirrors}
The isotope modulated \BC mirrors in this work have preliminary been an explorative study, but they nevertheless show a promise of high quality neutron optical components where only a few layers are needed to get a strong reflectance. This high performance can be attributed to the high contrast in terms of SLD in combination with the amorphous nature of this material system. The optical contrast in terms of real SLD between  $^\textrm{10}$B and  $^\textrm{11}$B is higher than that of the \BC containing Ni/Ti multilayers in this work, showing great potential for future optical components. Some additional research need to be done for these designs in the future. A specific challenge the samples deposited in this work is that the deposited stack started peeling from the substrate, which likely caused by stresses in the multilayer stack. For future depositions it would therefore be worthwhile to preheat the sample substrate before deposition which would alleviate these problems. A further challenge with these multilayers is that $^\textrm{10}$B is an absorbing isotope for neutrons. Both simulations and experimental measurements indicate that this is not a significant problem for the reflectivity performance of most optical components in practice, but this does make this particular design unsuitable for instruments at grazing incidence where the optical path through the multilayer is long. A good performance is achieved however for a broad-band mirror where only a specific range within q needs to be reflected. As \natBC is an abundant material, these kind of chemically homogenous interference mirrors have a strong potential for future development of neutron optical components.