\chapter{Reflectivity simulations}\label{reflectivitysimulations}
In order to obtain structural parameters for the samples, the experimental reflectivity curves need to be fitted to mathematical simulations for each sample. There are several possible mathematical descriptions that can be used, within this work two different descriptions have been used to simulate scattering experiments. Specular reflectivity is used described using the Paratt recursion formula, while the off-specular simulation is described using the Born-approximation. Both of these formalisms are described in this chapter. Fitting of the specular parameters using Parrat recursion is done with the GenX software \cite{genx} \cite{genx_new}. Off-specular simulations in this work, which use the Distorted Wave Born Approximation, are done using the BornAgain software \cite{BornAgain}.

\section{Parratt recursion}
The specular reflectivity of the samples is simulated using the Paratt recursion formalism. This description recursively accounts for the reflection from each subsequent interface in order to simulate the total intensity at detector. For a multilayer with a certain amount of layers, the reflectivity for the $j$'th layer in the sample can be calculated using \cite{parratt_recursion}: 
\begin{equation}
	\chi_j = \frac{R_j}{T_j}  = \exp(-2ik_{z,j}z_j)\frac{r_{j,j+1} + \chi_{j+1} \exp(2ik_{z,j}z_j)}{1 + r_{j,j+1} + \chi_{j+1} \exp(2ik_{z,j+1}z_j)}
\end{equation}
$R_j$ and $T_j$ in this equation describe the reflected and transmitted amplitude for layer $j$. The fraction of these, $\chi_j$ therefore describes the normalized amplitude from each layer. The factor $r_{j,j+1}$ is the Fresnel coefficient for the interface, and can be written as:
\begin{equation}
	r_{j,j+1} =  \frac{k_{z,j} - k_{z,j+1}}{k_{z,j} + k_{z,j+1}}.
\end{equation}
It follows from this description that this formalism is recursive, the reflectivity of layer j+1 is required to calculate the reflectivity of the j'th layer. Assuming the substrate is thick enough, there will be no reflection from the bottom and we can therefore use this position as the bottom boundary where $\chi_{N+1} = R_{N+1} = 0$. The incidence wave is normalized to unity, which gives us $T_1$ = 1. Using these known conditions as a starting point, the total reflectivity for the entire multilayer can be calculated recursively. This formalism is commonly used to simulate the specular reflection for neutrons and X-rays in reflectivity simulation software.
\section{The Born approximation}\label{bornapproximation}
The Born approximation (BA) was originally proposed as early as 1926 by Max Born \cite{born_approximation}. To derive the Born Approximation, we may first start with the general time-independent Schrödinger equation:
\begin{equation}
	\qty(-\frac{\hbar^2}{2m} \nabla^2 + V(\vb{r}))\Psi(\vb{r}) = E \Psi(\vb{r}) .
\end{equation}
We can re-write this as:
\begin{equation}\label{helmholtz}
	\qty(\nabla^2 + k^2)\Psi(\vb{r}) = \frac{2m}{\hbar^2} V(\vb{r})\Psi(\vb{r}) .
\end{equation}
Where $k$ is defined as usual,
\begin{equation}
	k^2 = \frac{2mE}{\hbar^2}.
\end{equation}
If we could find a function $G(\vb{r})$ that solves equation \ref{helmholtz} with a point source, we obtain the expression:
\begin{equation}
	\qty(\nabla^2 + \vb{k}^2)G( \vb{r} -  \vb{r}') = \delta( \vb{r} -  \vb{r}').
\end{equation}
Which is also known as Helmholt'z function \cite{Griffiths_QM} which we can express the wave function $\Psi$ as an integral:
\begin{equation}\label{general_wave}
	\Psi(\vb{r}) = \frac{2m}{\hbar^2}\int G( \vb{r} -  \vb{r}')V(\vb{r})\Psi(\vb{r})\dd^3\vb{r}'.
\end{equation}
The factor G in this equation is known as Green’s function. We can show that this indeed satisfies Schrödinger equation in equation \ref{helmholtz}:
\begin{alignat}{2}\label{born_se}
	\qty(\nabla^2 + k^2)\Psi(\vb{r})  = && \frac{2m}{\hbar^2}\int \qty[\qty(\nabla^2 + k^2)G( \vb{r} -  \vb{r}')]V(\vb{r'})\dd^3\vb{r}', \\ 
  = && \frac{2m}{\hbar^2} \int \delta^3(\vb{r} - \vb{r}')V(\vb{r}')\dd^3\vb{r}'  =  \frac{2m}{\hbar^2}V(\vb{r}).
\end{alignat}
The Green's function for the Helmholtz equation takes the form of \cite{Griffiths_QM}
\begin{equation}
	G(\vb{r}) = -\frac{e^{ikr}}{4\pi r},
\end{equation}
wich we can use in equation \ref{born_se} in order to get the general integral form of the Schrödingers equation \cite{Griffiths_QM},
\begin{equation}
	\label{bornscatteredwave}
	\Psi(\vb{r}) = \Psi_0(\vb{r}) - \frac{m}{2 \pi \hbar^2} \int \frac{e^{i\vb{k}| \vb{r} -  \vb{r}'|}}{\qty| \vb{r} -  \vb{r}'|}V(\vb{r}')\Psi(\vb{r}')\dd^3 \vb{r}'.
\end{equation}
We can simplify this for the case where the potential is more localized and $\abs{\vb{r}} \gg \abs{\vb{r}}$. This allows us to make the approximation $\abs{\vb{r} - \vb{r'}} \approx \abs{\vb{r}}$ in the denominator. We cannot make the same approximation in the exponent, where we need to keep this next term, this may be unintuitive at first sight but makes mathematical sense if we perform a Taylor series on both components \cite{Griffiths_QM}. Mathematically, $\vb{r} \cdot \vb{r'}$ describes the projection of of $\vb{r'}$ onto $\vb{r}$, from simple geometry in Figure \ref{vector_math} it follows that we can further rewrite the term in the exponent of equation \ref{bornscatteredwave} as
\begin{figure}
	\centering
	\def\svgwidth{\textwidth}
	\input{vector_math.pdf_tex}
	\caption{The dot-product $\vb{r} \cdot \vb{r}'$ represents the projection of $\vb{r}'$ onto $\vb{r}$. Using geometry we can show how $\vb{r} \cdot \vb{r}' = \abs{\vb{r}} - \abs{\vb{r} - \vb{r}'}$.}
	\label{vector_math}
\end{figure}
\begin{equation}
	k \abs{\vb{r} - \vb{r}'} = kr - k \vb{r}\cdot\vb{r}' = kr - \vb{k}_f \cdot \vb{r}'.
\end{equation}
Furthermore, we can fill in $\Psi_0$, which can be written as a planar wave \cite{born_detailed}
\begin{equation}\label{planar_wave},
	\Psi_0(\vb{r})  = e^{i\vb{k}\cdot \vb{r}}.
\end{equation}
From which we can rewrite equation \ref{bornscatteredwave} as
\begin{equation}\label{born_equation_almost}
	\Psi(\vb{r}) = e^{i\vb{k}\cdot \vb{r}} -  \frac{m}{2 \pi \hbar^2} \frac{e^{ikr}}{r} \int e^{-i\vb{k}_f\cdot \vb{r'}}V(\vb{r}')\Psi(\vb{r}')\dd^3 \vb{r}'.
\end{equation}
This can be solved using a so-called Born series \cite{born_detailed}, where we can find the first-order Born Approximation by putting the solution $\Psi$ of a planar wave from equation \ref{planar_wave} in to equation \ref{born_equation_almost}:
\begin{equation}\label{born_equation_first_order}
	\Psi_1(\vb{r}) = e^{i\vb{k}\cdot \vb{r}} -  \frac{m}{2 \pi \hbar^2} \frac{e^{ikr}}{r} \int e^{-i\vb{k}_f\cdot \vb{r'}}V(\vb{r}')e^{i\vb{k}\cdot\vb{r}-\vb{k}_f\cdot \vb{r}}\dd^3 \vb{r}'.
\end{equation}
This equation corresponds to the first-order Born approximation. The second order can then be found by inserting this first order solution into equation \ref{bornscatteredwave}. By continuing this way, any order can be obtained, but for most cases only the first order approximation is actually used. The Born approximation is most useful in order to calculate the scattering amplitude, which can be defined using the asymptotic wave function \cite{Griffiths_QM}:
\begin{equation}
	\Psi(\vb{r}) = e^{i\vb{k}\cdot \vb{r}} + f(\theta,\phi) \frac{e^{ikr}}{r}.
\end{equation}
where $f(\theta,\phi)$ is  the total scattering amplitude. In the Born approximation, we can see using equation \ref{born_equation_first_order} that this equals:
\begin{equation}\label{scattering_amplitude}
	 f(\theta,\phi) = -  \frac{m}{2 \pi \hbar^2} \int e^{-i\vb{k}_f\cdot \vb{r'}}V(\vb{r}')e^{i\vb{k}\cdot\vb{r}-\vb{k}_f\cdot \vb{r}}\dd^3 \vb{r}'
\end{equation}
From which we can see how the integral in equation corresponds to the Fourier transform of the potential. The scattering amplitude determines the probability of scattering in a given direction $\theta$, and is therefore directly related to the differential scattering cross-section which is equal to the absolute square of the scattering amplitude \cite{Griffiths_QM}. A common use of the Born approximation is therefore also to obtain the form of the potential after finding the differential cross section experimentally \cite{born_detailed}.
\subsection{The Distorted Wave Born Approximation}\label{DWBA_section}
\begin{figure}[b]
	\centering
	\def\svgwidth{\textwidth}
	\input{dwba_terms.pdf_tex}
	\caption{A representation of different terms that are taken into account for the DWBA. The first term is identical to the conventional BA, while the other three terms describe scattering events from the scattering entities themselves.}
	\label{dwba_terms}
\end{figure}
At large scattering intensities, the regular Born approximation no longer holds. In such cases, the Distorted Wave Born Approximation is used, which is the underlying basis for simulations performed with GISAXS and GISANS measurements in this work. The foundational principles of the DWBA is covered in this subsection. At high intensities, scattering entities themselves can introduce perturbations into the field and the earlier approximation with a planar wave can no longer be applied, instead the wave function needs to be described as a distorted wave, a superposition of a downwards and upwards traveling planar wave. The described wave function has a distorted form that can be described as downward and upward travelling waves for both the scattered and incident waves \cite{BornAgainManual}:
\begin{equation}
	\psi_w(\vb{r}) = \psi_w^-(\vb{r}) + \psi_w^+(\vb{r}), w = i, f.
\end{equation}
Where $\psi_w^-$ describes the downwards wave while $\psi_w^+$ describes the upwards wave. The relevant scattering elements can then be described using Dirac notation as follows \cite{BornAgainManual}:
\begin{equation}\label{distortedbornexpansion}
	\mel{\psi_i}{\delta v}{\psi_f} = \mel*{\psi_i^-}{\delta v}{\psi_f^+} + \mel*{\psi_i^-}{\delta v}{\psi_f^-} + \mel*{\psi_i^+}{\delta v}{\psi_f^+} + \mel*{\psi_i^+}{\delta v}{\psi_f^-}.
\end{equation} 
Where $\delta v$ describes a perturbation on the scattering potential that the incident wave experiences. If we expand the left-hand side in the integral notation we get:
\begin{equation}
	\mel{\psi_i}{\delta v}{\psi_f} = \int e^{i\vb{k}_i\vb{r}}\delta v e^{i\vb{k}_f\vb{r}}\dd^3 r = \int \delta v e^{i\vb{q}\vb{r}}\dd^3 r.
\end{equation}
Which gives us the Fourier transform of the perturbed potential $\delta v$, which is what is being measured at the detector. Note how the first term on the right-hand side in equation \ref{distortedbornexpansion} simply describes the interaction with the downwards and the upwards wave, which is the usual term as used by the conventional BA. The additional terms are added upon this in the DWBA, and these describe the additional scattering effects for intense scattering. These terms are all illustrated in Figure \ref{dwba_terms}.

\section{Sample description in the simulations}
The samples need to be described in a physical model in order to fit the experimental data to the described formalism. In order to do this in a meaningful way, some approximations need to be made. It is easy to describe a sample in as much detail as possible leaving a lot of possible parameters to fit to, but this leaves the possibility of overfitting, losing a physical meaning behind the result. It is therefore often better to make a robust and simple model, than to make a complicated model that fits better to the data. Mathematician George Box famously wrote that  ‘all models are wrong, but some models are useful’ \cite{george_box}. The goal of the model description is not to find a model that gives a perfect fit to our data, but instead to find a model that gives useful information.  In this work this means a model that reliably gives physically correct information about the structural parameters of the sample. All samples that are covered in this thesis are described using the same physical model. A stack of an N amount of bilayers consisting of Ni and Ti are deposited on top op a Si substrate. A thin layer of SiO$_\textrm{2}$ is assumed to be present between the Si substrate and the multilayer stack. Finally, a small oxide layer is assumed to be present on the top layer. A sketch describing this is shown in Figure \ref{modeldescription}. While the initial interface width of the Ni and Ti layers are considered independent from each other, the accumulation of the interface width is assumed to be equal. The total interface width for a Ni and Ti layer in the stack, is therefore determined by:
\begin{eqnarray}
	\sigma_{\textrm{Ti}} = A \cdot j + \sigma_{\textrm{Ti,i}}, \\
	\sigma_{\textrm{Ni}} = A \cdot j + \sigma_{\textrm{Ni,i}}.
\end{eqnarray}
Where A is the accumulated interface width per bilayer, i indicates the position of the bilayer, and $\sigma_{\textrm{Ni,i}}$ and $\sigma_{\textrm{Ti,i}}$ describe the initial interface width of the Ni- and Ti layers respectively. For real samples, it is also common for the layer thickness to increase over time. This can be caused by slow down in the computer software that runs the deposition system, but also deteriorating targets during sample growth slightly affects the growth rate even within a single deposition. It is for this reason that the thickness of the layers is also allowed to increase linearly increase in the growth direction, and the thickness can therefore be described by:
\begin{eqnarray}
	d_{\textrm{Ti}} = B \cdot j + d_{\textrm{Ti,i}}, \\
	d_{\textrm{Ni}} = B \cdot j + d_{\textrm{Ni,i}}.
\end{eqnarray}
Where B is the increase in layer thickness per bilayer, i indicates the position of the bilayer, and $d_{\textrm{Ni,i}}$ and $d_{\textrm{Ti,i}}$ describe the initial layer thickness of the Ni- and Ti layers respectively
\begin{figure}
	\centering
	\def\svgwidth{\textwidth}
	%LaTeX with PSTricks extensions
%%Creator: Inkscape 1.0.1 (3bc2e813f5, 2020-09-07)
%%Please note this file requires PSTricks extensions
\psset{xunit=.5pt,yunit=.5pt,runit=.5pt}
\begin{pspicture}(1301.95843073,588.51504829)
{
\newrgbcolor{curcolor}{0.79215688 0.90588236 0.98431373}
\pscustom[linestyle=none,fillstyle=solid,fillcolor=curcolor]
{
\newpath
\moveto(202.97386205,149.90455632)
\lineto(202.97386205,119.91088703)
\lineto(1042.91469354,119.91088703)
\lineto(1042.91469354,149.90455632)
\closepath
}
}
{
\newrgbcolor{curcolor}{0 0 0}
\pscustom[linewidth=0.99999871,linecolor=curcolor]
{
\newpath
\moveto(202.97386205,149.90455632)
\lineto(202.97386205,119.91088703)
\lineto(1042.91469354,119.91088703)
\lineto(1042.91469354,149.90455632)
\closepath
}
}
{
\newrgbcolor{curcolor}{0 0.58823532 1}
\pscustom[linestyle=none,fillstyle=solid,fillcolor=curcolor]
{
\newpath
\moveto(202.97383559,119.91088325)
\lineto(202.97383559,99.9151075)
\lineto(1042.91469354,99.9151075)
\lineto(1042.91469354,119.91088325)
\lineto(202.97383559,119.91088325)
}
}
{
\newrgbcolor{curcolor}{0 0 0}
\pscustom[linewidth=0.99999871,linecolor=curcolor]
{
\newpath
\moveto(202.97383559,119.91088325)
\lineto(202.97383559,99.9151075)
\lineto(1042.91469354,99.9151075)
\lineto(1042.91469354,119.91088325)
\lineto(202.97383559,119.91088325)
}
}
{
\newrgbcolor{curcolor}{0.79215688 0.90588236 0.98431373}
\pscustom[linestyle=none,fillstyle=solid,fillcolor=curcolor]
{
\newpath
\moveto(202.97383559,99.9151075)
\lineto(202.97383559,69.92143821)
\lineto(1042.91469354,69.92143821)
\lineto(1042.91469354,99.9151075)
\closepath
}
}
{
\newrgbcolor{curcolor}{0 0 0}
\pscustom[linewidth=0.99999871,linecolor=curcolor]
{
\newpath
\moveto(202.97383559,99.9151075)
\lineto(202.97383559,69.92143821)
\lineto(1042.91469354,69.92143821)
\lineto(1042.91469354,99.9151075)
\closepath
}
}
{
\newrgbcolor{curcolor}{0.7647059 0.7647059 0.7647059}
\pscustom[linestyle=none,fillstyle=solid,fillcolor=curcolor]
{
\newpath
\moveto(202.87112693,59.92319695)
\lineto(202.87112693,0.49999443)
\lineto(1042.91469354,0.49999443)
\lineto(1042.91469354,59.92319695)
\closepath
}
}
{
\newrgbcolor{curcolor}{0 0 0}
\pscustom[linewidth=0.99999871,linecolor=curcolor]
{
\newpath
\moveto(202.87112693,59.92319695)
\lineto(202.87112693,0.49999443)
\lineto(1042.91469354,0.49999443)
\lineto(1042.91469354,59.92319695)
\closepath
}
}
{
\newrgbcolor{curcolor}{0.59215689 0.59215689 0.59215689}
\pscustom[linestyle=none,fillstyle=solid,fillcolor=curcolor]
{
\newpath
\moveto(202.91467465,69.89710561)
\lineto(202.91467465,59.89710309)
\lineto(1042.91465575,59.89710309)
\lineto(1042.91465575,69.89710561)
\closepath
}
}
{
\newrgbcolor{curcolor}{0 0 0}
\pscustom[linewidth=0.99999871,linecolor=curcolor]
{
\newpath
\moveto(202.91467465,69.89710561)
\lineto(202.91467465,59.89710309)
\lineto(1042.91465575,59.89710309)
\lineto(1042.91465575,69.89710561)
\closepath
}
}
{
\newrgbcolor{curcolor}{0 0.58823532 1}
\pscustom[linestyle=none,fillstyle=solid,fillcolor=curcolor]
{
\newpath
\moveto(202.91467465,569.85489333)
\lineto(202.91467465,549.85911758)
\lineto(1042.85554394,549.85911758)
\lineto(1042.85554394,569.85489333)
\lineto(202.91467465,569.85489333)
}
}
{
\newrgbcolor{curcolor}{0 0 0}
\pscustom[linewidth=0.99999871,linecolor=curcolor]
{
\newpath
\moveto(202.91467465,569.85489333)
\lineto(202.91467465,549.85911758)
\lineto(1042.85554394,549.85911758)
\lineto(1042.85554394,569.85489333)
\lineto(202.91467465,569.85489333)
}
}
{
\newrgbcolor{curcolor}{0.79215688 0.90588236 0.98431373}
\pscustom[linestyle=none,fillstyle=solid,fillcolor=curcolor]
{
\newpath
\moveto(202.91463685,549.85911758)
\lineto(202.91463685,519.86544451)
\lineto(1042.85554394,519.86544451)
\lineto(1042.85554394,549.85911758)
\closepath
}
}
{
\newrgbcolor{curcolor}{0 0 0}
\pscustom[linewidth=0.99999871,linecolor=curcolor]
{
\newpath
\moveto(202.91463685,549.85911758)
\lineto(202.91463685,519.86544451)
\lineto(1042.85554394,519.86544451)
\lineto(1042.85554394,549.85911758)
\closepath
}
}
{
\newrgbcolor{curcolor}{0 0.58823532 1}
\pscustom[linestyle=none,fillstyle=solid,fillcolor=curcolor]
{
\newpath
\moveto(202.91463685,519.86544451)
\lineto(202.91463685,499.86966876)
\lineto(1042.85554394,499.86966876)
\lineto(1042.85554394,519.86544451)
\lineto(202.91463685,519.86544451)
}
}
{
\newrgbcolor{curcolor}{0 0 0}
\pscustom[linewidth=0.99999871,linecolor=curcolor]
{
\newpath
\moveto(202.91463685,519.86544451)
\lineto(202.91463685,499.86966876)
\lineto(1042.85554394,499.86966876)
\lineto(1042.85554394,519.86544451)
\lineto(202.91463685,519.86544451)
}
}
{
\newrgbcolor{curcolor}{0.79215688 0.90588236 0.98431373}
\pscustom[linestyle=none,fillstyle=solid,fillcolor=curcolor]
{
\newpath
\moveto(202.91463685,499.86966876)
\lineto(202.91463685,469.87600325)
\lineto(1042.85554394,469.87600325)
\lineto(1042.85554394,499.86966876)
\closepath
}
}
{
\newrgbcolor{curcolor}{0 0 0}
\pscustom[linewidth=0.99999871,linecolor=curcolor]
{
\newpath
\moveto(202.91463685,499.86966876)
\lineto(202.91463685,469.87600325)
\lineto(1042.85554394,469.87600325)
\lineto(1042.85554394,499.86966876)
\closepath
}
}
{
\newrgbcolor{curcolor}{0 0.58823532 1}
\pscustom[linestyle=none,fillstyle=solid,fillcolor=curcolor]
{
\newpath
\moveto(202.91471244,469.87600325)
\lineto(202.91471244,449.88022486)
\lineto(1042.85554394,449.88022486)
\lineto(1042.85554394,469.87600325)
\lineto(202.91471244,469.87600325)
}
}
{
\newrgbcolor{curcolor}{0 0 0}
\pscustom[linewidth=0.99999871,linecolor=curcolor]
{
\newpath
\moveto(202.91471244,469.87600325)
\lineto(202.91471244,449.88022486)
\lineto(1042.85554394,449.88022486)
\lineto(1042.85554394,469.87600325)
\lineto(202.91471244,469.87600325)
}
}
{
\newrgbcolor{curcolor}{0.79215688 0.90588236 0.98431373}
\pscustom[linestyle=none,fillstyle=solid,fillcolor=curcolor]
{
\newpath
\moveto(202.91471244,449.88022486)
\lineto(202.91471244,419.88656324)
\lineto(1042.85554394,419.88656324)
\lineto(1042.85554394,449.88022486)
\closepath
}
}
{
\newrgbcolor{curcolor}{0 0 0}
\pscustom[linewidth=0.99999871,linecolor=curcolor]
{
\newpath
\moveto(202.91471244,449.88022486)
\lineto(202.91471244,419.88656324)
\lineto(1042.85554394,419.88656324)
\lineto(1042.85554394,449.88022486)
\closepath
}
}
{
\newrgbcolor{curcolor}{0 0.58823532 1}
\pscustom[linestyle=none,fillstyle=solid,fillcolor=curcolor]
{
\newpath
\moveto(202.91467465,419.8865519)
\lineto(202.91467465,399.89077226)
\lineto(1042.85554394,399.89077226)
\lineto(1042.85554394,419.8865519)
\lineto(202.91467465,419.8865519)
}
}
{
\newrgbcolor{curcolor}{0 0 0}
\pscustom[linewidth=0.99999871,linecolor=curcolor]
{
\newpath
\moveto(202.91467465,419.8865519)
\lineto(202.91467465,399.89077226)
\lineto(1042.85554394,399.89077226)
\lineto(1042.85554394,419.8865519)
\lineto(202.91467465,419.8865519)
}
}
{
\newrgbcolor{curcolor}{0.79215688 0.90588236 0.98431373}
\pscustom[linestyle=none,fillstyle=solid,fillcolor=curcolor]
{
\newpath
\moveto(202.91467465,399.89077226)
\lineto(202.91467465,369.89709805)
\lineto(1042.85554394,369.89709805)
\lineto(1042.85554394,399.89077226)
\closepath
}
}
{
\newrgbcolor{curcolor}{0 0 0}
\pscustom[linewidth=0.99999871,linecolor=curcolor]
{
\newpath
\moveto(202.91467465,399.89077226)
\lineto(202.91467465,369.89709805)
\lineto(1042.85554394,369.89709805)
\lineto(1042.85554394,399.89077226)
\closepath
}
}
{
\newrgbcolor{curcolor}{0 0.58823532 1}
\pscustom[linestyle=none,fillstyle=solid,fillcolor=curcolor]
{
\newpath
\moveto(202.97379402,369.89709805)
\lineto(202.97379402,349.90131853)
\lineto(1042.91469354,349.90131853)
\lineto(1042.91469354,369.89709805)
\lineto(202.97379402,369.89709805)
}
}
{
\newrgbcolor{curcolor}{0 0 0}
\pscustom[linewidth=0.99999871,linecolor=curcolor]
{
\newpath
\moveto(202.97379402,369.89709805)
\lineto(202.97379402,349.90131853)
\lineto(1042.91469354,349.90131853)
\lineto(1042.91469354,369.89709805)
\lineto(202.97379402,369.89709805)
}
}
{
\newrgbcolor{curcolor}{0.45490196 0.58823532 0.68235296}
\pscustom[linestyle=none,fillstyle=solid,fillcolor=curcolor]
{
\newpath
\moveto(202.85548724,569.8938716)
\lineto(202.85548724,579.89387034)
\lineto(1042.85550614,579.89387034)
\lineto(1042.85550614,569.8938716)
\closepath
}
}
{
\newrgbcolor{curcolor}{0 0 0}
\pscustom[linewidth=0.99999871,linecolor=curcolor]
{
\newpath
\moveto(202.85548724,569.8938716)
\lineto(202.85548724,579.89387034)
\lineto(1042.85550614,579.89387034)
\lineto(1042.85550614,569.8938716)
\closepath
}
}
{
\newrgbcolor{curcolor}{0.79215688 0.90588236 0.98431373}
\pscustom[linestyle=none,fillstyle=solid,fillcolor=curcolor]
{
\newpath
\moveto(202.97373354,349.89710057)
\lineto(202.97373354,319.9034275)
\lineto(1042.91469354,319.9034275)
\lineto(1042.91469354,349.89710057)
\closepath
}
}
{
\newrgbcolor{curcolor}{0 0 0}
\pscustom[linewidth=0.99999871,linecolor=curcolor]
{
\newpath
\moveto(202.97373354,349.89710057)
\lineto(202.97373354,319.9034275)
\lineto(1042.91469354,319.9034275)
\lineto(1042.91469354,349.89710057)
\closepath
}
}
{
\newrgbcolor{curcolor}{0 0.58823532 1}
\pscustom[linestyle=none,fillstyle=solid,fillcolor=curcolor]
{
\newpath
\moveto(202.97379024,319.86122152)
\lineto(202.97379024,299.86544577)
\lineto(1042.91469354,299.86544577)
\lineto(1042.91469354,319.86122152)
\lineto(202.97379024,319.86122152)
}
}
{
\newrgbcolor{curcolor}{0 0 0}
\pscustom[linewidth=0.99999871,linecolor=curcolor]
{
\newpath
\moveto(202.97379024,319.86122152)
\lineto(202.97379024,299.86544577)
\lineto(1042.91469354,299.86544577)
\lineto(1042.91469354,319.86122152)
\lineto(202.97379024,319.86122152)
}
}
{
\newrgbcolor{curcolor}{0.79215688 0.90588236 0.98431373}
\pscustom[linestyle=none,fillstyle=solid,fillcolor=curcolor]
{
\newpath
\moveto(202.97379024,299.86544577)
\lineto(202.97379024,269.87178026)
\lineto(1042.91469354,269.87178026)
\lineto(1042.91469354,299.86544577)
\closepath
}
}
{
\newrgbcolor{curcolor}{0 0 0}
\pscustom[linewidth=0.99999871,linecolor=curcolor]
{
\newpath
\moveto(202.97379024,299.86544577)
\lineto(202.97379024,269.87178026)
\lineto(1042.91469354,269.87178026)
\lineto(1042.91469354,299.86544577)
\closepath
}
}
{
\newrgbcolor{curcolor}{0 0.58823532 1}
\pscustom[linestyle=none,fillstyle=solid,fillcolor=curcolor]
{
\newpath
\moveto(202.97384315,269.87178026)
\lineto(202.97384315,249.87600073)
\lineto(1042.91469354,249.87600073)
\lineto(1042.91469354,269.87178026)
\lineto(202.97384315,269.87178026)
}
}
{
\newrgbcolor{curcolor}{0 0 0}
\pscustom[linewidth=0.99999871,linecolor=curcolor]
{
\newpath
\moveto(202.97384315,269.87178026)
\lineto(202.97384315,249.87600073)
\lineto(1042.91469354,249.87600073)
\lineto(1042.91469354,269.87178026)
\lineto(202.97384315,269.87178026)
}
}
{
\newrgbcolor{curcolor}{0.79215688 0.90588236 0.98431373}
\pscustom[linestyle=none,fillstyle=solid,fillcolor=curcolor]
{
\newpath
\moveto(202.97384315,249.87600073)
\lineto(202.97384315,219.882339)
\lineto(1042.91469354,219.882339)
\lineto(1042.91469354,249.87600073)
\closepath
}
}
{
\newrgbcolor{curcolor}{0 0 0}
\pscustom[linewidth=0.99999871,linecolor=curcolor]
{
\newpath
\moveto(202.97384315,249.87600073)
\lineto(202.97384315,219.882339)
\lineto(1042.91469354,219.882339)
\lineto(1042.91469354,249.87600073)
\closepath
}
}
{
\newrgbcolor{curcolor}{0 0.58823532 1}
\pscustom[linestyle=none,fillstyle=solid,fillcolor=curcolor]
{
\newpath
\moveto(202.97381669,219.88232766)
\lineto(202.97381669,199.88654813)
\lineto(1042.91469354,199.88654813)
\lineto(1042.91469354,219.88232766)
\lineto(202.97381669,219.88232766)
}
}
{
\newrgbcolor{curcolor}{0 0 0}
\pscustom[linewidth=0.99999871,linecolor=curcolor]
{
\newpath
\moveto(202.97381669,219.88232766)
\lineto(202.97381669,199.88654813)
\lineto(1042.91469354,199.88654813)
\lineto(1042.91469354,219.88232766)
\lineto(202.97381669,219.88232766)
}
}
{
\newrgbcolor{curcolor}{0.79215688 0.90588236 0.98431373}
\pscustom[linestyle=none,fillstyle=solid,fillcolor=curcolor]
{
\newpath
\moveto(202.97381669,199.88654813)
\lineto(202.97381669,169.89287884)
\lineto(1042.91469354,169.89287884)
\lineto(1042.91469354,199.88654813)
\closepath
}
}
{
\newrgbcolor{curcolor}{0 0 0}
\pscustom[linewidth=0.99999871,linecolor=curcolor]
{
\newpath
\moveto(202.97381669,199.88654813)
\lineto(202.97381669,169.89287884)
\lineto(1042.91469354,169.89287884)
\lineto(1042.91469354,199.88654813)
\closepath
}
}
{
\newrgbcolor{curcolor}{0 0.58823532 1}
\pscustom[linestyle=none,fillstyle=solid,fillcolor=curcolor]
{
\newpath
\moveto(202.97381669,169.89287884)
\lineto(202.97381669,149.89709931)
\lineto(1042.91469354,149.89709931)
\lineto(1042.91469354,169.89287884)
\lineto(202.97381669,169.89287884)
}
}
{
\newrgbcolor{curcolor}{0 0 0}
\pscustom[linewidth=0.99999871,linecolor=curcolor]
{
\newpath
\moveto(202.97381669,169.89287884)
\lineto(202.97381669,149.89709931)
\lineto(1042.91469354,149.89709931)
\lineto(1042.91469354,169.89287884)
\lineto(202.97381669,169.89287884)
}
}
{
\newrgbcolor{curcolor}{0 0 0}
\pscustom[linewidth=1.89354326,linecolor=curcolor]
{
\newpath
\moveto(1047.91466835,567.89709805)
\lineto(1052.91468094,567.89709805)
\lineto(1052.91468094,521.89709931)
\lineto(1047.91466835,521.89709931)
}
}
{
\newrgbcolor{curcolor}{0 0 0}
\pscustom[linewidth=1.89354326,linecolor=curcolor]
{
\newpath
\moveto(1052.91468094,542.89709931)
\lineto(1057.91469354,542.89709931)
}
}
{
\newrgbcolor{curcolor}{0 0 0}
\pscustom[linestyle=none,fillstyle=solid,fillcolor=curcolor]
{
\newpath
\moveto(1069.01230426,536.23994895)
\lineto(1069.01230426,537.23194934)
\lineto(1069.78030457,537.23194934)
\curveto(1070.93230503,537.23194934)(1071.63630531,537.32794938)(1071.89230541,537.51994945)
\curveto(1072.16963885,537.71194953)(1072.30830557,538.11728303)(1072.30830557,538.73594994)
\lineto(1072.30830557,555.59995665)
\curveto(1072.30830557,556.21862356)(1072.16963885,556.62395706)(1071.89230541,556.81595713)
\curveto(1071.63630531,557.00795721)(1070.93230503,557.10395725)(1069.78030457,557.10395725)
\lineto(1069.01230426,557.10395725)
\lineto(1069.01230426,558.09595764)
\lineto(1080.72430892,558.09595764)
\curveto(1082.7509764,558.09595764)(1084.4363104,557.55195743)(1085.78031094,556.46395699)
\curveto(1087.14564481,555.37595656)(1087.82831175,554.11728939)(1087.82831175,552.68795549)
\curveto(1087.82831175,551.51462169)(1087.32697822,550.45862127)(1086.32431115,549.51995423)
\curveto(1085.34297743,548.58128719)(1084.07364359,547.96262028)(1082.51630964,547.66395349)
\curveto(1084.28697701,547.47195342)(1085.75897759,546.84261983)(1086.93231139,545.77595274)
\curveto(1088.10564519,544.70928565)(1088.6923121,543.48261849)(1088.6923121,542.09595128)
\curveto(1088.6923121,540.53861732)(1088.02031183,539.17328345)(1086.67631129,537.99994965)
\curveto(1085.33231076,536.82661585)(1083.62564341,536.23994895)(1081.55630926,536.23994895)
\closepath
\moveto(1074.96430663,538.51194985)
\curveto(1074.96430663,537.95728296)(1075.03897333,537.60528282)(1075.18830672,537.45594943)
\curveto(1075.35897346,537.30661604)(1075.80697363,537.23194934)(1076.53230726,537.23194934)
\lineto(1080.53230885,537.23194934)
\curveto(1082.06830946,537.23194934)(1083.27364327,537.73328287)(1084.14831029,538.73594994)
\curveto(1085.04431064,539.738617)(1085.49231082,540.86928412)(1085.49231082,542.12795129)
\curveto(1085.49231082,543.38661846)(1085.097644,544.54928559)(1084.30831035,545.61595268)
\curveto(1083.54031004,546.70395311)(1082.43097627,547.24795333)(1080.98030903,547.24795333)
\lineto(1074.96430663,547.24795333)
\closepath
\moveto(1074.96430663,547.95195361)
\lineto(1079.60430848,547.95195361)
\curveto(1081.2469758,547.95195361)(1082.50564297,548.44262047)(1083.38030998,549.42395419)
\curveto(1084.27631034,550.42662126)(1084.72431052,551.51462169)(1084.72431052,552.68795549)
\curveto(1084.72431052,553.73328924)(1084.37231038,554.72528964)(1083.6683101,555.66395668)
\curveto(1082.96430982,556.62395706)(1081.9083094,557.10395725)(1080.50030883,557.10395725)
\lineto(1076.53230726,557.10395725)
\curveto(1075.80697363,557.10395725)(1075.35897346,557.02929055)(1075.18830672,556.87995716)
\curveto(1075.03897333,556.73062377)(1074.96430663,556.37862363)(1074.96430663,555.82395674)
\closepath
}
}
{
\newrgbcolor{curcolor}{0 0 0}
\pscustom[linestyle=none,fillstyle=solid,fillcolor=curcolor]
{
\newpath
\moveto(1091.57225746,536.23994895)
\lineto(1091.57225746,537.23194934)
\curveto(1092.70292458,537.23194934)(1093.39625818,537.29594937)(1093.65225829,537.42394942)
\curveto(1093.92959173,537.57328281)(1094.06825845,537.98928297)(1094.06825845,538.67194991)
\lineto(1094.06825845,547.27995334)
\curveto(1094.06825845,548.06928699)(1093.92959173,548.55995385)(1093.65225829,548.75195392)
\curveto(1093.39625818,548.943954)(1092.74559126,549.03995404)(1091.70025751,549.03995404)
\lineto(1091.70025751,550.03195443)
\lineto(1096.18025929,550.38395457)
\lineto(1096.18025929,538.6399499)
\curveto(1096.18025929,537.99994965)(1096.286926,537.60528282)(1096.50025942,537.45594943)
\curveto(1096.73492618,537.30661604)(1097.37492644,537.23194934)(1098.42026018,537.23194934)
\lineto(1098.42026018,536.23994895)
\curveto(1096.24425932,536.30394897)(1095.13492554,536.33594898)(1095.09225886,536.33594898)
\curveto(1094.79359207,536.33594898)(1093.62025827,536.30394897)(1091.57225746,536.23994895)
\closepath
\moveto(1092.91625799,555.95195679)
\curveto(1092.91625799,556.37862363)(1093.07625806,556.76262378)(1093.39625818,557.10395725)
\curveto(1093.73759165,557.46662406)(1094.14292515,557.64795747)(1094.61225867,557.64795747)
\curveto(1095.08159219,557.64795747)(1095.47625901,557.4879574)(1095.79625914,557.16795727)
\curveto(1096.13759261,556.84795715)(1096.30825934,556.44262365)(1096.30825934,555.95195679)
\curveto(1096.30825934,555.46128993)(1096.13759261,555.05595643)(1095.79625914,554.73595631)
\curveto(1095.47625901,554.41595618)(1095.08159219,554.25595612)(1094.61225867,554.25595612)
\curveto(1094.12159181,554.25595612)(1093.71625831,554.42662285)(1093.39625818,554.76795632)
\curveto(1093.07625806,555.10928979)(1092.91625799,555.50395661)(1092.91625799,555.95195679)
\closepath
}
}
{
\newrgbcolor{curcolor}{0 0 0}
\pscustom[linestyle=none,fillstyle=solid,fillcolor=curcolor]
{
\newpath
\moveto(1100.43621298,536.23994895)
\lineto(1100.43621298,537.23194934)
\curveto(1101.5668801,537.23194934)(1102.26021371,537.29594937)(1102.51621381,537.42394942)
\curveto(1102.79354725,537.57328281)(1102.93221397,537.98928297)(1102.93221397,538.67194991)
\lineto(1102.93221397,555.31195654)
\curveto(1102.93221397,556.10129018)(1102.79354725,556.59195705)(1102.51621381,556.78395712)
\curveto(1102.23888036,556.99729054)(1101.54554675,557.10395725)(1100.43621298,557.10395725)
\lineto(1100.43621298,558.09595764)
\lineto(1105.04421481,558.44795778)
\lineto(1105.04421481,538.67194991)
\curveto(1105.04421481,537.98928297)(1105.17221486,537.57328281)(1105.42821497,537.42394942)
\curveto(1105.70554841,537.29594937)(1106.40954869,537.23194934)(1107.54021581,537.23194934)
\lineto(1107.54021581,536.23994895)
\curveto(1107.2842157,536.23994895)(1106.87888221,536.25061562)(1106.32421532,536.27194896)
\curveto(1105.79088178,536.2932823)(1105.32154826,536.30394897)(1104.91621476,536.30394897)
\curveto(1104.53221461,536.32528231)(1104.22288115,536.33594898)(1103.98821439,536.33594898)
\curveto(1103.73221429,536.33594898)(1102.54821382,536.30394897)(1100.43621298,536.23994895)
\closepath
}
}
{
\newrgbcolor{curcolor}{0 0 0}
\pscustom[linestyle=none,fillstyle=solid,fillcolor=curcolor]
{
\newpath
\moveto(1109.58817222,539.27995016)
\curveto(1109.58817222,541.09328421)(1110.65483931,542.47995143)(1112.78817349,543.43995181)
\curveto(1114.068174,544.05861872)(1116.03084145,544.44261888)(1118.67617584,544.59195227)
\lineto(1118.67617584,545.77595274)
\curveto(1118.67617584,547.09861993)(1118.3241757,548.11195367)(1117.62017542,548.81595395)
\curveto(1116.93750848,549.51995423)(1116.1588415,549.87195437)(1115.28417449,549.87195437)
\curveto(1113.72684053,549.87195437)(1112.57484007,549.38128751)(1111.82817311,548.39995378)
\curveto(1112.46817336,548.37862044)(1112.8948402,548.20795371)(1113.10817362,547.88795358)
\curveto(1113.34284038,547.56795345)(1113.46017376,547.24795333)(1113.46017376,546.9279532)
\curveto(1113.46017376,546.50128636)(1113.32150704,546.14928622)(1113.04417359,545.87195278)
\curveto(1112.78817349,545.59461934)(1112.43617335,545.45595261)(1111.98817317,545.45595261)
\curveto(1111.56150634,545.45595261)(1111.2095062,545.58395266)(1110.93217275,545.83995277)
\curveto(1110.65483931,546.11728621)(1110.51617259,546.49061969)(1110.51617259,546.95995321)
\curveto(1110.51617259,548.00528696)(1110.98550611,548.8692873)(1111.92417315,549.55195424)
\curveto(1112.86284019,550.23462118)(1114.00417398,550.57595465)(1115.34817451,550.57595465)
\curveto(1117.09750854,550.57595465)(1118.55884246,549.98928775)(1119.73217626,548.81595395)
\curveto(1120.09484307,548.45328714)(1120.36150984,548.03728697)(1120.53217657,547.56795345)
\curveto(1120.72417665,547.09861993)(1120.83084336,546.70395311)(1120.8521767,546.38395298)
\curveto(1120.87351004,546.0852862)(1120.88417671,545.63728602)(1120.88417671,545.03995245)
\lineto(1120.88417671,538.6399499)
\curveto(1120.88417671,538.51194985)(1120.90551006,538.34128311)(1120.94817674,538.1279497)
\curveto(1120.99084342,537.93594962)(1121.1081768,537.70128286)(1121.30017688,537.42394942)
\curveto(1121.49217696,537.16794931)(1121.74817706,537.03994926)(1122.06817719,537.03994926)
\curveto(1122.83617749,537.03994926)(1123.22017764,537.7226162)(1123.22017764,539.08795008)
\lineto(1123.22017764,540.87995079)
\lineto(1124.02017796,540.87995079)
\lineto(1124.02017796,539.08795008)
\curveto(1124.02017796,537.95728296)(1123.72151118,537.16794931)(1123.12417761,536.71994914)
\curveto(1122.52684403,536.27194896)(1121.95084381,536.04794887)(1121.39617692,536.04794887)
\curveto(1120.69217664,536.04794887)(1120.11617641,536.30394897)(1119.66817623,536.81594917)
\curveto(1119.22017605,537.34928272)(1118.96417595,537.96794963)(1118.90017592,538.67194991)
\curveto(1118.5801758,537.86128292)(1118.04684225,537.18928266)(1117.30017529,536.65594911)
\curveto(1116.57484167,536.14394891)(1115.71084132,535.88794881)(1114.70817426,535.88794881)
\curveto(1113.94017395,535.88794881)(1113.19350699,535.98394884)(1112.46817336,536.17594892)
\curveto(1111.74283974,536.367949)(1111.08150615,536.73061581)(1110.48417258,537.26394935)
\curveto(1109.886839,537.7972829)(1109.58817222,538.46928317)(1109.58817222,539.27995016)
\closepath
\moveto(1112.0521732,539.31195017)
\curveto(1112.0521732,538.52261652)(1112.32950664,537.87194959)(1112.88417353,537.35994939)
\curveto(1113.46017376,536.84794919)(1114.1428407,536.59194909)(1114.93217435,536.59194909)
\curveto(1115.8281747,536.59194909)(1116.6708417,536.93328255)(1117.46017535,537.61594949)
\curveto(1118.27084234,538.31994977)(1118.67617584,539.35461685)(1118.67617584,540.71995073)
\lineto(1118.67617584,543.919952)
\curveto(1116.30817489,543.83461863)(1114.61217422,543.32261843)(1113.58817381,542.38395139)
\curveto(1112.5641734,541.44528435)(1112.0521732,540.42128394)(1112.0521732,539.31195017)
\closepath
}
}
{
\newrgbcolor{curcolor}{0 0 0}
\pscustom[linestyle=none,fillstyle=solid,fillcolor=curcolor]
{
\newpath
\moveto(1123.95605905,532.27194737)
\curveto(1123.95605905,532.71994754)(1124.0840591,533.06128101)(1124.3400592,533.29594777)
\curveto(1124.61739264,533.53061453)(1124.93739277,533.64794791)(1125.30005958,533.64794791)
\curveto(1125.68405974,533.64794791)(1126.00405986,533.53061453)(1126.26005997,533.29594777)
\curveto(1126.51606007,533.03994767)(1126.64406012,532.70928087)(1126.64406012,532.30394738)
\curveto(1126.64406012,531.51461373)(1126.22805995,531.06661355)(1125.39605962,530.95994684)
\curveto(1125.80139312,530.57594669)(1126.30272665,530.38394661)(1126.90006022,530.38394661)
\curveto(1127.54006047,530.38394661)(1128.1160607,530.61861337)(1128.62806091,531.08794689)
\curveto(1129.14006111,531.55728041)(1129.50272792,532.01594726)(1129.71606134,532.46394744)
\curveto(1129.9507281,532.91194762)(1130.22806154,533.5412812)(1130.54806167,534.35194819)
\curveto(1130.84672846,534.99194845)(1131.11339523,535.62128203)(1131.34806199,536.23994895)
\lineto(1126.54806008,547.91995359)
\curveto(1126.33472666,548.4319538)(1126.06805989,548.74128725)(1125.74805976,548.84795396)
\curveto(1125.42805963,548.97595401)(1124.83072606,549.03995404)(1123.95605905,549.03995404)
\lineto(1123.95605905,550.03195443)
\curveto(1124.91605943,549.96795441)(1125.95072651,549.9359544)(1127.06006028,549.9359544)
\curveto(1127.72139388,549.9359544)(1128.88406101,549.96795441)(1130.54806167,550.03195443)
\lineto(1130.54806167,549.03995404)
\curveto(1129.35339453,549.03995404)(1128.75606096,548.7626206)(1128.75606096,548.20795371)
\curveto(1128.75606096,548.14395368)(1128.82006098,547.95195361)(1128.94806103,547.63195348)
\lineto(1132.50006245,539.02395005)
\lineto(1135.73206373,546.89595319)
\curveto(1135.86006379,547.19461997)(1135.92406381,547.45062007)(1135.92406381,547.66395349)
\curveto(1135.92406381,548.53862051)(1135.42273028,548.99728736)(1134.42006321,549.03995404)
\lineto(1134.42006321,550.03195443)
\curveto(1135.82806377,549.96795441)(1136.75606414,549.9359544)(1137.20406432,549.9359544)
\curveto(1138.07873134,549.9359544)(1138.87873165,549.96795441)(1139.60406528,550.03195443)
\lineto(1139.60406528,549.03995404)
\curveto(1138.17473137,549.03995404)(1137.18273098,548.3572871)(1136.62806409,546.99195322)
\lineto(1130.99606185,533.32794779)
\curveto(1129.97206144,530.89594682)(1128.60672757,529.67994633)(1126.90006022,529.67994633)
\curveto(1126.08939323,529.67994633)(1125.39605962,529.93594644)(1124.82005939,530.44794664)
\curveto(1124.24405916,530.9386135)(1123.95605905,531.54661374)(1123.95605905,532.27194737)
\closepath
}
}
{
\newrgbcolor{curcolor}{0 0 0}
\pscustom[linestyle=none,fillstyle=solid,fillcolor=curcolor]
{
\newpath
\moveto(1140.211959,543.27995175)
\curveto(1140.211959,545.28528588)(1140.85195925,547.0026199)(1142.13195976,548.4319538)
\curveto(1143.43329361,549.8612877)(1145.01196091,550.57595465)(1146.86796164,550.57595465)
\curveto(1148.74529573,550.57595465)(1150.16396296,549.96795441)(1151.12396334,548.75195392)
\curveto(1152.10529706,547.53595344)(1152.59596392,546.05328618)(1152.59596392,544.30395215)
\curveto(1152.59596392,543.98395203)(1152.55329724,543.79195195)(1152.46796387,543.72795193)
\curveto(1152.38263051,543.6639519)(1152.15863042,543.63195189)(1151.79596361,543.63195189)
\lineto(1142.86796005,543.63195189)
\curveto(1142.86796005,541.45595102)(1143.18796018,539.85595038)(1143.82796044,538.83194998)
\curveto(1144.72396079,537.40261607)(1145.9292946,536.68794912)(1147.44396187,536.68794912)
\curveto(1147.65729529,536.68794912)(1147.88129538,536.70928247)(1148.11596214,536.75194915)
\curveto(1148.37196224,536.79461583)(1148.7559624,536.90128254)(1149.2679626,537.07194928)
\curveto(1149.7799628,537.26394935)(1150.25996299,537.60528282)(1150.70796317,538.09594968)
\curveto(1151.15596335,538.58661655)(1151.50796349,539.21595013)(1151.76396359,539.98395044)
\curveto(1151.82796362,540.28261722)(1151.96663034,540.43195061)(1152.17996376,540.43195061)
\curveto(1152.4572972,540.43195061)(1152.59596392,540.30395056)(1152.59596392,540.04795046)
\curveto(1152.59596392,539.85595038)(1152.48929722,539.54661693)(1152.2759638,539.11995009)
\curveto(1152.08396372,538.7146166)(1151.79596361,538.25594975)(1151.41196345,537.74394954)
\curveto(1151.0279633,537.25328268)(1150.46262974,536.81594917)(1149.71596278,536.43194902)
\curveto(1148.96929581,536.06928221)(1148.14796215,535.88794881)(1147.2519618,535.88794881)
\curveto(1145.37462772,535.88794881)(1143.7319604,536.59194909)(1142.32395984,537.99994965)
\curveto(1140.91595928,539.42928355)(1140.211959,541.18928425)(1140.211959,543.27995175)
\closepath
\moveto(1142.89996007,544.30395215)
\lineto(1150.48396308,544.30395215)
\curveto(1150.48396308,544.75195233)(1150.4412964,545.22128585)(1150.35596303,545.71195272)
\curveto(1150.29196301,546.22395292)(1150.14262961,546.82128649)(1149.90796285,547.50395343)
\curveto(1149.69462944,548.20795371)(1149.32129595,548.77328727)(1148.78796241,549.1999541)
\curveto(1148.27596221,549.64795428)(1147.63596195,549.87195437)(1146.86796164,549.87195437)
\curveto(1146.52662818,549.87195437)(1146.16396136,549.79728767)(1145.77996121,549.64795428)
\curveto(1145.4172944,549.49862089)(1145.01196091,549.24262079)(1144.56396073,548.87995398)
\curveto(1144.11596055,548.53862051)(1143.7319604,547.96262028)(1143.41196027,547.15195329)
\curveto(1143.11329348,546.36261964)(1142.94262675,545.41328593)(1142.89996007,544.30395215)
\closepath
}
}
{
\newrgbcolor{curcolor}{0 0 0}
\pscustom[linestyle=none,fillstyle=solid,fillcolor=curcolor]
{
\newpath
\moveto(1154.4199125,536.23994895)
\lineto(1154.4199125,537.23194934)
\curveto(1155.55057962,537.23194934)(1156.24391323,537.29594937)(1156.49991333,537.42394942)
\curveto(1156.77724677,537.57328281)(1156.91591349,537.98928297)(1156.91591349,538.67194991)
\lineto(1156.91591349,547.24795333)
\curveto(1156.91591349,548.03728697)(1156.77724677,548.52795384)(1156.49991333,548.71995391)
\curveto(1156.22257989,548.93328733)(1155.52924628,549.03995404)(1154.4199125,549.03995404)
\lineto(1154.4199125,550.03195443)
\lineto(1158.86791427,550.38395457)
\lineto(1158.86791427,546.86395317)
\curveto(1159.16658106,547.78128687)(1159.63591458,548.59195386)(1160.27591483,549.29595414)
\curveto(1160.91591509,550.02128776)(1161.75858209,550.38395457)(1162.80391584,550.38395457)
\curveto(1163.48658278,550.38395457)(1164.05191633,550.1919545)(1164.49991651,549.80795435)
\curveto(1164.94791669,549.42395419)(1165.17191678,548.96528734)(1165.17191678,548.4319538)
\curveto(1165.17191678,547.96262028)(1165.02258339,547.61062014)(1164.7239166,547.37595338)
\curveto(1164.44658316,547.14128662)(1164.1372497,547.02395324)(1163.79591623,547.02395324)
\curveto(1163.41191608,547.02395324)(1163.08124928,547.14128662)(1162.80391584,547.37595338)
\curveto(1162.54791574,547.63195348)(1162.41991569,547.97328695)(1162.41991569,548.39995378)
\curveto(1162.41991569,548.65595389)(1162.47324904,548.87995398)(1162.57991575,549.07195405)
\curveto(1162.7079158,549.28528747)(1162.81458251,549.42395419)(1162.89991588,549.48795422)
\curveto(1163.00658259,549.57328759)(1163.09191595,549.62662094)(1163.15591598,549.64795428)
\curveto(1163.11324929,549.66928762)(1162.99591591,549.67995429)(1162.80391584,549.67995429)
\curveto(1161.63058204,549.67995429)(1160.70258167,549.09328739)(1160.01991473,547.91995359)
\curveto(1159.35858113,546.74661979)(1159.02791434,545.32795256)(1159.02791434,543.6639519)
\lineto(1159.02791434,538.73594994)
\curveto(1159.02791434,538.11728303)(1159.15591439,537.71194953)(1159.41191449,537.51994945)
\curveto(1159.66791459,537.32794938)(1160.35058153,537.23194934)(1161.4599153,537.23194934)
\lineto(1162.13191557,537.23194934)
\lineto(1162.13191557,536.23994895)
\curveto(1161.2785819,536.30394897)(1159.92391469,536.33594898)(1158.06791395,536.33594898)
\curveto(1157.81191385,536.33594898)(1157.48124705,536.32528231)(1157.07591356,536.30394897)
\curveto(1156.67058006,536.30394897)(1156.20124654,536.2932823)(1155.667913,536.27194896)
\curveto(1155.13457945,536.25061562)(1154.71857929,536.23994895)(1154.4199125,536.23994895)
\closepath
}
}
{
\newrgbcolor{curcolor}{0 0 0}
\pscustom[linestyle=none,fillstyle=solid,fillcolor=curcolor]
{
\newpath
\moveto(1177.74782941,536.23994895)
\lineto(1177.74782941,537.23194934)
\curveto(1178.87849653,537.23194934)(1179.57183014,537.29594937)(1179.82783024,537.42394942)
\curveto(1180.10516368,537.57328281)(1180.2438304,537.98928297)(1180.2438304,538.67194991)
\lineto(1180.2438304,547.27995334)
\curveto(1180.2438304,548.06928699)(1180.10516368,548.55995385)(1179.82783024,548.75195392)
\curveto(1179.57183014,548.943954)(1178.92116321,549.03995404)(1177.87582946,549.03995404)
\lineto(1177.87582946,550.03195443)
\lineto(1182.35583125,550.38395457)
\lineto(1182.35583125,538.6399499)
\curveto(1182.35583125,537.99994965)(1182.46249795,537.60528282)(1182.67583137,537.45594943)
\curveto(1182.91049813,537.30661604)(1183.55049839,537.23194934)(1184.59583214,537.23194934)
\lineto(1184.59583214,536.23994895)
\curveto(1182.41983127,536.30394897)(1181.3104975,536.33594898)(1181.26783081,536.33594898)
\curveto(1180.96916403,536.33594898)(1179.79583023,536.30394897)(1177.74782941,536.23994895)
\closepath
\moveto(1179.09182995,555.95195679)
\curveto(1179.09182995,556.37862363)(1179.25183001,556.76262378)(1179.57183014,557.10395725)
\curveto(1179.91316361,557.46662406)(1180.3184971,557.64795747)(1180.78783062,557.64795747)
\curveto(1181.25716414,557.64795747)(1181.65183097,557.4879574)(1181.97183109,557.16795727)
\curveto(1182.31316456,556.84795715)(1182.4838313,556.44262365)(1182.4838313,555.95195679)
\curveto(1182.4838313,555.46128993)(1182.31316456,555.05595643)(1181.97183109,554.73595631)
\curveto(1181.65183097,554.41595618)(1181.25716414,554.25595612)(1180.78783062,554.25595612)
\curveto(1180.29716376,554.25595612)(1179.89183026,554.42662285)(1179.57183014,554.76795632)
\curveto(1179.25183001,555.10928979)(1179.09182995,555.50395661)(1179.09182995,555.95195679)
\closepath
}
}
{
\newrgbcolor{curcolor}{0 0 0}
\pscustom[linestyle=none,fillstyle=solid,fillcolor=curcolor]
{
\newpath
\moveto(1198.03573054,541.10395088)
\curveto(1198.05706388,541.35995098)(1198.2703973,541.48795103)(1198.6757308,541.48795103)
\lineto(1218.2917386,541.48795103)
\curveto(1218.91040552,541.48795103)(1219.21973897,541.37061765)(1219.21973897,541.13595089)
\curveto(1219.21973897,540.87995079)(1218.93173886,540.75195074)(1218.35573863,540.75195074)
\lineto(1198.89973089,540.75195074)
\curveto(1198.32373066,540.75195074)(1198.03573054,540.86928412)(1198.03573054,541.10395088)
\closepath
\moveto(1198.03573054,547.34395336)
\curveto(1198.03573054,547.59995347)(1198.25973063,547.72795352)(1198.70773081,547.72795352)
\lineto(1218.25973859,547.72795352)
\curveto(1218.89973885,547.72795352)(1219.21973897,547.59995347)(1219.21973897,547.34395336)
\curveto(1219.21973897,547.1092866)(1218.9530722,546.99195322)(1218.41973865,546.99195322)
\lineto(1198.6757308,546.99195322)
\curveto(1198.24906396,546.99195322)(1198.03573054,547.1092866)(1198.03573054,547.34395336)
\closepath
}
}
{
\newrgbcolor{curcolor}{0 0 0}
\pscustom[linestyle=none,fillstyle=solid,fillcolor=curcolor]
{
\newpath
\moveto(1232.8196899,536.23994895)
\lineto(1232.8196899,537.23194934)
\curveto(1233.5876902,537.23194934)(1234.21702379,537.30661604)(1234.70769065,537.45594943)
\curveto(1235.19835751,537.60528282)(1235.52902431,537.81861624)(1235.69969104,538.09594968)
\curveto(1235.89169112,538.39461647)(1236.0090245,538.6399499)(1236.05169118,538.83194998)
\curveto(1236.09435787,539.02395005)(1236.11569121,539.27995016)(1236.11569121,539.59995028)
\lineto(1236.11569121,556.30395693)
\curveto(1236.11569121,556.60262372)(1236.07302453,556.78395712)(1235.98769116,556.84795715)
\curveto(1235.92369113,556.93329051)(1235.71035771,556.99729054)(1235.3476909,557.03995722)
\curveto(1234.70769065,557.08262391)(1234.1743571,557.10395725)(1233.74769027,557.10395725)
\lineto(1232.8196899,557.10395725)
\lineto(1232.8196899,558.09595764)
\lineto(1238.25969206,558.09595764)
\curveto(1238.55835885,558.09595764)(1238.75035892,558.0746243)(1238.83569229,558.03195762)
\curveto(1238.92102566,557.98929093)(1239.03835904,557.87195755)(1239.18769243,557.67995748)
\lineto(1250.48369693,541.07195087)
\lineto(1250.48369693,554.73595631)
\curveto(1250.48369693,555.05595643)(1250.46236359,555.31195654)(1250.4196969,555.50395661)
\curveto(1250.37703022,555.69595669)(1250.25969684,555.93062345)(1250.06769676,556.20795689)
\curveto(1249.89703003,556.50662368)(1249.56636323,556.73062377)(1249.07569637,556.87995716)
\curveto(1248.58502951,557.02929055)(1247.95569592,557.10395725)(1247.18769562,557.10395725)
\lineto(1247.18769562,558.09595764)
\curveto(1249.42769651,558.03195762)(1250.675697,557.99995761)(1250.93169711,557.99995761)
\curveto(1251.18769721,557.99995761)(1252.4356977,558.03195762)(1254.6756986,558.09595764)
\lineto(1254.6756986,557.10395725)
\curveto(1253.90769829,557.10395725)(1253.27836471,557.02929055)(1252.78769784,556.87995716)
\curveto(1252.29703098,556.73062377)(1251.95569751,556.50662368)(1251.76369744,556.20795689)
\curveto(1251.5930307,555.93062345)(1251.48636399,555.69595669)(1251.44369731,555.50395661)
\curveto(1251.40103063,555.31195654)(1251.37969728,555.05595643)(1251.37969728,554.73595631)
\lineto(1251.37969728,537.07194928)
\curveto(1251.37969728,536.73061581)(1251.35836394,536.50661572)(1251.31569726,536.39994901)
\curveto(1251.27303057,536.2932823)(1251.14503052,536.23994895)(1250.93169711,536.23994895)
\curveto(1250.78236371,536.23994895)(1250.60103031,536.37861567)(1250.38769689,536.65594911)
\lineto(1237.33169169,555.85595675)
\curveto(1237.24635833,555.9839568)(1237.13969162,556.11195685)(1237.01169157,556.23995691)
\lineto(1237.01169157,539.59995028)
\curveto(1237.01169157,539.27995016)(1237.03302491,539.02395005)(1237.07569159,538.83194998)
\curveto(1237.11835827,538.6399499)(1237.22502498,538.39461647)(1237.39569172,538.09594968)
\curveto(1237.5876918,537.81861624)(1237.92902526,537.60528282)(1238.41969213,537.45594943)
\curveto(1238.91035899,537.30661604)(1239.53969257,537.23194934)(1240.30769288,537.23194934)
\lineto(1240.30769288,536.23994895)
\curveto(1238.06769199,536.30394897)(1236.81969149,536.33594898)(1236.56369139,536.33594898)
\curveto(1236.30769129,536.33594898)(1235.05969079,536.30394897)(1232.8196899,536.23994895)
\closepath
}
}
{
\newrgbcolor{curcolor}{0 0 0}
\pscustom[linestyle=none,fillstyle=solid,fillcolor=curcolor]
{
\newpath
\moveto(1069.9459403,86.08462649)
\lineto(1069.9459403,87.07662688)
\lineto(1070.71394061,87.07662688)
\curveto(1071.86594107,87.07662688)(1072.56994135,87.17262692)(1072.82594145,87.364627)
\curveto(1073.10327489,87.55662707)(1073.24194162,87.96196057)(1073.24194162,88.58062748)
\lineto(1073.24194162,105.44463419)
\curveto(1073.24194162,106.06330111)(1073.10327489,106.4686346)(1072.82594145,106.66063468)
\curveto(1072.56994135,106.85263475)(1071.86594107,106.94863479)(1070.71394061,106.94863479)
\lineto(1069.9459403,106.94863479)
\lineto(1069.9459403,107.94063519)
\lineto(1081.65794497,107.94063519)
\curveto(1083.68461244,107.94063519)(1085.36994644,107.39663497)(1086.71394698,106.30863454)
\curveto(1088.07928085,105.2206341)(1088.76194779,103.96196694)(1088.76194779,102.53263303)
\curveto(1088.76194779,101.35929923)(1088.26061426,100.30329881)(1087.25794719,99.36463177)
\curveto(1086.27661347,98.42596473)(1085.00727963,97.80729782)(1083.44994568,97.50863103)
\curveto(1085.22061305,97.31663096)(1086.69261364,96.68729737)(1087.86594744,95.62063028)
\curveto(1089.03928124,94.55396319)(1089.62594814,93.32729604)(1089.62594814,91.94062882)
\curveto(1089.62594814,90.38329486)(1088.95394787,89.01796099)(1087.60994733,87.84462719)
\curveto(1086.2659468,86.67129339)(1084.55927945,86.08462649)(1082.4899453,86.08462649)
\closepath
\moveto(1075.89794267,88.35662739)
\curveto(1075.89794267,87.8019605)(1075.97260937,87.44996036)(1076.12194276,87.30062697)
\curveto(1076.2926095,87.15129358)(1076.74060968,87.07662688)(1077.4659433,87.07662688)
\lineto(1081.46594489,87.07662688)
\curveto(1083.0019455,87.07662688)(1084.20727931,87.57796042)(1085.08194633,88.58062748)
\curveto(1085.97794669,89.58329455)(1086.42594686,90.71396166)(1086.42594686,91.97262883)
\curveto(1086.42594686,93.231296)(1086.03128004,94.39396313)(1085.24194639,95.46063022)
\curveto(1084.47394609,96.54863065)(1083.36461231,97.09263087)(1081.91394507,97.09263087)
\lineto(1075.89794267,97.09263087)
\closepath
\moveto(1075.89794267,97.79663115)
\lineto(1080.53794452,97.79663115)
\curveto(1082.18061184,97.79663115)(1083.43927901,98.28729801)(1084.31394602,99.26863173)
\curveto(1085.20994638,100.2712988)(1085.65794656,101.35929923)(1085.65794656,102.53263303)
\curveto(1085.65794656,103.57796678)(1085.30594642,104.56996718)(1084.60194614,105.50863422)
\curveto(1083.89794586,106.4686346)(1082.84194544,106.94863479)(1081.43394488,106.94863479)
\lineto(1077.4659433,106.94863479)
\curveto(1076.74060968,106.94863479)(1076.2926095,106.87396809)(1076.12194276,106.7246347)
\curveto(1075.97260937,106.57530131)(1075.89794267,106.22330117)(1075.89794267,105.66863428)
\closepath
}
}
{
\newrgbcolor{curcolor}{0 0 0}
\pscustom[linestyle=none,fillstyle=solid,fillcolor=curcolor]
{
\newpath
\moveto(1092.5058935,86.08462649)
\lineto(1092.5058935,87.07662688)
\curveto(1093.63656062,87.07662688)(1094.32989423,87.14062691)(1094.58589433,87.26862696)
\curveto(1094.86322777,87.41796035)(1095.00189449,87.83396052)(1095.00189449,88.51662746)
\lineto(1095.00189449,97.12463088)
\curveto(1095.00189449,97.91396453)(1094.86322777,98.40463139)(1094.58589433,98.59663147)
\curveto(1094.32989423,98.78863154)(1093.6792273,98.88463158)(1092.63389355,98.88463158)
\lineto(1092.63389355,99.87663198)
\lineto(1097.11389534,100.22863212)
\lineto(1097.11389534,88.48462744)
\curveto(1097.11389534,87.84462719)(1097.22056204,87.44996036)(1097.43389546,87.30062697)
\curveto(1097.66856222,87.15129358)(1098.30856248,87.07662688)(1099.35389623,87.07662688)
\lineto(1099.35389623,86.08462649)
\curveto(1097.17789536,86.14862651)(1096.06856159,86.18062653)(1096.0258949,86.18062653)
\curveto(1095.72722812,86.18062653)(1094.55389432,86.14862651)(1092.5058935,86.08462649)
\closepath
\moveto(1093.84989404,105.79663433)
\curveto(1093.84989404,106.22330117)(1094.0098941,106.60730132)(1094.32989423,106.94863479)
\curveto(1094.6712277,107.3113016)(1095.07656119,107.49263501)(1095.54589471,107.49263501)
\curveto(1096.01522823,107.49263501)(1096.40989505,107.33263494)(1096.72989518,107.01263482)
\curveto(1097.07122865,106.69263469)(1097.24189539,106.28730119)(1097.24189539,105.79663433)
\curveto(1097.24189539,105.30596747)(1097.07122865,104.90063398)(1096.72989518,104.58063385)
\curveto(1096.40989505,104.26063372)(1096.01522823,104.10063366)(1095.54589471,104.10063366)
\curveto(1095.05522785,104.10063366)(1094.64989435,104.27130039)(1094.32989423,104.61263386)
\curveto(1094.0098941,104.95396733)(1093.84989404,105.34863415)(1093.84989404,105.79663433)
\closepath
}
}
{
\newrgbcolor{curcolor}{0 0 0}
\pscustom[linestyle=none,fillstyle=solid,fillcolor=curcolor]
{
\newpath
\moveto(1101.36984902,86.08462649)
\lineto(1101.36984902,87.07662688)
\curveto(1102.50051614,87.07662688)(1103.19384975,87.14062691)(1103.44984985,87.26862696)
\curveto(1103.72718329,87.41796035)(1103.86585001,87.83396052)(1103.86585001,88.51662746)
\lineto(1103.86585001,105.15663408)
\curveto(1103.86585001,105.94596773)(1103.72718329,106.43663459)(1103.44984985,106.62863466)
\curveto(1103.17251641,106.84196808)(1102.4791828,106.94863479)(1101.36984902,106.94863479)
\lineto(1101.36984902,107.94063519)
\lineto(1105.97785086,108.29263533)
\lineto(1105.97785086,88.51662746)
\curveto(1105.97785086,87.83396052)(1106.10585091,87.41796035)(1106.36185101,87.26862696)
\curveto(1106.63918445,87.14062691)(1107.34318473,87.07662688)(1108.47385185,87.07662688)
\lineto(1108.47385185,86.08462649)
\curveto(1108.21785175,86.08462649)(1107.81251825,86.09529316)(1107.25785136,86.1166265)
\curveto(1106.72451782,86.13795984)(1106.2551843,86.14862651)(1105.8498508,86.14862651)
\curveto(1105.46585065,86.16995985)(1105.1565172,86.18062653)(1104.92185043,86.18062653)
\curveto(1104.66585033,86.18062653)(1103.48184986,86.14862651)(1101.36984902,86.08462649)
\closepath
}
}
{
\newrgbcolor{curcolor}{0 0 0}
\pscustom[linestyle=none,fillstyle=solid,fillcolor=curcolor]
{
\newpath
\moveto(1110.52180826,89.1246277)
\curveto(1110.52180826,90.93796175)(1111.58847535,92.32462897)(1113.72180953,93.28462935)
\curveto(1115.00181004,93.90329627)(1116.96447749,94.28729642)(1119.60981188,94.43662981)
\lineto(1119.60981188,95.62063028)
\curveto(1119.60981188,96.94329748)(1119.25781174,97.95663121)(1118.55381146,98.66063149)
\curveto(1117.87114452,99.36463177)(1117.09247754,99.71663191)(1116.21781053,99.71663191)
\curveto(1114.66047657,99.71663191)(1113.50847612,99.22596505)(1112.76180915,98.24463133)
\curveto(1113.40180941,98.22329799)(1113.82847624,98.05263125)(1114.04180966,97.73263112)
\curveto(1114.27647642,97.412631)(1114.3938098,97.09263087)(1114.3938098,96.77263074)
\curveto(1114.3938098,96.3459639)(1114.25514308,95.99396376)(1113.97780964,95.71663032)
\curveto(1113.72180953,95.43929688)(1113.36980939,95.30063016)(1112.92180922,95.30063016)
\curveto(1112.49514238,95.30063016)(1112.14314224,95.42863021)(1111.8658088,95.68463031)
\curveto(1111.58847535,95.96196375)(1111.44980863,96.33529723)(1111.44980863,96.80463075)
\curveto(1111.44980863,97.8499645)(1111.91914215,98.71396485)(1112.85780919,99.39663179)
\curveto(1113.79647623,100.07929872)(1114.93781002,100.42063219)(1116.28181055,100.42063219)
\curveto(1118.03114458,100.42063219)(1119.4924785,99.83396529)(1120.6658123,98.66063149)
\curveto(1121.02847911,98.29796468)(1121.29514588,97.88196452)(1121.46581262,97.412631)
\curveto(1121.65781269,96.94329748)(1121.7644794,96.54863065)(1121.78581274,96.22863052)
\curveto(1121.80714609,95.92996374)(1121.81781276,95.48196356)(1121.81781276,94.88462999)
\lineto(1121.81781276,88.48462744)
\curveto(1121.81781276,88.35662739)(1121.8391461,88.18596066)(1121.88181278,87.97262724)
\curveto(1121.92447947,87.78062716)(1122.04181285,87.5459604)(1122.23381292,87.26862696)
\curveto(1122.425813,87.01262686)(1122.6818131,86.88462681)(1123.00181323,86.88462681)
\curveto(1123.76981353,86.88462681)(1124.15381369,87.56729374)(1124.15381369,88.93262762)
\lineto(1124.15381369,90.72462833)
\lineto(1124.953814,90.72462833)
\lineto(1124.953814,88.93262762)
\curveto(1124.953814,87.8019605)(1124.65514722,87.01262686)(1124.05781365,86.56462668)
\curveto(1123.46048008,86.1166265)(1122.88447985,85.89262641)(1122.32981296,85.89262641)
\curveto(1121.62581268,85.89262641)(1121.04981245,86.14862651)(1120.60181227,86.66062672)
\curveto(1120.15381209,87.19396026)(1119.89781199,87.81262718)(1119.83381197,88.51662746)
\curveto(1119.51381184,87.70596047)(1118.98047829,87.0339602)(1118.23381133,86.50062665)
\curveto(1117.50847771,85.98862645)(1116.64447736,85.73262635)(1115.6418103,85.73262635)
\curveto(1114.87380999,85.73262635)(1114.12714303,85.82862639)(1113.40180941,86.02062646)
\curveto(1112.67647578,86.21262654)(1112.01514219,86.57529335)(1111.41780862,87.10862689)
\curveto(1110.82047505,87.64196044)(1110.52180826,88.31396071)(1110.52180826,89.1246277)
\closepath
\moveto(1112.98580924,89.15662771)
\curveto(1112.98580924,88.36729406)(1113.26314269,87.71662714)(1113.81780957,87.20462693)
\curveto(1114.3938098,86.69262673)(1115.07647674,86.43662663)(1115.86581039,86.43662663)
\curveto(1116.76181074,86.43662663)(1117.60447775,86.7779601)(1118.39381139,87.46062704)
\curveto(1119.20447838,88.16462732)(1119.60981188,89.19929439)(1119.60981188,90.56462827)
\lineto(1119.60981188,93.76462954)
\curveto(1117.24181094,93.67929618)(1115.54581026,93.16729597)(1114.52180985,92.22862893)
\curveto(1113.49780945,91.28996189)(1112.98580924,90.26596148)(1112.98580924,89.15662771)
\closepath
}
}
{
\newrgbcolor{curcolor}{0 0 0}
\pscustom[linestyle=none,fillstyle=solid,fillcolor=curcolor]
{
\newpath
\moveto(1124.88969509,82.11662491)
\curveto(1124.88969509,82.56462509)(1125.01769514,82.90595856)(1125.27369524,83.14062532)
\curveto(1125.55102869,83.37529208)(1125.87102881,83.49262546)(1126.23369563,83.49262546)
\curveto(1126.61769578,83.49262546)(1126.93769591,83.37529208)(1127.19369601,83.14062532)
\curveto(1127.44969611,82.88462521)(1127.57769616,82.55395842)(1127.57769616,82.14862492)
\curveto(1127.57769616,81.35929127)(1127.16169599,80.9112911)(1126.32969566,80.80462439)
\curveto(1126.73502916,80.42062423)(1127.23636269,80.22862416)(1127.83369626,80.22862416)
\curveto(1128.47369652,80.22862416)(1129.04969675,80.46329092)(1129.56169695,80.93262444)
\curveto(1130.07369715,81.40195796)(1130.43636396,81.86062481)(1130.64969738,82.30862498)
\curveto(1130.88436414,82.75662516)(1131.16169759,83.38595875)(1131.48169771,84.19662574)
\curveto(1131.7803645,84.83662599)(1132.04703127,85.46595957)(1132.28169803,86.08462649)
\lineto(1127.48169612,97.76463114)
\curveto(1127.2683627,98.27663134)(1127.00169593,98.5859648)(1126.6816958,98.69263151)
\curveto(1126.36169568,98.82063156)(1125.7643621,98.88463158)(1124.88969509,98.88463158)
\lineto(1124.88969509,99.87663198)
\curveto(1125.84969547,99.81263195)(1126.88436255,99.78063194)(1127.99369633,99.78063194)
\curveto(1128.65502992,99.78063194)(1129.81769705,99.81263195)(1131.48169771,99.87663198)
\lineto(1131.48169771,98.88463158)
\curveto(1130.28703057,98.88463158)(1129.689697,98.60729814)(1129.689697,98.05263125)
\curveto(1129.689697,97.98863123)(1129.75369703,97.79663115)(1129.88169708,97.47663102)
\lineto(1133.43369849,88.8686276)
\lineto(1136.66569978,96.74063073)
\curveto(1136.79369983,97.03929751)(1136.85769985,97.29529762)(1136.85769985,97.50863103)
\curveto(1136.85769985,98.38329805)(1136.35636632,98.8419649)(1135.35369925,98.88463158)
\lineto(1135.35369925,99.87663198)
\curveto(1136.76169982,99.81263195)(1137.68970018,99.78063194)(1138.13770036,99.78063194)
\curveto(1139.01236738,99.78063194)(1139.8123677,99.81263195)(1140.53770132,99.87663198)
\lineto(1140.53770132,98.88463158)
\curveto(1139.10836742,98.88463158)(1138.11636702,98.20196464)(1137.56170013,96.83663077)
\lineto(1131.92969789,83.17262533)
\curveto(1130.90569748,80.74062436)(1129.54036361,79.52462388)(1127.83369626,79.52462388)
\curveto(1127.02302927,79.52462388)(1126.32969566,79.78062398)(1125.75369543,80.29262418)
\curveto(1125.1776952,80.78329104)(1124.88969509,81.39129129)(1124.88969509,82.11662491)
\closepath
}
}
{
\newrgbcolor{curcolor}{0 0 0}
\pscustom[linestyle=none,fillstyle=solid,fillcolor=curcolor]
{
\newpath
\moveto(1141.14559504,93.12462929)
\curveto(1141.14559504,95.12996342)(1141.78559529,96.84729744)(1143.0655958,98.27663134)
\curveto(1144.36692965,99.70596524)(1145.94559695,100.42063219)(1147.80159769,100.42063219)
\curveto(1149.67893177,100.42063219)(1151.097599,99.81263195)(1152.05759938,98.59663147)
\curveto(1153.0389331,97.38063098)(1153.52959997,95.89796373)(1153.52959997,94.1486297)
\curveto(1153.52959997,93.82862957)(1153.48693328,93.63662949)(1153.40159992,93.57262947)
\curveto(1153.31626655,93.50862944)(1153.09226646,93.47662943)(1152.72959965,93.47662943)
\lineto(1143.8015961,93.47662943)
\curveto(1143.8015961,91.30062856)(1144.12159622,89.70062793)(1144.76159648,88.67662752)
\curveto(1145.65759683,87.24729362)(1146.86293065,86.53262667)(1148.37759792,86.53262667)
\curveto(1148.59093133,86.53262667)(1148.81493142,86.55396001)(1149.04959818,86.59662669)
\curveto(1149.30559829,86.63929337)(1149.68959844,86.74596008)(1150.20159864,86.91662682)
\curveto(1150.71359885,87.10862689)(1151.19359904,87.44996036)(1151.64159922,87.94062723)
\curveto(1152.08959939,88.43129409)(1152.44159953,89.06062767)(1152.69759964,89.82862798)
\curveto(1152.76159966,90.12729476)(1152.90026638,90.27662816)(1153.1135998,90.27662816)
\curveto(1153.39093324,90.27662816)(1153.52959997,90.1486281)(1153.52959997,89.892628)
\curveto(1153.52959997,89.70062793)(1153.42293326,89.39129447)(1153.20959984,88.96462763)
\curveto(1153.01759976,88.55929414)(1152.72959965,88.10062729)(1152.3455995,87.58862709)
\curveto(1151.96159934,87.09796022)(1151.39626578,86.66062672)(1150.64959882,86.27662656)
\curveto(1149.90293186,85.91395975)(1149.0815982,85.73262635)(1148.18559784,85.73262635)
\curveto(1146.30826376,85.73262635)(1144.66559644,86.43662663)(1143.25759588,87.84462719)
\curveto(1141.84959532,89.27396109)(1141.14559504,91.03396179)(1141.14559504,93.12462929)
\closepath
\moveto(1143.83359611,94.1486297)
\lineto(1151.41759913,94.1486297)
\curveto(1151.41759913,94.59662988)(1151.37493244,95.0659634)(1151.28959908,95.55663026)
\curveto(1151.22559905,96.06863046)(1151.07626566,96.66596403)(1150.8415989,97.34863097)
\curveto(1150.62826548,98.05263125)(1150.254932,98.61796481)(1149.72159845,99.04463165)
\curveto(1149.20959825,99.49263182)(1148.56959799,99.71663191)(1147.80159769,99.71663191)
\curveto(1147.46026422,99.71663191)(1147.09759741,99.64196522)(1146.71359725,99.49263182)
\curveto(1146.35093044,99.34329843)(1145.94559695,99.08729833)(1145.49759677,98.72463152)
\curveto(1145.04959659,98.38329805)(1144.66559644,97.80729782)(1144.34559631,96.99663083)
\curveto(1144.04692953,96.20729718)(1143.87626279,95.25796347)(1143.83359611,94.1486297)
\closepath
}
}
{
\newrgbcolor{curcolor}{0 0 0}
\pscustom[linestyle=none,fillstyle=solid,fillcolor=curcolor]
{
\newpath
\moveto(1155.35354854,86.08462649)
\lineto(1155.35354854,87.07662688)
\curveto(1156.48421566,87.07662688)(1157.17754927,87.14062691)(1157.43354937,87.26862696)
\curveto(1157.71088282,87.41796035)(1157.84954954,87.83396052)(1157.84954954,88.51662746)
\lineto(1157.84954954,97.09263087)
\curveto(1157.84954954,97.88196452)(1157.71088282,98.37263138)(1157.43354937,98.56463145)
\curveto(1157.15621593,98.77796487)(1156.46288232,98.88463158)(1155.35354854,98.88463158)
\lineto(1155.35354854,99.87663198)
\lineto(1159.80155031,100.22863212)
\lineto(1159.80155031,96.70863072)
\curveto(1160.1002171,97.62596441)(1160.56955062,98.4366314)(1161.20955087,99.14063168)
\curveto(1161.84955113,99.86596531)(1162.69221813,100.22863212)(1163.73755188,100.22863212)
\curveto(1164.42021882,100.22863212)(1164.98555238,100.03663204)(1165.43355256,99.65263189)
\curveto(1165.88155273,99.26863173)(1166.10555282,98.80996489)(1166.10555282,98.27663134)
\curveto(1166.10555282,97.80729782)(1165.95621943,97.45529768)(1165.65755264,97.22063092)
\curveto(1165.3802192,96.98596416)(1165.07088574,96.86863078)(1164.72955228,96.86863078)
\curveto(1164.34555212,96.86863078)(1164.01488532,96.98596416)(1163.73755188,97.22063092)
\curveto(1163.48155178,97.47663102)(1163.35355173,97.81796449)(1163.35355173,98.24463133)
\curveto(1163.35355173,98.50063143)(1163.40688508,98.72463152)(1163.51355179,98.91663159)
\curveto(1163.64155184,99.12996501)(1163.74821855,99.26863173)(1163.83355192,99.33263176)
\curveto(1163.94021863,99.41796513)(1164.02555199,99.47129848)(1164.08955202,99.49263182)
\curveto(1164.04688534,99.51396517)(1163.92955196,99.52463184)(1163.73755188,99.52463184)
\curveto(1162.56421808,99.52463184)(1161.63621771,98.93796494)(1160.95355077,97.76463114)
\curveto(1160.29221718,96.59129734)(1159.96155038,95.1726301)(1159.96155038,93.50862944)
\lineto(1159.96155038,88.58062748)
\curveto(1159.96155038,87.96196057)(1160.08955043,87.55662707)(1160.34555053,87.364627)
\curveto(1160.60155063,87.17262692)(1161.28421757,87.07662688)(1162.39355135,87.07662688)
\lineto(1163.06555161,87.07662688)
\lineto(1163.06555161,86.08462649)
\curveto(1162.21221794,86.14862651)(1160.85755073,86.18062653)(1159.00155,86.18062653)
\curveto(1158.74554989,86.18062653)(1158.4148831,86.16995985)(1158.0095496,86.14862651)
\curveto(1157.60421611,86.14862651)(1157.13488259,86.13795984)(1156.60154904,86.1166265)
\curveto(1156.06821549,86.09529316)(1155.65221533,86.08462649)(1155.35354854,86.08462649)
\closepath
}
}
{
\newrgbcolor{curcolor}{0 0 0}
\pscustom[linestyle=none,fillstyle=solid,fillcolor=curcolor]
{
\newpath
\moveto(1178.68146545,86.08462649)
\lineto(1178.68146545,87.07662688)
\curveto(1179.81213257,87.07662688)(1180.50546618,87.14062691)(1180.76146628,87.26862696)
\curveto(1181.03879972,87.41796035)(1181.17746645,87.83396052)(1181.17746645,88.51662746)
\lineto(1181.17746645,97.12463088)
\curveto(1181.17746645,97.91396453)(1181.03879972,98.40463139)(1180.76146628,98.59663147)
\curveto(1180.50546618,98.78863154)(1179.85479925,98.88463158)(1178.8094655,98.88463158)
\lineto(1178.8094655,99.87663198)
\lineto(1183.28946729,100.22863212)
\lineto(1183.28946729,88.48462744)
\curveto(1183.28946729,87.84462719)(1183.396134,87.44996036)(1183.60946741,87.30062697)
\curveto(1183.84413417,87.15129358)(1184.48413443,87.07662688)(1185.52946818,87.07662688)
\lineto(1185.52946818,86.08462649)
\curveto(1183.35346731,86.14862651)(1182.24413354,86.18062653)(1182.20146685,86.18062653)
\curveto(1181.90280007,86.18062653)(1180.72946627,86.14862651)(1178.68146545,86.08462649)
\closepath
\moveto(1180.02546599,105.79663433)
\curveto(1180.02546599,106.22330117)(1180.18546605,106.60730132)(1180.50546618,106.94863479)
\curveto(1180.84679965,107.3113016)(1181.25213314,107.49263501)(1181.72146666,107.49263501)
\curveto(1182.19080018,107.49263501)(1182.58546701,107.33263494)(1182.90546713,107.01263482)
\curveto(1183.2468006,106.69263469)(1183.41746734,106.28730119)(1183.41746734,105.79663433)
\curveto(1183.41746734,105.30596747)(1183.2468006,104.90063398)(1182.90546713,104.58063385)
\curveto(1182.58546701,104.26063372)(1182.19080018,104.10063366)(1181.72146666,104.10063366)
\curveto(1181.2307998,104.10063366)(1180.82546631,104.27130039)(1180.50546618,104.61263386)
\curveto(1180.18546605,104.95396733)(1180.02546599,105.34863415)(1180.02546599,105.79663433)
\closepath
}
}
{
\newrgbcolor{curcolor}{0 0 0}
\pscustom[linestyle=none,fillstyle=solid,fillcolor=curcolor]
{
\newpath
\moveto(1198.96936658,90.94862842)
\curveto(1198.99069993,91.20462853)(1199.20403334,91.33262858)(1199.60936684,91.33262858)
\lineto(1219.22537465,91.33262858)
\curveto(1219.84404156,91.33262858)(1220.15337502,91.2152952)(1220.15337502,90.98062844)
\curveto(1220.15337502,90.72462833)(1219.8653749,90.59662828)(1219.28937467,90.59662828)
\lineto(1199.83336693,90.59662828)
\curveto(1199.2573667,90.59662828)(1198.96936658,90.71396166)(1198.96936658,90.94862842)
\closepath
\moveto(1198.96936658,97.18863091)
\curveto(1198.96936658,97.44463101)(1199.19336667,97.57263106)(1199.64136685,97.57263106)
\lineto(1219.19337463,97.57263106)
\curveto(1219.83337489,97.57263106)(1220.15337502,97.44463101)(1220.15337502,97.18863091)
\curveto(1220.15337502,96.95396415)(1219.88670824,96.83663077)(1219.3533747,96.83663077)
\lineto(1199.60936684,96.83663077)
\curveto(1199.1827,96.83663077)(1198.96936658,96.95396415)(1198.96936658,97.18863091)
\closepath
}
}
{
\newrgbcolor{curcolor}{0 0 0}
\pscustom[linestyle=none,fillstyle=solid,fillcolor=curcolor]
{
\newpath
\moveto(1235.54532665,104.35663376)
\lineto(1235.54532665,105.34863415)
\curveto(1238.10532767,105.34863415)(1240.04666178,106.03130109)(1241.36932897,107.39663497)
\curveto(1241.73199578,107.39663497)(1241.9453292,107.35396829)(1242.00932923,107.26863492)
\curveto(1242.07332925,107.18330155)(1242.10532926,106.94863479)(1242.10532926,106.56463464)
\lineto(1242.10532926,88.61262749)
\curveto(1242.10532926,87.97262724)(1242.25466266,87.55662707)(1242.55332944,87.364627)
\curveto(1242.87332957,87.17262692)(1243.71599657,87.07662688)(1245.08133045,87.07662688)
\lineto(1246.10533086,87.07662688)
\lineto(1246.10533086,86.08462649)
\curveto(1245.35866389,86.14862651)(1243.6306632,86.18062653)(1240.92132879,86.18062653)
\curveto(1238.21199438,86.18062653)(1236.48399369,86.14862651)(1235.73732673,86.08462649)
\lineto(1235.73732673,87.07662688)
\lineto(1236.76132714,87.07662688)
\curveto(1238.10532767,87.07662688)(1238.937328,87.17262692)(1239.25732813,87.364627)
\curveto(1239.57732826,87.55662707)(1239.73732832,87.97262724)(1239.73732832,88.61262749)
\lineto(1239.73732832,105.18863409)
\curveto(1238.62799455,104.6339672)(1237.23066066,104.35663376)(1235.54532665,104.35663376)
\closepath
}
}
{
\newrgbcolor{curcolor}{0 0 0}
\pscustom[linewidth=1.89354326,linecolor=curcolor]
{
\newpath
\moveto(1046.91468094,117.89709049)
\lineto(1051.91469354,117.89709049)
\lineto(1051.91469354,71.89709175)
\lineto(1046.91468094,71.89709175)
}
}
{
\newrgbcolor{curcolor}{0 0 0}
\pscustom[linewidth=1.89354326,linecolor=curcolor]
{
\newpath
\moveto(1051.91469354,92.89709931)
\lineto(1056.91470614,92.89709931)
}
}
{
\newrgbcolor{curcolor}{0 0 0}
\pscustom[linestyle=none,fillstyle=solid,fillcolor=curcolor]
{
\newpath
\moveto(159.11320564,21.15642171)
\lineto(159.11320564,27.55642426)
\lineto(159.11320564,28.03642445)
\curveto(159.13453898,28.12175782)(159.17720567,28.19642451)(159.24120569,28.26042454)
\curveto(159.30520572,28.32442456)(159.40120576,28.35642458)(159.52920581,28.35642458)
\curveto(159.76387257,28.35642458)(159.89187262,28.24975787)(159.91320596,28.03642445)
\curveto(159.97720599,25.92442361)(160.63853958,24.28175629)(161.89720675,23.10842249)
\curveto(163.13454058,21.95642203)(164.81987458,21.3804218)(166.95320876,21.3804218)
\curveto(168.23320927,21.3804218)(169.26787635,21.83908865)(170.05721,22.75642235)
\curveto(170.84654365,23.67375605)(171.24121047,24.74042314)(171.24121047,25.95642362)
\curveto(171.24121047,27.10842408)(170.91054367,28.0897578)(170.24921007,28.90042479)
\curveto(169.97187663,29.2630916)(169.6518765,29.56175839)(169.28920969,29.79642515)
\curveto(168.94787622,30.03109191)(168.67054278,30.1804253)(168.45720936,30.24442533)
\curveto(168.24387594,30.3297587)(167.93454249,30.42575873)(167.52920899,30.53242544)
\curveto(164.88387461,31.1724257)(163.47587405,31.52442584)(163.30520731,31.58842586)
\curveto(162.04654014,32.0150927)(161.03320641,32.76175966)(160.2652061,33.82842675)
\curveto(159.4972058,34.91642719)(159.11320564,36.121761)(159.11320564,37.44442819)
\curveto(159.11320564,39.15109554)(159.71053921,40.61242945)(160.90520636,41.82842994)
\curveto(162.12120684,43.04443042)(163.6038741,43.65243066)(165.35320813,43.65243066)
\curveto(166.22787514,43.65243066)(167.02787546,43.51376394)(167.75320908,43.2364305)
\curveto(168.49987605,42.95909706)(169.02254292,42.69243028)(169.32120971,42.43643018)
\curveto(169.61987649,42.20176342)(170.00387664,41.86042995)(170.47321016,41.41242977)
\lineto(171.59321061,43.2364305)
\curveto(171.76387734,43.51376394)(171.92387741,43.65243066)(172.0732108,43.65243066)
\curveto(172.24387754,43.65243066)(172.33987757,43.59909731)(172.36121092,43.4924306)
\curveto(172.4038776,43.40709723)(172.42521094,43.20443049)(172.42521094,42.88443036)
\lineto(172.42521094,36.4524278)
\curveto(172.42521094,36.21776104)(172.41454427,36.05776098)(172.39321093,35.97242761)
\curveto(172.39321093,35.88709424)(172.36121092,35.81242754)(172.29721089,35.74842752)
\curveto(172.23321086,35.70576084)(172.13721083,35.68442749)(172.00921078,35.68442749)
\curveto(171.79587736,35.68442749)(171.66787731,35.81242754)(171.62521062,36.06842765)
\curveto(171.02787705,40.52709609)(168.94787622,42.75643031)(165.38520814,42.75643031)
\curveto(164.16920765,42.75643031)(163.15587392,42.34043014)(162.34520693,41.50842981)
\curveto(161.55587328,40.69776282)(161.16120646,39.7164291)(161.16120646,38.56442864)
\curveto(161.16120646,37.6257616)(161.44920657,36.77242793)(162.0252068,36.00442762)
\curveto(162.62254037,35.23642732)(163.41187402,34.73509378)(164.39320774,34.50042702)
\lineto(168.48920937,33.50842663)
\curveto(169.85454325,33.1884265)(170.99587704,32.42042619)(171.91321074,31.20442571)
\curveto(172.83054444,30.00975857)(173.28921128,28.62309135)(173.28921128,27.04442405)
\curveto(173.28921128,25.23109)(172.69187771,23.66308938)(171.49721057,22.34042218)
\curveto(170.32387677,21.03908833)(168.81987617,20.38842141)(166.98520878,20.38842141)
\curveto(165.66254158,20.38842141)(164.48920778,20.6124215)(163.46520737,21.06042167)
\curveto(162.44120697,21.48708851)(161.65187332,22.00975538)(161.09720643,22.6284223)
\curveto(160.6705396,21.92442202)(160.31853946,21.36975513)(160.04120601,20.96442164)
\lineto(159.94520597,20.80442157)
\curveto(159.77453924,20.52708813)(159.61453918,20.38842141)(159.46520578,20.38842141)
\curveto(159.29453905,20.38842141)(159.18787234,20.44175476)(159.14520566,20.54842147)
\curveto(159.12387231,20.65508818)(159.11320564,20.85775493)(159.11320564,21.15642171)
\closepath
}
}
{
\newrgbcolor{curcolor}{0 0 0}
\pscustom[linestyle=none,fillstyle=solid,fillcolor=curcolor]
{
\newpath
\moveto(176.13716165,21.09242169)
\lineto(176.13716165,22.08442208)
\curveto(177.26782877,22.08442208)(177.96116238,22.14842211)(178.21716248,22.27642216)
\curveto(178.49449592,22.42575555)(178.63316264,22.84175572)(178.63316264,23.52442265)
\lineto(178.63316264,32.13242608)
\curveto(178.63316264,32.92175973)(178.49449592,33.41242659)(178.21716248,33.60442667)
\curveto(177.96116238,33.79642674)(177.31049545,33.89242678)(176.2651617,33.89242678)
\lineto(176.2651617,34.88442718)
\lineto(180.74516348,35.23642732)
\lineto(180.74516348,23.49242264)
\curveto(180.74516348,22.85242239)(180.85183019,22.45775556)(181.06516361,22.30842217)
\curveto(181.29983037,22.15908878)(181.93983063,22.08442208)(182.98516437,22.08442208)
\lineto(182.98516437,21.09242169)
\curveto(180.80916351,21.15642171)(179.69982973,21.18842172)(179.65716305,21.18842172)
\curveto(179.35849626,21.18842172)(178.18516246,21.15642171)(176.13716165,21.09242169)
\closepath
\moveto(177.48116218,40.80442953)
\curveto(177.48116218,41.23109637)(177.64116225,41.61509652)(177.96116238,41.95642999)
\curveto(178.30249584,42.3190968)(178.70782934,42.50043021)(179.17716286,42.50043021)
\curveto(179.64649638,42.50043021)(180.0411632,42.34043014)(180.36116333,42.02043002)
\curveto(180.7024968,41.70042989)(180.87316353,41.29509639)(180.87316353,40.80442953)
\curveto(180.87316353,40.31376267)(180.7024968,39.90842917)(180.36116333,39.58842905)
\curveto(180.0411632,39.26842892)(179.64649638,39.10842886)(179.17716286,39.10842886)
\curveto(178.686496,39.10842886)(178.2811625,39.27909559)(177.96116238,39.62042906)
\curveto(177.64116225,39.96176253)(177.48116218,40.35642935)(177.48116218,40.80442953)
\closepath
}
}
{
\newrgbcolor{curcolor}{0 0 0}
\pscustom[linestyle=none,fillstyle=solid,fillcolor=curcolor]
{
\newpath
\moveto(118.59967395,56.31223918)
\lineto(118.59967395,62.71224173)
\lineto(118.59967395,63.19224192)
\curveto(118.62100729,63.27757528)(118.66367398,63.35224198)(118.727674,63.41624201)
\curveto(118.79167403,63.48024203)(118.88767406,63.51224204)(119.01567412,63.51224204)
\curveto(119.25034088,63.51224204)(119.37834093,63.40557534)(119.39967427,63.19224192)
\curveto(119.46367429,61.08024108)(120.12500789,59.43757376)(121.38367506,58.26423996)
\curveto(122.62100888,57.1122395)(124.30634289,56.53623927)(126.43967707,56.53623927)
\curveto(127.71967758,56.53623927)(128.75434466,56.99490612)(129.54367831,57.91223982)
\curveto(130.33301195,58.82957351)(130.72767878,59.89624061)(130.72767878,61.11224109)
\curveto(130.72767878,62.26424155)(130.39701198,63.24557527)(129.73567838,64.05624226)
\curveto(129.45834494,64.41890907)(129.13834481,64.71757586)(128.775678,64.95224262)
\curveto(128.43434453,65.18690938)(128.15701109,65.33624277)(127.94367767,65.4002428)
\curveto(127.73034425,65.48557616)(127.42101079,65.5815762)(127.0156773,65.68824291)
\curveto(124.37034291,66.32824317)(122.96234235,66.68024331)(122.79167562,66.74424333)
\curveto(121.53300845,67.17091017)(120.51967471,67.91757713)(119.75167441,68.98424422)
\curveto(118.9836741,70.07224466)(118.59967395,71.27757847)(118.59967395,72.60024566)
\curveto(118.59967395,74.30691301)(119.19700752,75.76824692)(120.39167466,76.98424741)
\curveto(121.60767515,78.20024789)(123.0903424,78.80824813)(124.83967643,78.80824813)
\curveto(125.71434345,78.80824813)(126.51434377,78.66958141)(127.23967739,78.39224797)
\curveto(127.98634435,78.11491452)(128.50901123,77.84824775)(128.80767801,77.59224765)
\curveto(129.1063448,77.35758089)(129.49034495,77.01624742)(129.95967847,76.56824724)
\lineto(131.07967892,78.39224797)
\curveto(131.25034565,78.66958141)(131.41034572,78.80824813)(131.55967911,78.80824813)
\curveto(131.73034584,78.80824813)(131.82634588,78.75491478)(131.84767922,78.64824807)
\curveto(131.89034591,78.5629147)(131.91167925,78.36024795)(131.91167925,78.04024783)
\lineto(131.91167925,71.60824527)
\curveto(131.91167925,71.37357851)(131.90101258,71.21357844)(131.87967924,71.12824508)
\curveto(131.87967924,71.04291171)(131.84767922,70.96824501)(131.7836792,70.90424499)
\curveto(131.71967917,70.8615783)(131.62367913,70.84024496)(131.49567908,70.84024496)
\curveto(131.28234566,70.84024496)(131.15434561,70.96824501)(131.11167893,71.22424511)
\curveto(130.51434536,75.68291356)(128.43434453,77.91224778)(124.87167645,77.91224778)
\curveto(123.65567596,77.91224778)(122.64234223,77.49624761)(121.83167524,76.66424728)
\curveto(121.04234159,75.85358029)(120.64767477,74.87224657)(120.64767477,73.72024611)
\curveto(120.64767477,72.78157907)(120.93567488,71.92824539)(121.51167511,71.16024509)
\curveto(122.10900868,70.39224478)(122.89834233,69.89091125)(123.87967605,69.65624449)
\lineto(127.97567768,68.66424409)
\curveto(129.34101156,68.34424397)(130.48234535,67.57624366)(131.39967904,66.36024318)
\curveto(132.31701274,65.16557604)(132.77567959,63.77890882)(132.77567959,62.20024152)
\curveto(132.77567959,60.38690747)(132.17834602,58.81890684)(130.98367888,57.49623965)
\curveto(129.81034508,56.1949058)(128.30634448,55.54423887)(126.47167708,55.54423887)
\curveto(125.14900989,55.54423887)(123.97567609,55.76823896)(122.95167568,56.21623914)
\curveto(121.92767527,56.64290598)(121.13834163,57.16557285)(120.58367474,57.78423976)
\curveto(120.1570079,57.08023948)(119.80500776,56.5255726)(119.52767432,56.1202391)
\lineto(119.43167428,55.96023904)
\curveto(119.26100755,55.6829056)(119.10100748,55.54423887)(118.95167409,55.54423887)
\curveto(118.78100736,55.54423887)(118.67434065,55.59757223)(118.63167396,55.70423894)
\curveto(118.61034062,55.81090565)(118.59967395,56.01357239)(118.59967395,56.31223918)
\closepath
}
}
{
\newrgbcolor{curcolor}{0 0 0}
\pscustom[linestyle=none,fillstyle=solid,fillcolor=curcolor]
{
\newpath
\moveto(135.62362996,56.24823915)
\lineto(135.62362996,57.24023955)
\curveto(136.75429707,57.24023955)(137.44763068,57.30423957)(137.70363078,57.43223962)
\curveto(137.98096423,57.58157302)(138.11963095,57.99757318)(138.11963095,58.68024012)
\lineto(138.11963095,67.28824355)
\curveto(138.11963095,68.07757719)(137.98096423,68.56824406)(137.70363078,68.76024413)
\curveto(137.44763068,68.95224421)(136.79696376,69.04824425)(135.75163001,69.04824425)
\lineto(135.75163001,70.04024464)
\lineto(140.23163179,70.39224478)
\lineto(140.23163179,58.64824011)
\curveto(140.23163179,58.00823985)(140.3382985,57.61357303)(140.55163192,57.46423964)
\curveto(140.78629868,57.31490624)(141.42629893,57.24023955)(142.47163268,57.24023955)
\lineto(142.47163268,56.24823915)
\curveto(140.29563182,56.31223918)(139.18629804,56.34423919)(139.14363136,56.34423919)
\curveto(138.84496457,56.34423919)(137.67163077,56.31223918)(135.62362996,56.24823915)
\closepath
\moveto(136.96763049,75.960247)
\curveto(136.96763049,76.38691384)(137.12763056,76.77091399)(137.44763068,77.11224746)
\curveto(137.78896415,77.47491427)(138.19429765,77.65624767)(138.66363117,77.65624767)
\curveto(139.13296469,77.65624767)(139.52763151,77.49624761)(139.84763164,77.17624748)
\curveto(140.18896511,76.85624736)(140.35963184,76.45091386)(140.35963184,75.960247)
\curveto(140.35963184,75.46958014)(140.18896511,75.06424664)(139.84763164,74.74424651)
\curveto(139.52763151,74.42424639)(139.13296469,74.26424632)(138.66363117,74.26424632)
\curveto(138.1729643,74.26424632)(137.76763081,74.43491306)(137.44763068,74.77624653)
\curveto(137.12763056,75.11758)(136.96763049,75.51224682)(136.96763049,75.960247)
\closepath
}
}
{
\newrgbcolor{curcolor}{0 0 0}
\pscustom[linestyle=none,fillstyle=solid,fillcolor=curcolor]
{
\newpath
\moveto(145.22358757,67.09624347)
\curveto(145.22358757,70.36024477)(146.26892132,73.12291254)(148.35958882,75.38424677)
\curveto(150.45025632,77.66691434)(152.94625731,78.80824813)(155.8475918,78.80824813)
\curveto(158.77025963,78.80824813)(161.27692729,77.66691434)(163.36759479,75.38424677)
\curveto(165.45826229,73.10157919)(166.50359604,70.33891143)(166.50359604,67.09624347)
\curveto(166.50359604,63.87490886)(165.45826229,61.1442411)(163.36759479,58.90424021)
\curveto(161.27692729,56.66423932)(158.77025963,55.54423887)(155.8475918,55.54423887)
\curveto(152.96759065,55.54423887)(150.47158966,56.65357265)(148.35958882,58.8722402)
\curveto(146.26892132,61.11224109)(145.22358757,63.85357551)(145.22358757,67.09624347)
\closepath
\moveto(148.51958888,67.54424365)
\curveto(148.51958888,65.56024286)(148.75425564,63.81090883)(149.22358916,62.29624156)
\curveto(149.69292268,60.78157429)(150.30092293,59.60824049)(151.04758989,58.77624016)
\curveto(151.8155902,57.96557317)(152.60492384,57.35757293)(153.41559083,56.95223943)
\curveto(154.22625782,56.56823928)(155.04759148,56.3762392)(155.87959181,56.3762392)
\curveto(156.6902588,56.3762392)(157.49025912,56.56823928)(158.27959277,56.95223943)
\curveto(159.09025976,57.33623959)(159.86892673,57.93357316)(160.6155937,58.74424015)
\curveto(161.383594,59.55490714)(162.00226092,60.71757427)(162.47159444,62.23224154)
\curveto(162.9622613,63.76824215)(163.20759473,65.53890952)(163.20759473,67.54424365)
\curveto(163.20759473,69.1015776)(163.036928,70.50957816)(162.69559453,71.76824533)
\curveto(162.3755944,73.0269125)(161.94892756,74.02957956)(161.41559402,74.77624653)
\curveto(160.90359381,75.54424683)(160.30626024,76.17358042)(159.6235933,76.66424728)
\curveto(158.96225971,77.17624748)(158.31159278,77.52824762)(157.67159253,77.7202477)
\curveto(157.05292561,77.91224778)(156.44492537,78.00824781)(155.8475918,78.00824781)
\curveto(155.10092484,78.00824781)(154.33292453,77.83758108)(153.54359088,77.49624761)
\curveto(152.77559058,77.17624748)(151.9969236,76.64291394)(151.20758995,75.89624697)
\curveto(150.41825631,75.17091335)(149.76758938,74.08291292)(149.25558918,72.63224567)
\curveto(148.76492232,71.20291177)(148.51958888,69.5069111)(148.51958888,67.54424365)
\closepath
}
}
{
\newrgbcolor{curcolor}{0 0 0}
\pscustom[linestyle=none,fillstyle=solid,fillcolor=curcolor]
{
\newpath
\moveto(169.89554758,56.24823915)
\curveto(169.89554758,56.63223931)(169.90621425,56.87757274)(169.92754759,56.98423945)
\curveto(169.97021428,57.1122395)(170.07688099,57.26157289)(170.24754772,57.43223962)
\lineto(176.39155017,64.28024235)
\curveto(178.63155106,66.79757669)(179.7515515,69.15491096)(179.7515515,71.35224516)
\curveto(179.7515515,72.78157907)(179.37821802,74.00824622)(178.63155106,75.03224663)
\curveto(177.88488409,76.05624704)(176.82888367,76.56824724)(175.4635498,76.56824724)
\curveto(174.52488276,76.56824724)(173.66088241,76.28024713)(172.87154877,75.7042469)
\curveto(172.08221512,75.12824667)(171.50621489,74.32824635)(171.14354808,73.30424594)
\curveto(171.2075481,73.32557928)(171.34621483,73.33624595)(171.55954824,73.33624595)
\curveto(172.09288179,73.33624595)(172.50888196,73.16557922)(172.80754874,72.82424575)
\curveto(173.10621553,72.50424562)(173.25554892,72.12024547)(173.25554892,71.67224529)
\curveto(173.25554892,71.09624506)(173.06354884,70.66957823)(172.67954869,70.39224478)
\curveto(172.31688188,70.11491134)(171.95421507,69.97624462)(171.59154826,69.97624462)
\curveto(171.44221486,69.97624462)(171.27154813,69.98691129)(171.07954805,70.00824463)
\curveto(170.88754798,70.05091131)(170.64221455,70.22157805)(170.34354776,70.52024483)
\curveto(170.04488097,70.81891162)(169.89554758,71.23491178)(169.89554758,71.76824533)
\curveto(169.89554758,73.26157926)(170.46088114,74.59491312)(171.59154826,75.76824692)
\curveto(172.72221537,76.96291406)(174.15154928,77.56024764)(175.87954996,77.56024764)
\curveto(177.84221741,77.56024764)(179.46355139,76.97358074)(180.7435519,75.80024694)
\curveto(182.02355241,74.64824648)(182.66355266,73.16557922)(182.66355266,71.35224516)
\curveto(182.66355266,70.71224491)(182.56755263,70.10424467)(182.37555255,69.52824444)
\curveto(182.18355247,68.95224421)(181.95955238,68.44024401)(181.70355228,67.99224383)
\curveto(181.44755218,67.54424365)(180.97821866,66.95757675)(180.29555172,66.23224313)
\curveto(179.61288478,65.52824285)(178.99421787,64.9202426)(178.43955098,64.4082424)
\curveto(177.88488409,63.8962422)(176.98888374,63.09624188)(175.75154991,62.00824145)
\lineto(172.35954856,58.71224013)
\lineto(178.11955085,58.71224013)
\curveto(179.99688494,58.71224013)(181.01021867,58.7975735)(181.15955206,58.96824024)
\curveto(181.37288548,59.26690702)(181.60755224,60.21624073)(181.86355235,61.81624137)
\lineto(182.66355266,61.81624137)
\lineto(181.76755231,56.24823915)
\closepath
}
}
{
\newrgbcolor{curcolor}{0 0 0}
\pscustom[linestyle=none,fillstyle=solid,fillcolor=curcolor]
{
\newpath
\moveto(-0.00000175,566.01903974)
\lineto(-0.00000175,572.41904229)
\lineto(-0.00000175,572.89904248)
\curveto(0.0213316,572.98437585)(0.06399828,573.05904255)(0.1279983,573.12304257)
\curveto(0.19199833,573.1870426)(0.28799837,573.21904261)(0.41599842,573.21904261)
\curveto(0.65066518,573.21904261)(0.77866523,573.1123759)(0.79999857,572.89904248)
\curveto(0.8639986,570.78704164)(1.52533219,569.14437432)(2.78399936,567.97104052)
\curveto(4.02133319,566.81904006)(5.70666719,566.24303983)(7.84000137,566.24303983)
\curveto(9.12000188,566.24303983)(10.15466896,566.70170668)(10.94400261,567.61904038)
\curveto(11.73333626,568.53637408)(12.12800308,569.60304117)(12.12800308,570.81904166)
\curveto(12.12800308,571.97104211)(11.79733628,572.95237584)(11.13600269,573.76304283)
\curveto(10.85866924,574.12570964)(10.53866911,574.42437642)(10.1760023,574.65904318)
\curveto(9.83466883,574.89370994)(9.55733539,575.04304334)(9.34400197,575.10704336)
\curveto(9.13066855,575.19237673)(8.8213351,575.28837677)(8.4160016,575.39504348)
\curveto(5.77066722,576.03504373)(4.36266666,576.38704387)(4.19199992,576.4510439)
\curveto(2.93333275,576.87771073)(1.91999902,577.6243777)(1.15199871,578.69104479)
\curveto(0.38399841,579.77904522)(-0.00000175,580.98437903)(-0.00000175,582.30704623)
\curveto(-0.00000175,584.01371357)(0.59733182,585.47504749)(1.79199897,586.69104797)
\curveto(3.00799945,587.90704846)(4.49066671,588.5150487)(6.24000074,588.5150487)
\curveto(7.11466775,588.5150487)(7.91466807,588.37638198)(8.64000169,588.09904853)
\curveto(9.38666866,587.82171509)(9.90933553,587.55504832)(10.20800232,587.29904821)
\curveto(10.5066691,587.06438145)(10.89066925,586.72304798)(11.36000277,586.27504781)
\lineto(12.48000322,588.09904853)
\curveto(12.65066995,588.37638198)(12.81067002,588.5150487)(12.96000341,588.5150487)
\curveto(13.13067015,588.5150487)(13.22667018,588.46171534)(13.24800353,588.35504863)
\curveto(13.29067021,588.26971527)(13.31200355,588.06704852)(13.31200355,587.74704839)
\lineto(13.31200355,581.31504583)
\curveto(13.31200355,581.08037907)(13.30133688,580.92037901)(13.28000354,580.83504564)
\curveto(13.28000354,580.74971227)(13.24800353,580.67504558)(13.1840035,580.61104555)
\curveto(13.12000348,580.56837887)(13.02400344,580.54704553)(12.89600339,580.54704553)
\curveto(12.68266997,580.54704553)(12.55466992,580.67504558)(12.51200323,580.93104568)
\curveto(11.91466966,585.38971412)(9.83466883,587.61904834)(6.27200075,587.61904834)
\curveto(5.05600027,587.61904834)(4.04266653,587.20304818)(3.23199954,586.37104784)
\curveto(2.44266589,585.56038086)(2.04799907,584.57904713)(2.04799907,583.42704667)
\curveto(2.04799907,582.48837963)(2.33599918,581.63504596)(2.91199941,580.86704565)
\curveto(3.50933298,580.09904535)(4.29866663,579.59771182)(5.28000035,579.36304506)
\lineto(9.37600199,578.37104466)
\curveto(10.74133586,578.05104453)(11.88266965,577.28304423)(12.80000335,576.06704374)
\curveto(13.71733705,574.8723766)(14.1760039,573.48570938)(14.1760039,571.90704209)
\curveto(14.1760039,570.09370803)(13.57867032,568.52570741)(12.38400318,567.20304022)
\curveto(11.21066938,565.90170636)(9.70666878,565.25103944)(7.87200139,565.25103944)
\curveto(6.54933419,565.25103944)(5.37600039,565.47503953)(4.35199999,565.92303971)
\curveto(3.32799958,566.34970654)(2.53866593,566.87237342)(1.98399904,567.49104033)
\curveto(1.55733221,566.78704005)(1.20533207,566.23237316)(0.92799862,565.82703967)
\lineto(0.83199858,565.6670396)
\curveto(0.66133185,565.38970616)(0.50133179,565.25103944)(0.35199839,565.25103944)
\curveto(0.18133166,565.25103944)(0.07466495,565.30437279)(0.03199827,565.4110395)
\curveto(0.01066492,565.51770621)(-0.00000175,565.72037296)(-0.00000175,566.01903974)
\closepath
}
}
{
\newrgbcolor{curcolor}{0 0 0}
\pscustom[linestyle=none,fillstyle=solid,fillcolor=curcolor]
{
\newpath
\moveto(16.99195425,578.75504481)
\lineto(16.99195425,579.74704521)
\lineto(21.69595612,580.09904535)
\lineto(21.69595612,569.47504112)
\curveto(21.69595612,568.96304092)(21.71728946,568.55770742)(21.75995615,568.25904064)
\curveto(21.80262283,567.96037385)(21.90928954,567.64037372)(22.07995627,567.29904025)
\curveto(22.25062301,566.95770679)(22.53862312,566.70170668)(22.94395662,566.53103995)
\curveto(23.34929011,566.38170656)(23.87195699,566.30703986)(24.51195724,566.30703986)
\curveto(25.6639577,566.30703986)(26.59195807,566.77637338)(27.29595835,567.71504042)
\curveto(28.02129197,568.6750408)(28.38395878,569.85904127)(28.38395878,571.26704183)
\lineto(28.38395878,576.9630441)
\curveto(28.38395878,577.75237775)(28.24529206,578.24304461)(27.96795862,578.43504469)
\curveto(27.69062517,578.6483781)(26.99729156,578.75504481)(25.88795779,578.75504481)
\lineto(25.88795779,579.74704521)
\lineto(30.59195966,580.09904535)
\lineto(30.59195966,568.73904083)
\curveto(30.59195966,567.94970718)(30.73062638,567.44837365)(31.00795983,567.23504023)
\curveto(31.28529327,567.04304015)(31.97862688,566.94704011)(33.08796065,566.94704011)
\lineto(33.08796065,565.95503972)
\lineto(28.47995882,565.60303958)
\lineto(28.47995882,568.48304073)
\curveto(27.58395846,566.56303996)(26.20795792,565.60303958)(24.35195718,565.60303958)
\curveto(23.41329014,565.60303958)(22.62395649,565.72037296)(21.98395623,565.95503972)
\curveto(21.34395598,566.18970648)(20.86395579,566.45637325)(20.54395566,566.75504004)
\curveto(20.22395553,567.05370682)(19.9786221,567.49104033)(19.80795537,568.06704056)
\curveto(19.63728863,568.64304079)(19.5412886,569.10170764)(19.51995525,569.44304111)
\curveto(19.49862191,569.80570792)(19.48795524,570.32837479)(19.48795524,571.01104173)
\lineto(19.48795524,575.81104364)
\curveto(19.48795524,577.26171089)(19.38128853,578.11504456)(19.16795511,578.37104466)
\curveto(18.9546217,578.62704476)(18.22928807,578.75504481)(16.99195425,578.75504481)
\closepath
}
}
{
\newrgbcolor{curcolor}{0 0 0}
\pscustom[linestyle=none,fillstyle=solid,fillcolor=curcolor]
{
\newpath
\moveto(34.6239105,565.95503972)
\lineto(34.6239105,566.94704011)
\curveto(35.75457761,566.94704011)(36.44791122,567.01104014)(36.70391132,567.13904019)
\curveto(36.98124477,567.28837358)(37.11991149,567.70437375)(37.11991149,568.38704069)
\lineto(37.11991149,576.9630441)
\curveto(37.11991149,577.75237775)(36.98124477,578.24304461)(36.70391132,578.43504469)
\curveto(36.42657788,578.6483781)(35.73324427,578.75504481)(34.6239105,578.75504481)
\lineto(34.6239105,579.74704521)
\lineto(39.07191227,580.09904535)
\lineto(39.07191227,576.57904395)
\curveto(39.37057905,577.49637765)(39.83991257,578.30704464)(40.47991283,579.01104492)
\curveto(41.11991308,579.73637854)(41.96258008,580.09904535)(43.00791383,580.09904535)
\curveto(43.69058077,580.09904535)(44.25591433,579.90704527)(44.70391451,579.52304512)
\curveto(45.15191469,579.13904497)(45.37591478,578.68037812)(45.37591478,578.14704457)
\curveto(45.37591478,577.67771105)(45.22658138,577.32571091)(44.9279146,577.09104415)
\curveto(44.65058115,576.85637739)(44.3412477,576.73904401)(43.99991423,576.73904401)
\curveto(43.61591408,576.73904401)(43.28524728,576.85637739)(43.00791383,577.09104415)
\curveto(42.75191373,577.34704425)(42.62391368,577.68837772)(42.62391368,578.11504456)
\curveto(42.62391368,578.37104466)(42.67724703,578.59504475)(42.78391374,578.78704483)
\curveto(42.9119138,579.00037824)(43.0185805,579.13904497)(43.10391387,579.20304499)
\curveto(43.21058058,579.28837836)(43.29591395,579.34171171)(43.35991397,579.36304506)
\curveto(43.31724729,579.3843784)(43.19991391,579.39504507)(43.00791383,579.39504507)
\curveto(41.83458003,579.39504507)(40.90657966,578.80837817)(40.22391273,577.63504437)
\curveto(39.56257913,576.46171057)(39.23191233,575.04304334)(39.23191233,573.37904267)
\lineto(39.23191233,568.45104071)
\curveto(39.23191233,567.8323738)(39.35991238,567.42704031)(39.61591248,567.23504023)
\curveto(39.87191259,567.04304015)(40.55457952,566.94704011)(41.6639133,566.94704011)
\lineto(42.33591357,566.94704011)
\lineto(42.33591357,565.95503972)
\curveto(41.48257989,566.01903974)(40.12791269,566.05103976)(38.27191195,566.05103976)
\curveto(38.01591185,566.05103976)(37.68524505,566.04037309)(37.27991155,566.01903974)
\curveto(36.87457806,566.01903974)(36.40524454,566.00837307)(35.87191099,565.98703973)
\curveto(35.33857745,565.96570639)(34.92257728,565.95503972)(34.6239105,565.95503972)
\closepath
}
}
{
\newrgbcolor{curcolor}{0 0 0}
\pscustom[linestyle=none,fillstyle=solid,fillcolor=curcolor]
{
\newpath
\moveto(47.2958785,578.75504481)
\lineto(47.2958785,579.74704521)
\lineto(49.8238795,579.74704521)
\lineto(49.8238795,583.42704667)
\curveto(49.8238795,585.00571397)(50.33587971,586.24304779)(51.35988011,587.13904815)
\curveto(52.38388052,588.05638185)(53.52521431,588.5150487)(54.78388148,588.5150487)
\curveto(55.61588181,588.5150487)(56.29854875,588.29104861)(56.83188229,587.84304843)
\curveto(57.38654918,587.39504825)(57.66388262,586.87238138)(57.66388262,586.27504781)
\curveto(57.66388262,585.86971431)(57.53588257,585.52838084)(57.27988247,585.2510474)
\curveto(57.04521571,584.9950473)(56.70388224,584.86704725)(56.25588206,584.86704725)
\curveto(55.82921523,584.86704725)(55.48788176,584.9950473)(55.23188166,585.2510474)
\curveto(54.9972149,585.52838084)(54.87988152,585.85904764)(54.87988152,586.24304779)
\curveto(54.87988152,586.92571473)(55.19988164,587.37371491)(55.8398819,587.58704833)
\curveto(55.51988177,587.73638172)(55.16788163,587.81104842)(54.78388148,587.81104842)
\curveto(54.01588117,587.81104842)(53.33321423,587.41638159)(52.73588066,586.62704795)
\curveto(52.13854709,585.8377143)(51.83988031,584.78171388)(51.83988031,583.45904669)
\lineto(51.83988031,579.74704521)
\lineto(55.5838818,579.74704521)
\lineto(55.5838818,578.75504481)
\lineto(51.93588034,578.75504481)
\lineto(51.93588034,568.45104071)
\curveto(51.93588034,567.8323738)(52.06388039,567.42704031)(52.3198805,567.23504023)
\curveto(52.5758806,567.04304015)(53.25854754,566.94704011)(54.36788131,566.94704011)
\lineto(55.03988158,566.94704011)
\lineto(55.03988158,565.95503972)
\curveto(54.18654791,566.01903974)(52.8318807,566.05103976)(50.97587996,566.05103976)
\curveto(50.71987986,566.05103976)(50.38921306,566.04037309)(49.98387957,566.01903974)
\curveto(49.57854607,566.01903974)(49.10921255,566.00837307)(48.57587901,565.98703973)
\curveto(48.04254546,565.96570639)(47.6265453,565.95503972)(47.32787851,565.95503972)
\lineto(47.32787851,566.94704011)
\curveto(48.45854563,566.94704011)(49.15187924,567.01104014)(49.40787934,567.13904019)
\curveto(49.68521278,567.28837358)(49.8238795,567.70437375)(49.8238795,568.38704069)
\lineto(49.8238795,578.75504481)
\closepath
}
}
{
\newrgbcolor{curcolor}{0 0 0}
\pscustom[linestyle=none,fillstyle=solid,fillcolor=curcolor]
{
\newpath
\moveto(57.34383263,568.99504093)
\curveto(57.34383263,570.80837498)(58.41049972,572.1950422)(60.5438339,573.15504259)
\curveto(61.82383441,573.7737095)(63.78650186,574.15770965)(66.43183625,574.30704304)
\lineto(66.43183625,575.49104351)
\curveto(66.43183625,576.81371071)(66.07983611,577.82704444)(65.37583583,578.53104472)
\curveto(64.69316889,579.235045)(63.91450191,579.58704514)(63.0398349,579.58704514)
\curveto(61.48250094,579.58704514)(60.33050048,579.09637828)(59.58383352,578.11504456)
\curveto(60.22383377,578.09371122)(60.65050061,577.92304448)(60.86383403,577.60304436)
\curveto(61.09850079,577.28304423)(61.21583417,576.9630441)(61.21583417,576.64304397)
\curveto(61.21583417,576.21637714)(61.07716745,575.864377)(60.799834,575.58704355)
\curveto(60.5438339,575.30971011)(60.19183376,575.17104339)(59.74383358,575.17104339)
\curveto(59.31716675,575.17104339)(58.96516661,575.29904344)(58.68783316,575.55504354)
\curveto(58.41049972,575.83237698)(58.271833,576.20571047)(58.271833,576.67504399)
\curveto(58.271833,577.72037774)(58.74116652,578.58437808)(59.67983356,579.26704502)
\curveto(60.6185006,579.94971196)(61.75983439,580.29104543)(63.10383492,580.29104543)
\curveto(64.85316895,580.29104543)(66.31450287,579.70437852)(67.48783667,578.53104472)
\curveto(67.85050348,578.16837791)(68.11717025,577.75237775)(68.28783698,577.28304423)
\curveto(68.47983706,576.81371071)(68.58650377,576.41904388)(68.60783711,576.09904376)
\curveto(68.62917045,575.80037697)(68.63983712,575.35237679)(68.63983712,574.75504322)
\lineto(68.63983712,568.35504067)
\curveto(68.63983712,568.22704062)(68.66117047,568.05637389)(68.70383715,567.84304047)
\curveto(68.74650383,567.65104039)(68.86383721,567.41637363)(69.05583729,567.13904019)
\curveto(69.24783737,566.88304009)(69.50383747,566.75504004)(69.8238376,566.75504004)
\curveto(70.5918379,566.75504004)(70.97583805,567.43770698)(70.97583805,568.80304085)
\lineto(70.97583805,570.59504157)
\lineto(71.77583837,570.59504157)
\lineto(71.77583837,568.80304085)
\curveto(71.77583837,567.67237374)(71.47717159,566.88304009)(70.87983802,566.43503991)
\curveto(70.28250444,565.98703973)(69.70650422,565.76303964)(69.15183733,565.76303964)
\curveto(68.44783705,565.76303964)(67.87183682,566.01903974)(67.42383664,566.53103995)
\curveto(66.97583646,567.06437349)(66.71983636,567.68304041)(66.65583633,568.38704069)
\curveto(66.33583621,567.5763737)(65.80250266,566.90437343)(65.0558357,566.37103989)
\curveto(64.33050208,565.85903968)(63.46650173,565.60303958)(62.46383467,565.60303958)
\curveto(61.69583436,565.60303958)(60.9491674,565.69903962)(60.22383377,565.89103969)
\curveto(59.49850015,566.08303977)(58.83716656,566.44570658)(58.23983299,566.97904013)
\curveto(57.64249941,567.51237367)(57.34383263,568.18437394)(57.34383263,568.99504093)
\closepath
\moveto(59.80783361,569.02704094)
\curveto(59.80783361,568.23770729)(60.08516705,567.58704037)(60.63983394,567.07504017)
\curveto(61.21583417,566.56303996)(61.89850111,566.30703986)(62.68783476,566.30703986)
\curveto(63.58383511,566.30703986)(64.42650211,566.64837333)(65.21583576,567.33104027)
\curveto(66.02650275,568.03504055)(66.43183625,569.06970763)(66.43183625,570.4350415)
\lineto(66.43183625,573.63504278)
\curveto(64.0638353,573.54970941)(62.36783463,573.0377092)(61.34383422,572.09904216)
\curveto(60.31983381,571.16037512)(59.80783361,570.13637472)(59.80783361,569.02704094)
\closepath
}
}
{
\newrgbcolor{curcolor}{0 0 0}
\pscustom[linestyle=none,fillstyle=solid,fillcolor=curcolor]
{
\newpath
\moveto(75.10377662,567.71504042)
\curveto(73.75977608,569.12304098)(73.08777582,570.840375)(73.08777582,572.86704247)
\curveto(73.08777582,574.89370994)(73.74910941,576.6323773)(75.07177661,578.08304455)
\curveto(76.41577714,579.55504513)(78.06911113,580.29104543)(80.03177858,580.29104543)
\curveto(81.33311243,580.29104543)(82.47444622,579.98171197)(83.45577994,579.36304506)
\curveto(84.43711367,578.74437814)(84.92778053,577.91237781)(84.92778053,576.86704406)
\curveto(84.92778053,576.39771054)(84.78911381,576.02437706)(84.51178036,575.74704362)
\curveto(84.23444692,575.49104351)(83.88244678,575.36304346)(83.45577994,575.36304346)
\curveto(83.00777976,575.36304346)(82.64511295,575.50171019)(82.36777951,575.77904363)
\curveto(82.11177941,576.05637707)(81.98377936,576.40837721)(81.98377936,576.83504405)
\curveto(81.98377936,577.02704413)(82.01577937,577.20837753)(82.0797794,577.37904427)
\curveto(82.14377942,577.57104434)(82.29311281,577.76304442)(82.52777957,577.9550445)
\curveto(82.76244633,578.16837791)(83.08244646,578.28571129)(83.48777996,578.30704464)
\curveto(82.71977965,579.09637828)(81.57844586,579.49104511)(80.06377859,579.49104511)
\curveto(78.97577816,579.49104511)(77.98377777,578.9790449)(77.08777741,577.9550445)
\curveto(76.19177705,576.93104409)(75.74377687,575.25637675)(75.74377687,572.9310425)
\curveto(75.74377687,571.71504201)(75.89311027,570.65904159)(76.19177705,569.76304124)
\curveto(76.49044384,568.86704088)(76.88511066,568.18437394)(77.37577752,567.71504042)
\curveto(77.86644439,567.26704024)(78.34644458,566.93637344)(78.8157781,566.72304003)
\curveto(79.30644496,566.50970661)(79.78644515,566.4030399)(80.25577867,566.4030399)
\curveto(82.34644617,566.4030399)(83.74378006,567.52304034)(84.44778034,569.76304124)
\curveto(84.51178036,569.97637465)(84.65044709,570.08304136)(84.8637805,570.08304136)
\curveto(85.14111395,570.08304136)(85.27978067,569.97637465)(85.27978067,569.76304124)
\curveto(85.27978067,569.65637453)(85.23711399,569.46437445)(85.15178062,569.18704101)
\curveto(85.06644725,568.9310409)(84.89578052,568.58970743)(84.63978041,568.1630406)
\curveto(84.38378031,567.73637376)(84.06378019,567.33104027)(83.67978003,566.94704011)
\curveto(83.31711322,566.5843733)(82.80511302,566.26437318)(82.14377942,565.98703973)
\curveto(81.50377917,565.73103963)(80.77844554,565.60303958)(79.96777856,565.60303958)
\curveto(78.09044447,565.60303958)(76.4691105,566.30703986)(75.10377662,567.71504042)
\closepath
}
}
{
\newrgbcolor{curcolor}{0 0 0}
\pscustom[linestyle=none,fillstyle=solid,fillcolor=curcolor]
{
\newpath
\moveto(87.10372925,572.99504252)
\curveto(87.10372925,575.00037665)(87.7437295,576.71771067)(89.02373001,578.14704457)
\curveto(90.32506386,579.57637847)(91.90373116,580.29104543)(93.7597319,580.29104543)
\curveto(95.63706598,580.29104543)(97.05573321,579.68304518)(98.01573359,578.4670447)
\curveto(98.99706731,577.25104422)(99.48773417,575.76837696)(99.48773417,574.01904293)
\curveto(99.48773417,573.6990428)(99.44506749,573.50704273)(99.35973412,573.4430427)
\curveto(99.27440076,573.37904267)(99.05040067,573.34704266)(98.68773386,573.34704266)
\lineto(89.7597303,573.34704266)
\curveto(89.7597303,571.1710418)(90.07973043,569.57104116)(90.71973069,568.54704075)
\curveto(91.61573104,567.11770685)(92.82106485,566.4030399)(94.33573212,566.4030399)
\curveto(94.54906554,566.4030399)(94.77306563,566.42437324)(95.00773239,566.46703992)
\curveto(95.26373249,566.50970661)(95.64773265,566.61637332)(96.15973285,566.78704005)
\curveto(96.67173305,566.97904013)(97.15173325,567.3203736)(97.59973342,567.81104046)
\curveto(98.0477336,568.30170732)(98.39973374,568.9310409)(98.65573384,569.69904121)
\curveto(98.71973387,569.997708)(98.85840059,570.14704139)(99.07173401,570.14704139)
\curveto(99.34906745,570.14704139)(99.48773417,570.01904134)(99.48773417,569.76304124)
\curveto(99.48773417,569.57104116)(99.38106747,569.2617077)(99.16773405,568.83504087)
\curveto(98.97573397,568.42970737)(98.68773386,567.97104052)(98.3037337,567.45904032)
\curveto(97.91973355,566.96837346)(97.35439999,566.53103995)(96.60773303,566.1470398)
\curveto(95.86106606,565.78437298)(95.0397324,565.60303958)(94.14373205,565.60303958)
\curveto(92.26639797,565.60303958)(90.62373065,566.30703986)(89.21573009,567.71504042)
\curveto(87.80772953,569.14437432)(87.10372925,570.90437502)(87.10372925,572.99504252)
\closepath
\moveto(89.79173032,574.01904293)
\lineto(97.37573333,574.01904293)
\curveto(97.37573333,574.46704311)(97.33306665,574.93637663)(97.24773328,575.42704349)
\curveto(97.18373326,575.93904369)(97.03439987,576.53637726)(96.7997331,577.2190442)
\curveto(96.58639969,577.92304448)(96.2130662,578.48837804)(95.67973266,578.91504488)
\curveto(95.16773246,579.36304506)(94.5277322,579.58704514)(93.7597319,579.58704514)
\curveto(93.41839843,579.58704514)(93.05573161,579.51237845)(92.67173146,579.36304506)
\curveto(92.30906465,579.21371166)(91.90373116,578.95771156)(91.45573098,578.59504475)
\curveto(91.0077308,578.25371128)(90.62373065,577.67771105)(90.30373052,576.86704406)
\curveto(90.00506373,576.07771041)(89.834397,575.1283767)(89.79173032,574.01904293)
\closepath
}
}
{
\newrgbcolor{curcolor}{0 0 0}
\pscustom[linestyle=none,fillstyle=solid,fillcolor=curcolor]
{
\newpath
\moveto(114.0796289,567.68304041)
\curveto(112.67162834,569.06970763)(111.96762806,570.77637497)(111.96762806,572.80304244)
\curveto(111.96762806,574.82970992)(112.65029499,576.57904395)(114.01562887,578.05104453)
\curveto(115.40229609,579.54437846)(117.08763009,580.29104543)(119.07163088,580.29104543)
\curveto(121.01296499,580.29104543)(122.67696565,579.55504513)(124.06363287,578.08304455)
\curveto(125.45030009,576.61104396)(126.1436337,574.85104326)(126.1436337,572.80304244)
\curveto(126.1436337,570.79770831)(125.43963342,569.09104097)(124.03163286,567.68304041)
\curveto(122.64496564,566.29637319)(120.98096498,565.60303958)(119.03963087,565.60303958)
\curveto(117.14096345,565.60303958)(115.48762946,566.29637319)(114.0796289,567.68304041)
\closepath
\moveto(114.62362911,573.05904255)
\curveto(114.62362911,570.98970839)(114.89029589,569.48570779)(115.42362943,568.54704075)
\curveto(116.25562976,567.11770685)(117.47163025,566.4030399)(119.07163088,566.4030399)
\curveto(119.86096453,566.4030399)(120.58629815,566.61637332)(121.24763175,567.04304015)
\curveto(121.90896535,567.46970699)(122.42096555,568.04570722)(122.78363236,568.77104084)
\curveto(123.25296588,569.70970788)(123.48763264,571.13904178)(123.48763264,573.05904255)
\curveto(123.48763264,575.10704336)(123.2102992,576.58971062)(122.65563231,577.50704432)
\curveto(121.82363198,578.89371154)(120.61829817,579.58704514)(119.03963087,579.58704514)
\curveto(118.35696393,579.58704514)(117.67429699,579.40571174)(116.99163006,579.04304493)
\curveto(116.33029646,578.68037812)(115.79696291,578.14704457)(115.39162942,577.44304429)
\curveto(114.87962922,576.50437725)(114.62362911,575.04304334)(114.62362911,573.05904255)
\closepath
}
}
{
\newrgbcolor{curcolor}{0 0 0}
\pscustom[linestyle=none,fillstyle=solid,fillcolor=curcolor]
{
\newpath
\moveto(126.55952218,565.95503972)
\lineto(126.55952218,566.94704011)
\curveto(128.22352285,566.94704011)(129.58885672,567.58704037)(130.65552381,568.86704088)
\lineto(133.66352501,572.67504239)
\lineto(130.49552375,576.80304404)
\curveto(129.74885679,577.76304442)(129.20485657,578.32837798)(128.8635231,578.49904471)
\curveto(128.54352297,578.66971145)(127.82885602,578.75504481)(126.71952225,578.75504481)
\lineto(126.71952225,579.74704521)
\curveto(127.59418926,579.68304518)(128.607523,579.65104517)(129.75952346,579.65104517)
\curveto(130.39952371,579.65104517)(131.55152417,579.68304518)(133.21552483,579.74704521)
\lineto(133.21552483,578.75504481)
\curveto(132.46885787,578.71237813)(132.09552439,578.44571136)(132.09552439,577.9550445)
\curveto(132.09552439,577.78437776)(132.19152443,577.59237768)(132.3835245,577.37904427)
\lineto(134.8795255,574.14704298)
\lineto(136.86352628,576.64304397)
\curveto(137.18352641,577.02704413)(137.34352648,577.40037761)(137.34352648,577.76304442)
\curveto(137.34352648,578.36037799)(137.03419302,578.69104479)(136.41552611,578.75504481)
\lineto(136.41552611,579.74704521)
\curveto(137.24752644,579.68304518)(138.27152685,579.65104517)(139.48752733,579.65104517)
\curveto(140.40486103,579.65104517)(141.27952804,579.68304518)(142.11152837,579.74704521)
\lineto(142.11152837,578.75504481)
\curveto(140.4901944,578.71237813)(139.22086056,578.15771124)(138.30352686,577.09104415)
\curveto(137.89819336,576.64304397)(136.91685964,575.43771016)(135.35952569,573.47504271)
\lineto(139.64752739,567.93904051)
\curveto(139.98886086,567.49104033)(140.35152767,567.21370689)(140.73552783,567.10704018)
\curveto(141.11952798,567.00037347)(141.7701949,566.94704011)(142.6875286,566.94704011)
\lineto(142.6875286,565.95503972)
\curveto(141.72752822,566.01903974)(140.70352781,566.05103976)(139.61552738,566.05103976)
\curveto(139.23152723,566.05103976)(138.07952677,566.01903974)(136.159526,565.95503972)
\lineto(136.159526,566.94704011)
\curveto(136.54352616,566.96837346)(136.83152627,567.06437349)(137.02352635,567.23504023)
\curveto(137.21552643,567.40570696)(137.31152646,567.5763737)(137.31152646,567.74704043)
\curveto(137.31152646,567.8323738)(137.08752637,568.18437394)(136.6395262,568.80304085)
\lineto(134.20752523,572.00304213)
\curveto(132.58619125,570.08304136)(131.66885755,568.89904089)(131.45552413,568.45104071)
\curveto(131.37019077,568.25904064)(131.32752408,568.0883739)(131.32752408,567.93904051)
\curveto(131.32752408,567.36304028)(131.63685754,567.03237348)(132.25552445,566.94704011)
\lineto(132.25552445,565.95503972)
\curveto(130.71952384,566.01903974)(129.7061901,566.05103976)(129.21552324,566.05103976)
\curveto(128.2768562,566.05103976)(127.39152252,566.01903974)(126.55952218,565.95503972)
\closepath
}
}
{
\newrgbcolor{curcolor}{0 0 0}
\pscustom[linestyle=none,fillstyle=solid,fillcolor=curcolor]
{
\newpath
\moveto(144.09547845,565.95503972)
\lineto(144.09547845,566.94704011)
\curveto(145.22614557,566.94704011)(145.91947918,567.01104014)(146.17547928,567.13904019)
\curveto(146.45281272,567.28837358)(146.59147945,567.70437375)(146.59147945,568.38704069)
\lineto(146.59147945,576.99504411)
\curveto(146.59147945,577.78437776)(146.45281272,578.27504462)(146.17547928,578.4670447)
\curveto(145.91947918,578.65904478)(145.26881225,578.75504481)(144.2234785,578.75504481)
\lineto(144.2234785,579.74704521)
\lineto(148.70348029,580.09904535)
\lineto(148.70348029,568.35504067)
\curveto(148.70348029,567.71504042)(148.810147,567.3203736)(149.02348041,567.1710402)
\curveto(149.25814717,567.02170681)(149.89814743,566.94704011)(150.94348118,566.94704011)
\lineto(150.94348118,565.95503972)
\curveto(148.76748031,566.01903974)(147.65814654,566.05103976)(147.61547985,566.05103976)
\curveto(147.31681307,566.05103976)(146.14347927,566.01903974)(144.09547845,565.95503972)
\closepath
\moveto(145.43947899,585.66704756)
\curveto(145.43947899,586.0937144)(145.59947905,586.47771455)(145.91947918,586.81904802)
\curveto(146.26081265,587.18171483)(146.66614614,587.36304824)(147.13547966,587.36304824)
\curveto(147.60481318,587.36304824)(147.99948001,587.20304818)(148.31948013,586.88304805)
\curveto(148.6608136,586.56304792)(148.83148034,586.15771443)(148.83148034,585.66704756)
\curveto(148.83148034,585.1763807)(148.6608136,584.77104721)(148.31948013,584.45104708)
\curveto(147.99948001,584.13104695)(147.60481318,583.97104689)(147.13547966,583.97104689)
\curveto(146.6448128,583.97104689)(146.23947931,584.14171362)(145.91947918,584.48304709)
\curveto(145.59947905,584.82438056)(145.43947899,585.21904739)(145.43947899,585.66704756)
\closepath
}
}
{
\newrgbcolor{curcolor}{0 0 0}
\pscustom[linestyle=none,fillstyle=solid,fillcolor=curcolor]
{
\newpath
\moveto(152.99144119,572.83504246)
\curveto(152.99144119,574.86170993)(153.69544147,576.57904395)(155.10344203,577.98704451)
\curveto(156.53277594,579.39504507)(158.20744327,580.09904535)(160.12744403,580.09904535)
\curveto(161.74877801,580.09904535)(163.09277855,579.43771175)(164.15944564,578.11504456)
\lineto(164.15944564,585.02704731)
\curveto(164.15944564,585.81638096)(164.02077892,586.30704782)(163.74344547,586.4990479)
\curveto(163.46611203,586.71238131)(162.77277842,586.81904802)(161.66344465,586.81904802)
\lineto(161.66344465,587.81104842)
\lineto(166.27144648,588.16304856)
\lineto(166.27144648,568.73904083)
\curveto(166.27144648,567.94970718)(166.4101132,567.44837365)(166.68744665,567.23504023)
\curveto(166.96478009,567.04304015)(167.6581137,566.94704011)(168.76744747,566.94704011)
\lineto(168.76744747,565.95503972)
\lineto(164.0634456,565.60303958)
\lineto(164.0634456,567.71504042)
\curveto(162.95411183,566.30703986)(161.52477792,565.60303958)(159.77544389,565.60303958)
\curveto(157.9407765,565.60303958)(156.35144253,566.30703986)(155.007442,567.71504042)
\curveto(153.66344146,569.12304098)(152.99144119,570.82970833)(152.99144119,572.83504246)
\closepath
\moveto(155.64744225,572.80304244)
\curveto(155.64744225,570.81904166)(155.95677571,569.3363744)(156.57544262,568.35504067)
\curveto(157.42877629,566.9897068)(158.54877674,566.30703986)(159.93544396,566.30703986)
\curveto(161.51411125,566.30703986)(162.77277842,567.05370682)(163.71144546,568.54704075)
\curveto(163.94611222,568.90970756)(164.0634456,569.30437439)(164.0634456,569.73104122)
\lineto(164.0634456,576.29104383)
\curveto(164.0634456,576.71771067)(163.94611222,577.11237749)(163.71144546,577.4750443)
\curveto(162.83677845,578.75504481)(161.68477799,579.39504507)(160.25544409,579.39504507)
\curveto(158.76211016,579.39504507)(157.55677634,578.71237813)(156.63944265,577.34704425)
\curveto(155.97810905,576.32304385)(155.64744225,574.80837658)(155.64744225,572.80304244)
\closepath
}
}
{
\newrgbcolor{curcolor}{0 0 0}
\pscustom[linestyle=none,fillstyle=solid,fillcolor=curcolor]
{
\newpath
\moveto(170.55938841,572.99504252)
\curveto(170.55938841,575.00037665)(171.19938866,576.71771067)(172.47938917,578.14704457)
\curveto(173.78072302,579.57637847)(175.35939032,580.29104543)(177.21539106,580.29104543)
\curveto(179.09272514,580.29104543)(180.51139237,579.68304518)(181.47139275,578.4670447)
\curveto(182.45272647,577.25104422)(182.94339334,575.76837696)(182.94339334,574.01904293)
\curveto(182.94339334,573.6990428)(182.90072665,573.50704273)(182.81539328,573.4430427)
\curveto(182.73005992,573.37904267)(182.50605983,573.34704266)(182.14339302,573.34704266)
\lineto(173.21538946,573.34704266)
\curveto(173.21538946,571.1710418)(173.53538959,569.57104116)(174.17538985,568.54704075)
\curveto(175.0713902,567.11770685)(176.27672402,566.4030399)(177.79139128,566.4030399)
\curveto(178.0047247,566.4030399)(178.22872479,566.42437324)(178.46339155,566.46703992)
\curveto(178.71939165,566.50970661)(179.10339181,566.61637332)(179.61539201,566.78704005)
\curveto(180.12739221,566.97904013)(180.60739241,567.3203736)(181.05539258,567.81104046)
\curveto(181.50339276,568.30170732)(181.8553929,568.9310409)(182.111393,569.69904121)
\curveto(182.17539303,569.997708)(182.31405975,570.14704139)(182.52739317,570.14704139)
\curveto(182.80472661,570.14704139)(182.94339334,570.01904134)(182.94339334,569.76304124)
\curveto(182.94339334,569.57104116)(182.83672663,569.2617077)(182.62339321,568.83504087)
\curveto(182.43139313,568.42970737)(182.14339302,567.97104052)(181.75939286,567.45904032)
\curveto(181.37539271,566.96837346)(180.81005915,566.53103995)(180.06339219,566.1470398)
\curveto(179.31672523,565.78437298)(178.49539156,565.60303958)(177.59939121,565.60303958)
\curveto(175.72205713,565.60303958)(174.07938981,566.30703986)(172.67138925,567.71504042)
\curveto(171.26338869,569.14437432)(170.55938841,570.90437502)(170.55938841,572.99504252)
\closepath
\moveto(173.24738948,574.01904293)
\lineto(180.83139249,574.01904293)
\curveto(180.83139249,574.46704311)(180.78872581,574.93637663)(180.70339244,575.42704349)
\curveto(180.63939242,575.93904369)(180.49005903,576.53637726)(180.25539227,577.2190442)
\curveto(180.04205885,577.92304448)(179.66872537,578.48837804)(179.13539182,578.91504488)
\curveto(178.62339162,579.36304506)(177.98339136,579.58704514)(177.21539106,579.58704514)
\curveto(176.87405759,579.58704514)(176.51139078,579.51237845)(176.12739062,579.36304506)
\curveto(175.76472381,579.21371166)(175.35939032,578.95771156)(174.91139014,578.59504475)
\curveto(174.46338996,578.25371128)(174.07938981,577.67771105)(173.75938968,576.86704406)
\curveto(173.46072289,576.07771041)(173.29005616,575.1283767)(173.24738948,574.01904293)
\closepath
}
}
{
\newrgbcolor{curcolor}{0 0 0}
\pscustom[linestyle=none,fillstyle=solid,fillcolor=curcolor]
{
\newpath
\moveto(1069.83123266,486.05511575)
\lineto(1069.83123266,487.04711614)
\lineto(1070.59923296,487.04711614)
\curveto(1071.75123342,487.04711614)(1072.4552337,487.14311618)(1072.7112338,487.33511626)
\curveto(1072.98856725,487.52711633)(1073.12723397,487.93244983)(1073.12723397,488.55111674)
\lineto(1073.12723397,505.41512345)
\curveto(1073.12723397,506.03379037)(1072.98856725,506.43912386)(1072.7112338,506.63112394)
\curveto(1072.4552337,506.82312401)(1071.75123342,506.91912405)(1070.59923296,506.91912405)
\lineto(1069.83123266,506.91912405)
\lineto(1069.83123266,507.91112445)
\lineto(1081.54323732,507.91112445)
\curveto(1083.56990479,507.91112445)(1085.2552388,507.36712423)(1086.59923933,506.2791238)
\curveto(1087.96457321,505.19112337)(1088.64724015,503.9324562)(1088.64724015,502.5031223)
\curveto(1088.64724015,501.32978849)(1088.14590661,500.27378807)(1087.14323955,499.33512103)
\curveto(1086.16190582,498.39645399)(1084.89257199,497.77778708)(1083.33523803,497.4791203)
\curveto(1085.1059054,497.28712022)(1086.57790599,496.65778664)(1087.75123979,495.59111954)
\curveto(1088.92457359,494.52445245)(1089.51124049,493.2977853)(1089.51124049,491.91111808)
\curveto(1089.51124049,490.35378413)(1088.83924022,488.98845025)(1087.49523969,487.81511645)
\curveto(1086.15123915,486.64178265)(1084.44457181,486.05511575)(1082.37523765,486.05511575)
\closepath
\moveto(1075.78323503,488.32711665)
\curveto(1075.78323503,487.77244977)(1075.85790172,487.42044963)(1076.00723512,487.27111623)
\curveto(1076.17790185,487.12178284)(1076.62590203,487.04711614)(1077.35123565,487.04711614)
\lineto(1081.35123724,487.04711614)
\curveto(1082.88723785,487.04711614)(1084.09257167,487.54844968)(1084.96723868,488.55111674)
\curveto(1085.86323904,489.55378381)(1086.31123922,490.68445092)(1086.31123922,491.94311809)
\curveto(1086.31123922,493.20178526)(1085.91657239,494.36445239)(1085.12723875,495.43111948)
\curveto(1084.35923844,496.51911991)(1083.24990467,497.06312013)(1081.79923742,497.06312013)
\lineto(1075.78323503,497.06312013)
\closepath
\moveto(1075.78323503,497.76712041)
\lineto(1080.42323687,497.76712041)
\curveto(1082.06590419,497.76712041)(1083.32457136,498.25778727)(1084.19923838,499.239121)
\curveto(1085.09523873,500.24178806)(1085.54323891,501.32978849)(1085.54323891,502.5031223)
\curveto(1085.54323891,503.54845604)(1085.19123877,504.54045644)(1084.48723849,505.47912348)
\curveto(1083.78323821,506.43912386)(1082.72723779,506.91912405)(1081.31923723,506.91912405)
\lineto(1077.35123565,506.91912405)
\curveto(1076.62590203,506.91912405)(1076.17790185,506.84445736)(1076.00723512,506.69512396)
\curveto(1075.85790172,506.54579057)(1075.78323503,506.19379043)(1075.78323503,505.63912354)
\closepath
}
}
{
\newrgbcolor{curcolor}{0 0 0}
\pscustom[linestyle=none,fillstyle=solid,fillcolor=curcolor]
{
\newpath
\moveto(1092.39118585,486.05511575)
\lineto(1092.39118585,487.04711614)
\curveto(1093.52185297,487.04711614)(1094.21518658,487.11111617)(1094.47118668,487.23911622)
\curveto(1094.74852013,487.38844961)(1094.88718685,487.80444978)(1094.88718685,488.48711672)
\lineto(1094.88718685,497.09512014)
\curveto(1094.88718685,497.88445379)(1094.74852013,498.37512065)(1094.47118668,498.56712073)
\curveto(1094.21518658,498.75912081)(1093.56451965,498.85512084)(1092.51918591,498.85512084)
\lineto(1092.51918591,499.84712124)
\lineto(1096.99918769,500.19912138)
\lineto(1096.99918769,488.4551167)
\curveto(1096.99918769,487.81511645)(1097.1058544,487.42044963)(1097.31918782,487.27111623)
\curveto(1097.55385458,487.12178284)(1098.19385483,487.04711614)(1099.23918858,487.04711614)
\lineto(1099.23918858,486.05511575)
\curveto(1097.06318771,486.11911577)(1095.95385394,486.15111579)(1095.91118726,486.15111579)
\curveto(1095.61252047,486.15111579)(1094.43918667,486.11911577)(1092.39118585,486.05511575)
\closepath
\moveto(1093.73518639,505.76712359)
\curveto(1093.73518639,506.19379043)(1093.89518645,506.57779058)(1094.21518658,506.91912405)
\curveto(1094.55652005,507.28179086)(1094.96185354,507.46312427)(1095.43118706,507.46312427)
\curveto(1095.90052058,507.46312427)(1096.29518741,507.30312421)(1096.61518754,506.98312408)
\curveto(1096.956521,506.66312395)(1097.12718774,506.25779046)(1097.12718774,505.76712359)
\curveto(1097.12718774,505.27645673)(1096.956521,504.87112324)(1096.61518754,504.55112311)
\curveto(1096.29518741,504.23112298)(1095.90052058,504.07112292)(1095.43118706,504.07112292)
\curveto(1094.9405202,504.07112292)(1094.53518671,504.24178965)(1094.21518658,504.58312312)
\curveto(1093.89518645,504.92445659)(1093.73518639,505.31912342)(1093.73518639,505.76712359)
\closepath
}
}
{
\newrgbcolor{curcolor}{0 0 0}
\pscustom[linestyle=none,fillstyle=solid,fillcolor=curcolor]
{
\newpath
\moveto(1101.25514137,486.05511575)
\lineto(1101.25514137,487.04711614)
\curveto(1102.38580849,487.04711614)(1103.0791421,487.11111617)(1103.3351422,487.23911622)
\curveto(1103.61247565,487.38844961)(1103.75114237,487.80444978)(1103.75114237,488.48711672)
\lineto(1103.75114237,505.12712334)
\curveto(1103.75114237,505.91645699)(1103.61247565,506.40712385)(1103.3351422,506.59912393)
\curveto(1103.05780876,506.81245734)(1102.36447515,506.91912405)(1101.25514137,506.91912405)
\lineto(1101.25514137,507.91112445)
\lineto(1105.86314321,508.26312459)
\lineto(1105.86314321,488.48711672)
\curveto(1105.86314321,487.80444978)(1105.99114326,487.38844961)(1106.24714336,487.23911622)
\curveto(1106.52447681,487.11111617)(1107.22847709,487.04711614)(1108.3591442,487.04711614)
\lineto(1108.3591442,486.05511575)
\curveto(1108.1031441,486.05511575)(1107.69781061,486.06578242)(1107.14314372,486.08711576)
\curveto(1106.60981017,486.1084491)(1106.14047665,486.11911577)(1105.73514316,486.11911577)
\curveto(1105.35114301,486.14044912)(1105.04180955,486.15111579)(1104.80714279,486.15111579)
\curveto(1104.55114269,486.15111579)(1103.36714222,486.11911577)(1101.25514137,486.05511575)
\closepath
}
}
{
\newrgbcolor{curcolor}{0 0 0}
\pscustom[linestyle=none,fillstyle=solid,fillcolor=curcolor]
{
\newpath
\moveto(1110.40710061,489.09511696)
\curveto(1110.40710061,490.90845101)(1111.47376771,492.29511823)(1113.60710189,493.25511861)
\curveto(1114.8871024,493.87378553)(1116.84976985,494.25778568)(1119.49510423,494.40711907)
\lineto(1119.49510423,495.59111954)
\curveto(1119.49510423,496.91378674)(1119.14310409,497.92712047)(1118.43910381,498.63112075)
\curveto(1117.75643687,499.33512103)(1116.9777699,499.68712117)(1116.10310288,499.68712117)
\curveto(1114.54576893,499.68712117)(1113.39376847,499.19645431)(1112.64710151,498.21512059)
\curveto(1113.28710176,498.19378725)(1113.7137686,498.02312051)(1113.92710202,497.70312038)
\curveto(1114.16176878,497.38312026)(1114.27910216,497.06312013)(1114.27910216,496.74312)
\curveto(1114.27910216,496.31645317)(1114.14043543,495.96445303)(1113.86310199,495.68711958)
\curveto(1113.60710189,495.40978614)(1113.25510175,495.27111942)(1112.80710157,495.27111942)
\curveto(1112.38043473,495.27111942)(1112.02843459,495.39911947)(1111.75110115,495.65511957)
\curveto(1111.47376771,495.93245301)(1111.33510098,496.3057865)(1111.33510098,496.77512002)
\curveto(1111.33510098,497.82045376)(1111.8044345,498.68445411)(1112.74310154,499.36712105)
\curveto(1113.68176858,500.04978799)(1114.82310237,500.39112145)(1116.16710291,500.39112145)
\curveto(1117.91643694,500.39112145)(1119.37777085,499.80445455)(1120.55110465,498.63112075)
\curveto(1120.91377146,498.26845394)(1121.18043824,497.85245378)(1121.35110497,497.38312026)
\curveto(1121.54310505,496.91378674)(1121.64977176,496.51911991)(1121.6711051,496.19911979)
\curveto(1121.69243844,495.900453)(1121.70310511,495.45245282)(1121.70310511,494.85511925)
\lineto(1121.70310511,488.4551167)
\curveto(1121.70310511,488.32711665)(1121.72443845,488.15644992)(1121.76710514,487.9431165)
\curveto(1121.80977182,487.75111642)(1121.9271052,487.51644966)(1122.11910528,487.23911622)
\curveto(1122.31110535,486.98311612)(1122.56710545,486.85511607)(1122.88710558,486.85511607)
\curveto(1123.65510589,486.85511607)(1124.03910604,487.53778301)(1124.03910604,488.90311688)
\lineto(1124.03910604,490.6951176)
\lineto(1124.83910636,490.6951176)
\lineto(1124.83910636,488.90311688)
\curveto(1124.83910636,487.77244977)(1124.54043957,486.98311612)(1123.943106,486.53511594)
\curveto(1123.34577243,486.08711576)(1122.7697722,485.86311567)(1122.21510531,485.86311567)
\curveto(1121.51110503,485.86311567)(1120.9351048,486.11911577)(1120.48710463,486.63111598)
\curveto(1120.03910445,487.16444952)(1119.78310435,487.78311644)(1119.71910432,488.48711672)
\curveto(1119.39910419,487.67644973)(1118.86577065,487.00444946)(1118.11910368,486.47111591)
\curveto(1117.39377006,485.95911571)(1116.52976972,485.70311561)(1115.52710265,485.70311561)
\curveto(1114.75910235,485.70311561)(1114.01243538,485.79911565)(1113.28710176,485.99111572)
\curveto(1112.56176814,486.1831158)(1111.90043454,486.54578261)(1111.30310097,487.07911616)
\curveto(1110.7057674,487.6124497)(1110.40710061,488.28444997)(1110.40710061,489.09511696)
\closepath
\moveto(1112.87110159,489.12711697)
\curveto(1112.87110159,488.33778332)(1113.14843504,487.6871164)(1113.70310193,487.17511619)
\curveto(1114.27910216,486.66311599)(1114.96176909,486.40711589)(1115.75110274,486.40711589)
\curveto(1116.6471031,486.40711589)(1117.4897701,486.74844936)(1118.27910375,487.4311163)
\curveto(1119.08977074,488.13511658)(1119.49510423,489.16978366)(1119.49510423,490.53511753)
\lineto(1119.49510423,493.73511881)
\curveto(1117.12710329,493.64978544)(1115.43110261,493.13778523)(1114.40710221,492.19911819)
\curveto(1113.3831018,491.26045115)(1112.87110159,490.23645075)(1112.87110159,489.12711697)
\closepath
}
}
{
\newrgbcolor{curcolor}{0 0 0}
\pscustom[linestyle=none,fillstyle=solid,fillcolor=curcolor]
{
\newpath
\moveto(1124.77498744,482.08711417)
\curveto(1124.77498744,482.53511435)(1124.90298749,482.87644782)(1125.1589876,483.11111458)
\curveto(1125.43632104,483.34578134)(1125.75632117,483.46311472)(1126.11898798,483.46311472)
\curveto(1126.50298813,483.46311472)(1126.82298826,483.34578134)(1127.07898836,483.11111458)
\curveto(1127.33498846,482.85511448)(1127.46298851,482.52444768)(1127.46298851,482.11911418)
\curveto(1127.46298851,481.32978054)(1127.04698835,480.88178036)(1126.21498802,480.77511365)
\curveto(1126.62032151,480.39111349)(1127.12165504,480.19911342)(1127.71898862,480.19911342)
\curveto(1128.35898887,480.19911342)(1128.9349891,480.43378018)(1129.4469893,480.9031137)
\curveto(1129.95898951,481.37244722)(1130.32165632,481.83111407)(1130.53498974,482.27911425)
\curveto(1130.7696565,482.72711442)(1131.04698994,483.35644801)(1131.36699007,484.167115)
\curveto(1131.66565685,484.80711525)(1131.93232363,485.43644884)(1132.16699039,486.05511575)
\lineto(1127.36698848,497.7351204)
\curveto(1127.15365506,498.2471206)(1126.88698828,498.55645406)(1126.56698816,498.66312077)
\curveto(1126.24698803,498.79112082)(1125.64965446,498.85512084)(1124.77498744,498.85512084)
\lineto(1124.77498744,499.84712124)
\curveto(1125.73498783,499.78312121)(1126.7696549,499.7511212)(1127.87898868,499.7511212)
\curveto(1128.54032228,499.7511212)(1129.70298941,499.78312121)(1131.36699007,499.84712124)
\lineto(1131.36699007,498.85512084)
\curveto(1130.17232293,498.85512084)(1129.57498935,498.5777874)(1129.57498935,498.02312051)
\curveto(1129.57498935,497.95912049)(1129.63898938,497.76712041)(1129.76698943,497.44712028)
\lineto(1133.31899084,488.83911686)
\lineto(1136.55099213,496.71111999)
\curveto(1136.67899218,497.00978678)(1136.74299221,497.26578688)(1136.74299221,497.4791203)
\curveto(1136.74299221,498.35378731)(1136.24165867,498.81245416)(1135.23899161,498.85512084)
\lineto(1135.23899161,499.84712124)
\curveto(1136.64699217,499.78312121)(1137.57499254,499.7511212)(1138.02299272,499.7511212)
\curveto(1138.89765973,499.7511212)(1139.69766005,499.78312121)(1140.42299367,499.84712124)
\lineto(1140.42299367,498.85512084)
\curveto(1138.99365977,498.85512084)(1138.00165937,498.17245391)(1137.44699249,496.80712003)
\lineto(1131.81499025,483.14311459)
\curveto(1130.79098984,480.71111362)(1129.42565596,479.49511314)(1127.71898862,479.49511314)
\curveto(1126.90832163,479.49511314)(1126.21498802,479.75111324)(1125.63898779,480.26311344)
\curveto(1125.06298756,480.75378031)(1124.77498744,481.36178055)(1124.77498744,482.08711417)
\closepath
}
}
{
\newrgbcolor{curcolor}{0 0 0}
\pscustom[linestyle=none,fillstyle=solid,fillcolor=curcolor]
{
\newpath
\moveto(1141.03088739,493.09511855)
\curveto(1141.03088739,495.10045268)(1141.67088765,496.8177867)(1142.95088816,498.2471206)
\curveto(1144.25222201,499.6764545)(1145.8308893,500.39112145)(1147.68689004,500.39112145)
\curveto(1149.56422412,500.39112145)(1150.98289135,499.78312121)(1151.94289173,498.56712073)
\curveto(1152.92422546,497.35112024)(1153.41489232,495.86845299)(1153.41489232,494.11911896)
\curveto(1153.41489232,493.79911883)(1153.37222564,493.60711875)(1153.28689227,493.54311873)
\curveto(1153.2015589,493.4791187)(1152.97755881,493.44711869)(1152.614892,493.44711869)
\lineto(1143.68688845,493.44711869)
\curveto(1143.68688845,491.27111783)(1144.00688858,489.67111719)(1144.64688883,488.64711678)
\curveto(1145.54288919,487.21778288)(1146.748223,486.50311593)(1148.26289027,486.50311593)
\curveto(1148.47622369,486.50311593)(1148.70022378,486.52444927)(1148.93489054,486.56711595)
\curveto(1149.19089064,486.60978264)(1149.57489079,486.71644935)(1150.086891,486.88711608)
\curveto(1150.5988912,487.07911616)(1151.07889139,487.42044963)(1151.52689157,487.91111649)
\curveto(1151.97489175,488.40178335)(1152.32689189,489.03111693)(1152.58289199,489.79911724)
\curveto(1152.64689201,490.09778402)(1152.78555874,490.24711742)(1152.99889215,490.24711742)
\curveto(1153.2762256,490.24711742)(1153.41489232,490.11911737)(1153.41489232,489.86311726)
\curveto(1153.41489232,489.67111719)(1153.30822561,489.36178373)(1153.09489219,488.9351169)
\curveto(1152.90289212,488.5297834)(1152.614892,488.07111655)(1152.23089185,487.55911635)
\curveto(1151.8468917,487.06844949)(1151.28155814,486.63111598)(1150.53489117,486.24711583)
\curveto(1149.78822421,485.88444901)(1148.96689055,485.70311561)(1148.07089019,485.70311561)
\curveto(1146.19355611,485.70311561)(1144.55088879,486.40711589)(1143.14288823,487.81511645)
\curveto(1141.73488767,489.24445035)(1141.03088739,491.00445105)(1141.03088739,493.09511855)
\closepath
\moveto(1143.71888846,494.11911896)
\lineto(1151.30289148,494.11911896)
\curveto(1151.30289148,494.56711914)(1151.2602248,495.03645266)(1151.17489143,495.52711952)
\curveto(1151.1108914,496.03911972)(1150.96155801,496.63645329)(1150.72689125,497.31912023)
\curveto(1150.51355783,498.02312051)(1150.14022435,498.58845407)(1149.6068908,499.01512091)
\curveto(1149.0948906,499.46312109)(1148.45489035,499.68712117)(1147.68689004,499.68712117)
\curveto(1147.34555657,499.68712117)(1146.98288976,499.61245448)(1146.59888961,499.46312109)
\curveto(1146.2362228,499.31378769)(1145.8308893,499.05778759)(1145.38288912,498.69512078)
\curveto(1144.93488895,498.35378731)(1144.55088879,497.77778708)(1144.23088867,496.96712009)
\curveto(1143.93222188,496.17778644)(1143.76155514,495.22845273)(1143.71888846,494.11911896)
\closepath
}
}
{
\newrgbcolor{curcolor}{0 0 0}
\pscustom[linestyle=none,fillstyle=solid,fillcolor=curcolor]
{
\newpath
\moveto(1155.2388409,486.05511575)
\lineto(1155.2388409,487.04711614)
\curveto(1156.36950801,487.04711614)(1157.06284162,487.11111617)(1157.31884172,487.23911622)
\curveto(1157.59617517,487.38844961)(1157.73484189,487.80444978)(1157.73484189,488.48711672)
\lineto(1157.73484189,497.06312013)
\curveto(1157.73484189,497.85245378)(1157.59617517,498.34312064)(1157.31884172,498.53512072)
\curveto(1157.04150828,498.74845413)(1156.34817467,498.85512084)(1155.2388409,498.85512084)
\lineto(1155.2388409,499.84712124)
\lineto(1159.68684267,500.19912138)
\lineto(1159.68684267,496.67911998)
\curveto(1159.98550945,497.59645368)(1160.45484297,498.40712067)(1161.09484323,499.11112095)
\curveto(1161.73484348,499.83645457)(1162.57751048,500.19912138)(1163.62284423,500.19912138)
\curveto(1164.30551117,500.19912138)(1164.87084473,500.0071213)(1165.31884491,499.62312115)
\curveto(1165.76684509,499.239121)(1165.99084518,498.78045415)(1165.99084518,498.2471206)
\curveto(1165.99084518,497.77778708)(1165.84151178,497.42578694)(1165.542845,497.19112018)
\curveto(1165.26551155,496.95645342)(1164.9561781,496.83912004)(1164.61484463,496.83912004)
\curveto(1164.23084448,496.83912004)(1163.90017768,496.95645342)(1163.62284423,497.19112018)
\curveto(1163.36684413,497.44712028)(1163.23884408,497.78845375)(1163.23884408,498.21512059)
\curveto(1163.23884408,498.47112069)(1163.29217744,498.69512078)(1163.39884414,498.88712086)
\curveto(1163.5268442,499.10045427)(1163.6335109,499.239121)(1163.71884427,499.30312102)
\curveto(1163.82551098,499.38845439)(1163.91084435,499.44178774)(1163.97484437,499.46312109)
\curveto(1163.93217769,499.48445443)(1163.81484431,499.4951211)(1163.62284423,499.4951211)
\curveto(1162.44951043,499.4951211)(1161.52151006,498.9084542)(1160.83884313,497.7351204)
\curveto(1160.17750953,496.5617866)(1159.84684273,495.14311937)(1159.84684273,493.4791187)
\lineto(1159.84684273,488.55111674)
\curveto(1159.84684273,487.93244983)(1159.97484278,487.52711633)(1160.23084288,487.33511626)
\curveto(1160.48684299,487.14311618)(1161.16950992,487.04711614)(1162.2788437,487.04711614)
\lineto(1162.95084397,487.04711614)
\lineto(1162.95084397,486.05511575)
\curveto(1162.09751029,486.11911577)(1160.74284309,486.15111579)(1158.88684235,486.15111579)
\curveto(1158.63084225,486.15111579)(1158.30017545,486.14044912)(1157.89484195,486.11911577)
\curveto(1157.48950846,486.11911577)(1157.02017494,486.1084491)(1156.48684139,486.08711576)
\curveto(1155.95350785,486.06578242)(1155.53750768,486.05511575)(1155.2388409,486.05511575)
\closepath
}
}
{
\newrgbcolor{curcolor}{0 0 0}
\pscustom[linestyle=none,fillstyle=solid,fillcolor=curcolor]
{
\newpath
\moveto(1178.56675781,486.05511575)
\lineto(1178.56675781,487.04711614)
\curveto(1179.69742492,487.04711614)(1180.39075853,487.11111617)(1180.64675863,487.23911622)
\curveto(1180.92409208,487.38844961)(1181.0627588,487.80444978)(1181.0627588,488.48711672)
\lineto(1181.0627588,497.09512014)
\curveto(1181.0627588,497.88445379)(1180.92409208,498.37512065)(1180.64675863,498.56712073)
\curveto(1180.39075853,498.75912081)(1179.74009161,498.85512084)(1178.69475786,498.85512084)
\lineto(1178.69475786,499.84712124)
\lineto(1183.17475964,500.19912138)
\lineto(1183.17475964,488.4551167)
\curveto(1183.17475964,487.81511645)(1183.28142635,487.42044963)(1183.49475977,487.27111623)
\curveto(1183.72942653,487.12178284)(1184.36942678,487.04711614)(1185.41476053,487.04711614)
\lineto(1185.41476053,486.05511575)
\curveto(1183.23875967,486.11911577)(1182.12942589,486.15111579)(1182.08675921,486.15111579)
\curveto(1181.78809242,486.15111579)(1180.61475862,486.11911577)(1178.56675781,486.05511575)
\closepath
\moveto(1179.91075834,505.76712359)
\curveto(1179.91075834,506.19379043)(1180.07075841,506.57779058)(1180.39075853,506.91912405)
\curveto(1180.732092,507.28179086)(1181.1374255,507.46312427)(1181.60675902,507.46312427)
\curveto(1182.07609254,507.46312427)(1182.47075936,507.30312421)(1182.79075949,506.98312408)
\curveto(1183.13209296,506.66312395)(1183.30275969,506.25779046)(1183.30275969,505.76712359)
\curveto(1183.30275969,505.27645673)(1183.13209296,504.87112324)(1182.79075949,504.55112311)
\curveto(1182.47075936,504.23112298)(1182.07609254,504.07112292)(1181.60675902,504.07112292)
\curveto(1181.11609215,504.07112292)(1180.71075866,504.24178965)(1180.39075853,504.58312312)
\curveto(1180.07075841,504.92445659)(1179.91075834,505.31912342)(1179.91075834,505.76712359)
\closepath
}
}
{
\newrgbcolor{curcolor}{0 0 0}
\pscustom[linestyle=none,fillstyle=solid,fillcolor=curcolor]
{
\newpath
\moveto(1198.85465894,490.91911768)
\curveto(1198.87599228,491.17511779)(1199.0893257,491.30311784)(1199.49465919,491.30311784)
\lineto(1219.110667,491.30311784)
\curveto(1219.72933391,491.30311784)(1220.03866737,491.18578446)(1220.03866737,490.9511177)
\curveto(1220.03866737,490.6951176)(1219.75066725,490.56711754)(1219.17466702,490.56711754)
\lineto(1199.71865928,490.56711754)
\curveto(1199.14265905,490.56711754)(1198.85465894,490.68445092)(1198.85465894,490.91911768)
\closepath
\moveto(1198.85465894,497.15912017)
\curveto(1198.85465894,497.41512027)(1199.07865903,497.54312032)(1199.52665921,497.54312032)
\lineto(1219.07866699,497.54312032)
\curveto(1219.71866724,497.54312032)(1220.03866737,497.41512027)(1220.03866737,497.15912017)
\curveto(1220.03866737,496.92445341)(1219.7720006,496.80712003)(1219.23866705,496.80712003)
\lineto(1199.49465919,496.80712003)
\curveto(1199.06799236,496.80712003)(1198.85465894,496.92445341)(1198.85465894,497.15912017)
\closepath
}
}
{
\newrgbcolor{curcolor}{0 0 0}
\pscustom[linestyle=none,fillstyle=solid,fillcolor=curcolor]
{
\newpath
\moveto(1233.63861829,486.05511575)
\lineto(1233.63861829,487.04711614)
\curveto(1234.4066186,487.04711614)(1235.03595218,487.12178284)(1235.52661904,487.27111623)
\curveto(1236.01728591,487.42044963)(1236.3479527,487.63378304)(1236.51861944,487.91111649)
\curveto(1236.71061952,488.20978327)(1236.8279529,488.4551167)(1236.87061958,488.64711678)
\curveto(1236.91328626,488.83911686)(1236.9346196,489.09511696)(1236.9346196,489.41511709)
\lineto(1236.9346196,506.11912373)
\curveto(1236.9346196,506.41779052)(1236.89195292,506.59912393)(1236.80661955,506.66312395)
\curveto(1236.74261953,506.74845732)(1236.52928611,506.81245734)(1236.1666193,506.85512403)
\curveto(1235.52661904,506.89779071)(1234.9932855,506.91912405)(1234.56661866,506.91912405)
\lineto(1233.63861829,506.91912405)
\lineto(1233.63861829,507.91112445)
\lineto(1239.07862046,507.91112445)
\curveto(1239.37728724,507.91112445)(1239.56928732,507.88979111)(1239.65462069,507.84712442)
\curveto(1239.73995405,507.80445774)(1239.85728743,507.68712436)(1240.00662083,507.49512428)
\lineto(1251.30262532,490.88711767)
\lineto(1251.30262532,504.55112311)
\curveto(1251.30262532,504.87112324)(1251.28129198,505.12712334)(1251.2386253,505.31912342)
\curveto(1251.19595861,505.51112349)(1251.07862523,505.74579025)(1250.88662516,506.0231237)
\curveto(1250.71595842,506.32179048)(1250.38529162,506.54579057)(1249.89462476,506.69512396)
\curveto(1249.4039579,506.84445736)(1248.77462432,506.91912405)(1248.00662401,506.91912405)
\lineto(1248.00662401,507.91112445)
\curveto(1250.2466249,507.84712442)(1251.4946254,507.81512441)(1251.7506255,507.81512441)
\curveto(1252.0066256,507.81512441)(1253.2546261,507.84712442)(1255.49462699,507.91112445)
\lineto(1255.49462699,506.91912405)
\curveto(1254.72662669,506.91912405)(1254.0972931,506.84445736)(1253.60662624,506.69512396)
\curveto(1253.11595938,506.54579057)(1252.77462591,506.32179048)(1252.58262583,506.0231237)
\curveto(1252.4119591,505.74579025)(1252.30529239,505.51112349)(1252.26262571,505.31912342)
\curveto(1252.21995902,505.12712334)(1252.19862568,504.87112324)(1252.19862568,504.55112311)
\lineto(1252.19862568,486.88711608)
\curveto(1252.19862568,486.54578261)(1252.17729234,486.32178252)(1252.13462565,486.21511581)
\curveto(1252.09195897,486.1084491)(1251.96395892,486.05511575)(1251.7506255,486.05511575)
\curveto(1251.60129211,486.05511575)(1251.4199587,486.19378247)(1251.20662528,486.47111591)
\lineto(1238.15062009,505.67112356)
\curveto(1238.06528672,505.79912361)(1237.95862001,505.92712366)(1237.83061996,506.05512371)
\lineto(1237.83061996,489.41511709)
\curveto(1237.83061996,489.09511696)(1237.8519533,488.83911686)(1237.89461999,488.64711678)
\curveto(1237.93728667,488.4551167)(1238.04395338,488.20978327)(1238.21462011,487.91111649)
\curveto(1238.40662019,487.63378304)(1238.74795366,487.42044963)(1239.23862052,487.27111623)
\curveto(1239.72928738,487.12178284)(1240.35862097,487.04711614)(1241.12662127,487.04711614)
\lineto(1241.12662127,486.05511575)
\curveto(1238.88662038,486.11911577)(1237.63861989,486.15111579)(1237.38261978,486.15111579)
\curveto(1237.12661968,486.15111579)(1235.87861918,486.11911577)(1233.63861829,486.05511575)
\closepath
}
}
{
\newrgbcolor{curcolor}{0 0 0}
\pscustom[linestyle=none,fillstyle=solid,fillcolor=curcolor]
{
\newpath
\moveto(1267.55851428,492.00711812)
\lineto(1267.55851428,493.89511887)
\lineto(1276.07051767,493.89511887)
\lineto(1276.07051767,492.00711812)
\closepath
}
}
{
\newrgbcolor{curcolor}{0 0 0}
\pscustom[linestyle=none,fillstyle=solid,fillcolor=curcolor]
{
\newpath
\moveto(1291.39842032,504.32712302)
\lineto(1291.39842032,505.31912342)
\curveto(1293.95842134,505.31912342)(1295.89975544,506.00179035)(1297.22242264,507.36712423)
\curveto(1297.58508945,507.36712423)(1297.79842287,507.32445755)(1297.86242289,507.23912418)
\curveto(1297.92642292,507.15379081)(1297.95842293,506.91912405)(1297.95842293,506.5351239)
\lineto(1297.95842293,488.58311676)
\curveto(1297.95842293,487.9431165)(1298.10775632,487.52711633)(1298.40642311,487.33511626)
\curveto(1298.72642323,487.14311618)(1299.56909024,487.04711614)(1300.93442411,487.04711614)
\lineto(1301.95842452,487.04711614)
\lineto(1301.95842452,486.05511575)
\curveto(1301.21175756,486.11911577)(1299.48375687,486.15111579)(1296.77442246,486.15111579)
\curveto(1294.06508805,486.15111579)(1292.33708736,486.11911577)(1291.59042039,486.05511575)
\lineto(1291.59042039,487.04711614)
\lineto(1292.6144208,487.04711614)
\curveto(1293.95842134,487.04711614)(1294.79042167,487.14311618)(1295.1104218,487.33511626)
\curveto(1295.43042192,487.52711633)(1295.59042199,487.9431165)(1295.59042199,488.58311676)
\lineto(1295.59042199,505.15912335)
\curveto(1294.48108821,504.60445646)(1293.08375432,504.32712302)(1291.39842032,504.32712302)
\closepath
}
}
{
\newrgbcolor{curcolor}{0 0 0}
\pscustom[linewidth=1.89354326,linecolor=curcolor]
{
\newpath
\moveto(1047.91466835,517.89710057)
\lineto(1052.91468094,517.89710057)
\lineto(1052.91468094,471.89709805)
\lineto(1047.91466835,471.89709805)
}
}
{
\newrgbcolor{curcolor}{0 0 0}
\pscustom[linewidth=1.89354326,linecolor=curcolor]
{
\newpath
\moveto(1052.91468094,492.89709805)
\lineto(1057.91469354,492.89709805)
}
}
{
\newrgbcolor{curcolor}{0 0 0}
\pscustom[linestyle=none,fillstyle=solid,fillcolor=curcolor]
{
\newpath
\moveto(1070.31515013,136.05878026)
\lineto(1070.31515013,137.05078066)
\lineto(1071.08315044,137.05078066)
\curveto(1072.2351509,137.05078066)(1072.93915118,137.14678069)(1073.19515128,137.33878077)
\curveto(1073.47248472,137.53078085)(1073.61115145,137.93611434)(1073.61115145,138.55478125)
\lineto(1073.61115145,155.41878797)
\curveto(1073.61115145,156.03745488)(1073.47248472,156.44278837)(1073.19515128,156.63478845)
\curveto(1072.93915118,156.82678853)(1072.2351509,156.92278857)(1071.08315044,156.92278857)
\lineto(1070.31515013,156.92278857)
\lineto(1070.31515013,157.91478896)
\lineto(1082.0271548,157.91478896)
\curveto(1084.05382227,157.91478896)(1085.73915627,157.37078874)(1087.08315681,156.28278831)
\curveto(1088.44849069,155.19478788)(1089.13115762,153.93612071)(1089.13115762,152.50678681)
\curveto(1089.13115762,151.33345301)(1088.62982409,150.27745259)(1087.62715702,149.33878555)
\curveto(1086.6458233,148.40011851)(1085.37648946,147.78145159)(1083.81915551,147.48278481)
\curveto(1085.58982288,147.29078473)(1087.06182347,146.66145115)(1088.23515727,145.59478406)
\curveto(1089.40849107,144.52811697)(1089.99515797,143.30144981)(1089.99515797,141.91478259)
\curveto(1089.99515797,140.35744864)(1089.3231577,138.99211476)(1087.97915717,137.81878096)
\curveto(1086.63515663,136.64544716)(1084.92848928,136.05878026)(1082.85915513,136.05878026)
\closepath
\moveto(1076.2671525,138.33078117)
\curveto(1076.2671525,137.77611428)(1076.3418192,137.42411414)(1076.49115259,137.27478075)
\curveto(1076.66181933,137.12544735)(1077.10981951,137.05078066)(1077.83515313,137.05078066)
\lineto(1081.83515472,137.05078066)
\curveto(1083.37115533,137.05078066)(1084.57648914,137.55211419)(1085.45115616,138.55478125)
\curveto(1086.34715652,139.55744832)(1086.79515669,140.68811544)(1086.79515669,141.9467826)
\curveto(1086.79515669,143.20544977)(1086.40048987,144.3681169)(1085.61115622,145.43478399)
\curveto(1084.84315592,146.52278443)(1083.73382214,147.06678464)(1082.2831549,147.06678464)
\lineto(1076.2671525,147.06678464)
\closepath
\moveto(1076.2671525,147.77078492)
\lineto(1080.90715435,147.77078492)
\curveto(1082.54982167,147.77078492)(1083.80848884,148.26145178)(1084.68315585,149.24278551)
\curveto(1085.57915621,150.24545257)(1086.02715639,151.33345301)(1086.02715639,152.50678681)
\curveto(1086.02715639,153.55212056)(1085.67515625,154.54412095)(1084.97115597,155.48278799)
\curveto(1084.26715569,156.44278837)(1083.21115527,156.92278857)(1081.80315471,156.92278857)
\lineto(1077.83515313,156.92278857)
\curveto(1077.10981951,156.92278857)(1076.66181933,156.84812187)(1076.49115259,156.69878848)
\curveto(1076.3418192,156.54945508)(1076.2671525,156.19745494)(1076.2671525,155.64278806)
\closepath
}
}
{
\newrgbcolor{curcolor}{0 0 0}
\pscustom[linestyle=none,fillstyle=solid,fillcolor=curcolor]
{
\newpath
\moveto(1092.87510333,136.05878026)
\lineto(1092.87510333,137.05078066)
\curveto(1094.00577045,137.05078066)(1094.69910406,137.11478068)(1094.95510416,137.24278073)
\curveto(1095.2324376,137.39211413)(1095.37110432,137.80811429)(1095.37110432,138.49078123)
\lineto(1095.37110432,147.09878466)
\curveto(1095.37110432,147.8881183)(1095.2324376,148.37878516)(1094.95510416,148.57078524)
\curveto(1094.69910406,148.76278532)(1094.04843713,148.85878536)(1093.00310338,148.85878536)
\lineto(1093.00310338,149.85078575)
\lineto(1097.48310517,150.20278589)
\lineto(1097.48310517,138.45878122)
\curveto(1097.48310517,137.81878096)(1097.58977187,137.42411414)(1097.80310529,137.27478075)
\curveto(1098.03777205,137.12544735)(1098.67777231,137.05078066)(1099.72310606,137.05078066)
\lineto(1099.72310606,136.05878026)
\curveto(1097.54710519,136.12278029)(1096.43777142,136.1547803)(1096.39510473,136.1547803)
\curveto(1096.09643795,136.1547803)(1094.92310415,136.12278029)(1092.87510333,136.05878026)
\closepath
\moveto(1094.21910387,155.77078811)
\curveto(1094.21910387,156.19745494)(1094.37910393,156.5814551)(1094.69910406,156.92278857)
\curveto(1095.04043753,157.28545538)(1095.44577102,157.46678878)(1095.91510454,157.46678878)
\curveto(1096.38443806,157.46678878)(1096.77910489,157.30678872)(1097.09910501,156.98678859)
\curveto(1097.44043848,156.66678846)(1097.61110522,156.26145497)(1097.61110522,155.77078811)
\curveto(1097.61110522,155.28012124)(1097.44043848,154.87478775)(1097.09910501,154.55478762)
\curveto(1096.77910489,154.2347875)(1096.38443806,154.07478743)(1095.91510454,154.07478743)
\curveto(1095.42443768,154.07478743)(1095.01910418,154.24545417)(1094.69910406,154.58678764)
\curveto(1094.37910393,154.9281211)(1094.21910387,155.32278793)(1094.21910387,155.77078811)
\closepath
}
}
{
\newrgbcolor{curcolor}{0 0 0}
\pscustom[linestyle=none,fillstyle=solid,fillcolor=curcolor]
{
\newpath
\moveto(1101.73905885,136.05878026)
\lineto(1101.73905885,137.05078066)
\curveto(1102.86972597,137.05078066)(1103.56305958,137.11478068)(1103.81905968,137.24278073)
\curveto(1104.09639312,137.39211413)(1104.23505984,137.80811429)(1104.23505984,138.49078123)
\lineto(1104.23505984,155.13078785)
\curveto(1104.23505984,155.9201215)(1104.09639312,156.41078836)(1103.81905968,156.60278844)
\curveto(1103.54172624,156.81612186)(1102.84839263,156.92278857)(1101.73905885,156.92278857)
\lineto(1101.73905885,157.91478896)
\lineto(1106.34706069,158.2667891)
\lineto(1106.34706069,138.49078123)
\curveto(1106.34706069,137.80811429)(1106.47506074,137.39211413)(1106.73106084,137.24278073)
\curveto(1107.00839428,137.11478068)(1107.71239456,137.05078066)(1108.84306168,137.05078066)
\lineto(1108.84306168,136.05878026)
\curveto(1108.58706158,136.05878026)(1108.18172808,136.06944693)(1107.62706119,136.09078027)
\curveto(1107.09372765,136.11211362)(1106.62439413,136.12278029)(1106.21906063,136.12278029)
\curveto(1105.83506048,136.14411363)(1105.52572703,136.1547803)(1105.29106027,136.1547803)
\curveto(1105.03506016,136.1547803)(1103.85105969,136.12278029)(1101.73905885,136.05878026)
\closepath
}
}
{
\newrgbcolor{curcolor}{0 0 0}
\pscustom[linestyle=none,fillstyle=solid,fillcolor=curcolor]
{
\newpath
\moveto(1110.89101809,139.09878147)
\curveto(1110.89101809,140.91211553)(1111.95768518,142.29878274)(1114.09101936,143.25878313)
\curveto(1115.37101987,143.87745004)(1117.33368732,144.26145019)(1119.97902171,144.41078359)
\lineto(1119.97902171,145.59478406)
\curveto(1119.97902171,146.91745125)(1119.62702157,147.93078499)(1118.92302129,148.63478527)
\curveto(1118.24035435,149.33878555)(1117.46168737,149.69078569)(1116.58702036,149.69078569)
\curveto(1115.0296864,149.69078569)(1113.87768595,149.20011882)(1113.13101898,148.2187851)
\curveto(1113.77101924,148.19745176)(1114.19768607,148.02678502)(1114.41101949,147.7067849)
\curveto(1114.64568625,147.38678477)(1114.76301963,147.06678464)(1114.76301963,146.74678452)
\curveto(1114.76301963,146.32011768)(1114.62435291,145.96811754)(1114.34701947,145.69078409)
\curveto(1114.09101936,145.41345065)(1113.73901922,145.27478393)(1113.29101905,145.27478393)
\curveto(1112.86435221,145.27478393)(1112.51235207,145.40278398)(1112.23501863,145.65878408)
\curveto(1111.95768518,145.93611753)(1111.81901846,146.30945101)(1111.81901846,146.77878453)
\curveto(1111.81901846,147.82411828)(1112.28835198,148.68811862)(1113.22701902,149.37078556)
\curveto(1114.16568606,150.0534525)(1115.30701985,150.39478597)(1116.65102038,150.39478597)
\curveto(1118.40035441,150.39478597)(1119.86168833,149.80811907)(1121.03502213,148.63478527)
\curveto(1121.39768894,148.27211846)(1121.66435571,147.85611829)(1121.83502245,147.38678477)
\curveto(1122.02702252,146.91745125)(1122.13368923,146.52278443)(1122.15502257,146.2027843)
\curveto(1122.17635592,145.90411751)(1122.18702259,145.45611733)(1122.18702259,144.85878376)
\lineto(1122.18702259,138.45878122)
\curveto(1122.18702259,138.33078117)(1122.20835593,138.16011443)(1122.25102261,137.94678101)
\curveto(1122.2936893,137.75478094)(1122.41102268,137.52011418)(1122.60302275,137.24278073)
\curveto(1122.79502283,136.98678063)(1123.05102293,136.85878058)(1123.37102306,136.85878058)
\curveto(1124.13902336,136.85878058)(1124.52302352,137.54144752)(1124.52302352,138.90678139)
\lineto(1124.52302352,140.69878211)
\lineto(1125.32302383,140.69878211)
\lineto(1125.32302383,138.90678139)
\curveto(1125.32302383,137.77611428)(1125.02435705,136.98678063)(1124.42702348,136.53878045)
\curveto(1123.82968991,136.09078027)(1123.25368968,135.86678019)(1122.69902279,135.86678019)
\curveto(1121.99502251,135.86678019)(1121.41902228,136.12278029)(1120.9710221,136.63478049)
\curveto(1120.52302192,137.16811404)(1120.26702182,137.78678095)(1120.2030218,138.49078123)
\curveto(1119.88302167,137.68011424)(1119.34968812,137.00811397)(1118.60302116,136.47478043)
\curveto(1117.87768754,135.96278022)(1117.01368719,135.70678012)(1116.01102013,135.70678012)
\curveto(1115.24301982,135.70678012)(1114.49635286,135.80278016)(1113.77101924,135.99478024)
\curveto(1113.04568562,136.18678031)(1112.38435202,136.54944712)(1111.78701845,137.08278067)
\curveto(1111.18968488,137.61611421)(1110.89101809,138.28811448)(1110.89101809,139.09878147)
\closepath
\moveto(1113.35501907,139.13078148)
\curveto(1113.35501907,138.34144784)(1113.63235252,137.69078091)(1114.1870194,137.17878071)
\curveto(1114.76301963,136.6667805)(1115.44568657,136.4107804)(1116.23502022,136.4107804)
\curveto(1117.13102057,136.4107804)(1117.97368758,136.75211387)(1118.76302122,137.43478081)
\curveto(1119.57368821,138.13878109)(1119.97902171,139.17344817)(1119.97902171,140.53878204)
\lineto(1119.97902171,143.73878332)
\curveto(1117.61102077,143.65344995)(1115.91502009,143.14144975)(1114.89101968,142.20278271)
\curveto(1113.86701928,141.26411567)(1113.35501907,140.24011526)(1113.35501907,139.13078148)
\closepath
}
}
{
\newrgbcolor{curcolor}{0 0 0}
\pscustom[linestyle=none,fillstyle=solid,fillcolor=curcolor]
{
\newpath
\moveto(1125.25890492,132.09077868)
\curveto(1125.25890492,132.53877886)(1125.38690497,132.88011233)(1125.64290507,133.11477909)
\curveto(1125.92023852,133.34944585)(1126.24023864,133.46677923)(1126.60290546,133.46677923)
\curveto(1126.98690561,133.46677923)(1127.30690574,133.34944585)(1127.56290584,133.11477909)
\curveto(1127.81890594,132.85877899)(1127.94690599,132.52811219)(1127.94690599,132.12277869)
\curveto(1127.94690599,131.33344505)(1127.53090582,130.88544487)(1126.69890549,130.77877816)
\curveto(1127.10423899,130.39477801)(1127.60557252,130.20277793)(1128.20290609,130.20277793)
\curveto(1128.84290635,130.20277793)(1129.41890658,130.43744469)(1129.93090678,130.90677821)
\curveto(1130.44290698,131.37611173)(1130.80557379,131.83477858)(1131.01890721,132.28277876)
\curveto(1131.25357397,132.73077894)(1131.53090742,133.36011252)(1131.85090754,134.17077951)
\curveto(1132.14957433,134.81077976)(1132.4162411,135.44011335)(1132.65090786,136.05878026)
\lineto(1127.85090595,147.73878491)
\curveto(1127.63757253,148.25078511)(1127.37090576,148.56011857)(1127.05090563,148.66678528)
\curveto(1126.73090551,148.79478533)(1126.13357194,148.85878536)(1125.25890492,148.85878536)
\lineto(1125.25890492,149.85078575)
\curveto(1126.2189053,149.78678573)(1127.25357238,149.75478571)(1128.36290616,149.75478571)
\curveto(1129.02423975,149.75478571)(1130.18690688,149.78678573)(1131.85090754,149.85078575)
\lineto(1131.85090754,148.85878536)
\curveto(1130.6562404,148.85878536)(1130.05890683,148.58145191)(1130.05890683,148.02678502)
\curveto(1130.05890683,147.962785)(1130.12290686,147.77078492)(1130.25090691,147.4507848)
\lineto(1133.80290832,138.84278137)
\lineto(1137.03490961,146.7147845)
\curveto(1137.16290966,147.01345129)(1137.22690968,147.26945139)(1137.22690968,147.48278481)
\curveto(1137.22690968,148.35745182)(1136.72557615,148.81611867)(1135.72290908,148.85878536)
\lineto(1135.72290908,149.85078575)
\curveto(1137.13090965,149.78678573)(1138.05891001,149.75478571)(1138.50691019,149.75478571)
\curveto(1139.38157721,149.75478571)(1140.18157753,149.78678573)(1140.90691115,149.85078575)
\lineto(1140.90691115,148.85878536)
\curveto(1139.47757725,148.85878536)(1138.48557685,148.17611842)(1137.93090996,146.81078454)
\lineto(1132.29890772,133.1467791)
\curveto(1131.27490731,130.71477813)(1129.90957344,129.49877765)(1128.20290609,129.49877765)
\curveto(1127.3922391,129.49877765)(1126.69890549,129.75477775)(1126.12290526,130.26677796)
\curveto(1125.54690504,130.75744482)(1125.25890492,131.36544506)(1125.25890492,132.09077868)
\closepath
}
}
{
\newrgbcolor{curcolor}{0 0 0}
\pscustom[linestyle=none,fillstyle=solid,fillcolor=curcolor]
{
\newpath
\moveto(1141.51480487,143.09878306)
\curveto(1141.51480487,145.10411719)(1142.15480512,146.82145121)(1143.43480563,148.25078511)
\curveto(1144.73613948,149.68011902)(1146.31480678,150.39478597)(1148.17080752,150.39478597)
\curveto(1150.0481416,150.39478597)(1151.46680883,149.78678573)(1152.42680921,148.57078524)
\curveto(1153.40814293,147.35478476)(1153.8988098,145.8721175)(1153.8988098,144.12278347)
\curveto(1153.8988098,143.80278334)(1153.85614311,143.61078327)(1153.77080975,143.54678324)
\curveto(1153.68547638,143.48278322)(1153.46147629,143.4507832)(1153.09880948,143.4507832)
\lineto(1144.17080593,143.4507832)
\curveto(1144.17080593,141.27478234)(1144.49080605,139.6747817)(1145.13080631,138.65078129)
\curveto(1146.02680666,137.22144739)(1147.23214048,136.50678044)(1148.74680775,136.50678044)
\curveto(1148.96014116,136.50678044)(1149.18414125,136.52811378)(1149.41880801,136.57078047)
\curveto(1149.67480812,136.61344715)(1150.05880827,136.72011386)(1150.57080847,136.89078059)
\curveto(1151.08280868,137.08278067)(1151.56280887,137.42411414)(1152.01080905,137.914781)
\curveto(1152.45880922,138.40544786)(1152.81080936,139.03478145)(1153.06680947,139.80278175)
\curveto(1153.13080949,140.10144854)(1153.26947621,140.25078193)(1153.48280963,140.25078193)
\curveto(1153.76014308,140.25078193)(1153.8988098,140.12278188)(1153.8988098,139.86678178)
\curveto(1153.8988098,139.6747817)(1153.79214309,139.36544824)(1153.57880967,138.93878141)
\curveto(1153.38680959,138.53344791)(1153.09880948,138.07478106)(1152.71480933,137.56278086)
\curveto(1152.33080917,137.072114)(1151.76547561,136.63478049)(1151.01880865,136.25078034)
\curveto(1150.27214169,135.88811353)(1149.45080803,135.70678012)(1148.55480767,135.70678012)
\curveto(1146.67747359,135.70678012)(1145.03480627,136.4107804)(1143.62680571,137.81878096)
\curveto(1142.21880515,139.24811486)(1141.51480487,141.00811556)(1141.51480487,143.09878306)
\closepath
\moveto(1144.20280594,144.12278347)
\lineto(1151.78680896,144.12278347)
\curveto(1151.78680896,144.57078365)(1151.74414227,145.04011717)(1151.65880891,145.53078403)
\curveto(1151.59480888,146.04278423)(1151.44547549,146.64011781)(1151.21080873,147.32278474)
\curveto(1150.99747531,148.02678502)(1150.62414183,148.59211858)(1150.09080828,149.01878542)
\curveto(1149.57880808,149.4667856)(1148.93880782,149.69078569)(1148.17080752,149.69078569)
\curveto(1147.82947405,149.69078569)(1147.46680724,149.61611899)(1147.08280708,149.4667856)
\curveto(1146.72014027,149.3174522)(1146.31480678,149.0614521)(1145.8668066,148.69878529)
\curveto(1145.41880642,148.35745182)(1145.03480627,147.78145159)(1144.71480614,146.9707846)
\curveto(1144.41613936,146.18145096)(1144.24547262,145.23211725)(1144.20280594,144.12278347)
\closepath
}
}
{
\newrgbcolor{curcolor}{0 0 0}
\pscustom[linestyle=none,fillstyle=solid,fillcolor=curcolor]
{
\newpath
\moveto(1155.72275837,136.05878026)
\lineto(1155.72275837,137.05078066)
\curveto(1156.85342549,137.05078066)(1157.5467591,137.11478068)(1157.8027592,137.24278073)
\curveto(1158.08009265,137.39211413)(1158.21875937,137.80811429)(1158.21875937,138.49078123)
\lineto(1158.21875937,147.06678464)
\curveto(1158.21875937,147.85611829)(1158.08009265,148.34678515)(1157.8027592,148.53878523)
\curveto(1157.52542576,148.75211865)(1156.83209215,148.85878536)(1155.72275837,148.85878536)
\lineto(1155.72275837,149.85078575)
\lineto(1160.17076014,150.20278589)
\lineto(1160.17076014,146.68278449)
\curveto(1160.46942693,147.60011819)(1160.93876045,148.41078518)(1161.5787607,149.11478546)
\curveto(1162.21876096,149.84011908)(1163.06142796,150.20278589)(1164.10676171,150.20278589)
\curveto(1164.78942865,150.20278589)(1165.35476221,150.01078581)(1165.80276239,149.62678566)
\curveto(1166.25076256,149.24278551)(1166.47476265,148.78411866)(1166.47476265,148.25078511)
\curveto(1166.47476265,147.78145159)(1166.32542926,147.42945145)(1166.02676247,147.19478469)
\curveto(1165.74942903,146.96011793)(1165.44009557,146.84278455)(1165.09876211,146.84278455)
\curveto(1164.71476195,146.84278455)(1164.38409515,146.96011793)(1164.10676171,147.19478469)
\curveto(1163.85076161,147.4507848)(1163.72276156,147.79211826)(1163.72276156,148.2187851)
\curveto(1163.72276156,148.4747852)(1163.77609491,148.69878529)(1163.88276162,148.89078537)
\curveto(1164.01076167,149.10411879)(1164.11742838,149.24278551)(1164.20276175,149.30678553)
\curveto(1164.30942846,149.3921189)(1164.39476183,149.44545226)(1164.45876185,149.4667856)
\curveto(1164.41609517,149.48811894)(1164.29876179,149.49878561)(1164.10676171,149.49878561)
\curveto(1162.93342791,149.49878561)(1162.00542754,148.91211871)(1161.3227606,147.73878491)
\curveto(1160.66142701,146.56545111)(1160.33076021,145.14678388)(1160.33076021,143.48278322)
\lineto(1160.33076021,138.55478125)
\curveto(1160.33076021,137.93611434)(1160.45876026,137.53078085)(1160.71476036,137.33878077)
\curveto(1160.97076046,137.14678069)(1161.6534274,137.05078066)(1162.76276118,137.05078066)
\lineto(1163.43476144,137.05078066)
\lineto(1163.43476144,136.05878026)
\curveto(1162.58142777,136.12278029)(1161.22676056,136.1547803)(1159.37075983,136.1547803)
\curveto(1159.11475972,136.1547803)(1158.78409293,136.14411363)(1158.37875943,136.12278029)
\curveto(1157.97342594,136.12278029)(1157.50409242,136.11211362)(1156.97075887,136.09078027)
\curveto(1156.43742532,136.06944693)(1156.02142516,136.05878026)(1155.72275837,136.05878026)
\closepath
}
}
{
\newrgbcolor{curcolor}{0 0 0}
\pscustom[linestyle=none,fillstyle=solid,fillcolor=curcolor]
{
\newpath
\moveto(1179.05067528,136.05878026)
\lineto(1179.05067528,137.05078066)
\curveto(1180.1813424,137.05078066)(1180.87467601,137.11478068)(1181.13067611,137.24278073)
\curveto(1181.40800956,137.39211413)(1181.54667628,137.80811429)(1181.54667628,138.49078123)
\lineto(1181.54667628,147.09878466)
\curveto(1181.54667628,147.8881183)(1181.40800956,148.37878516)(1181.13067611,148.57078524)
\curveto(1180.87467601,148.76278532)(1180.22400908,148.85878536)(1179.17867533,148.85878536)
\lineto(1179.17867533,149.85078575)
\lineto(1183.65867712,150.20278589)
\lineto(1183.65867712,138.45878122)
\curveto(1183.65867712,137.81878096)(1183.76534383,137.42411414)(1183.97867724,137.27478075)
\curveto(1184.213344,137.12544735)(1184.85334426,137.05078066)(1185.89867801,137.05078066)
\lineto(1185.89867801,136.05878026)
\curveto(1183.72267714,136.12278029)(1182.61334337,136.1547803)(1182.57067668,136.1547803)
\curveto(1182.2720099,136.1547803)(1181.0986761,136.12278029)(1179.05067528,136.05878026)
\closepath
\moveto(1180.39467582,155.77078811)
\curveto(1180.39467582,156.19745494)(1180.55467588,156.5814551)(1180.87467601,156.92278857)
\curveto(1181.21600948,157.28545538)(1181.62134297,157.46678878)(1182.09067649,157.46678878)
\curveto(1182.56001001,157.46678878)(1182.95467684,157.30678872)(1183.27467696,156.98678859)
\curveto(1183.61601043,156.66678846)(1183.78667717,156.26145497)(1183.78667717,155.77078811)
\curveto(1183.78667717,155.28012124)(1183.61601043,154.87478775)(1183.27467696,154.55478762)
\curveto(1182.95467684,154.2347875)(1182.56001001,154.07478743)(1182.09067649,154.07478743)
\curveto(1181.60000963,154.07478743)(1181.19467614,154.24545417)(1180.87467601,154.58678764)
\curveto(1180.55467588,154.9281211)(1180.39467582,155.32278793)(1180.39467582,155.77078811)
\closepath
}
}
{
\newrgbcolor{curcolor}{0 0 0}
\pscustom[linestyle=none,fillstyle=solid,fillcolor=curcolor]
{
\newpath
\moveto(1199.33857641,140.9227822)
\curveto(1199.35990976,141.1787823)(1199.57324317,141.30678235)(1199.97857667,141.30678235)
\lineto(1219.59458448,141.30678235)
\curveto(1220.21325139,141.30678235)(1220.52258485,141.18944897)(1220.52258485,140.95478221)
\curveto(1220.52258485,140.69878211)(1220.23458473,140.57078206)(1219.6585845,140.57078206)
\lineto(1200.20257676,140.57078206)
\curveto(1199.62657653,140.57078206)(1199.33857641,140.68811544)(1199.33857641,140.9227822)
\closepath
\moveto(1199.33857641,147.16278468)
\curveto(1199.33857641,147.41878478)(1199.5625765,147.54678483)(1200.01057668,147.54678483)
\lineto(1219.56258446,147.54678483)
\curveto(1220.20258472,147.54678483)(1220.52258485,147.41878478)(1220.52258485,147.16278468)
\curveto(1220.52258485,146.92811792)(1220.25591807,146.81078454)(1219.72258453,146.81078454)
\lineto(1199.97857667,146.81078454)
\curveto(1199.55190983,146.81078454)(1199.33857641,146.92811792)(1199.33857641,147.16278468)
\closepath
}
}
{
\newrgbcolor{curcolor}{0 0 0}
\pscustom[linestyle=none,fillstyle=solid,fillcolor=curcolor]
{
\newpath
\moveto(1234.66653599,136.05878026)
\curveto(1234.66653599,136.44278041)(1234.67720266,136.68811385)(1234.698536,136.79478055)
\curveto(1234.74120268,136.92278061)(1234.84786939,137.072114)(1235.01853613,137.24278073)
\lineto(1241.16253857,144.09078346)
\curveto(1243.40253946,146.60811779)(1244.52253991,148.96545206)(1244.52253991,151.16278627)
\curveto(1244.52253991,152.59212017)(1244.14920643,153.81878733)(1243.40253946,154.84278774)
\curveto(1242.6558725,155.86678814)(1241.59987208,156.37878835)(1240.2345382,156.37878835)
\curveto(1239.29587116,156.37878835)(1238.43187082,156.09078823)(1237.64253717,155.514788)
\curveto(1236.85320352,154.93878778)(1236.27720329,154.13878746)(1235.91453648,153.11478705)
\curveto(1235.97853651,153.13612039)(1236.11720323,153.14678706)(1236.33053665,153.14678706)
\curveto(1236.86387019,153.14678706)(1237.27987036,152.97612033)(1237.57853715,152.63478686)
\curveto(1237.87720393,152.31478673)(1238.02653732,151.93078658)(1238.02653732,151.4827864)
\curveto(1238.02653732,150.90678617)(1237.83453725,150.48011933)(1237.45053709,150.20278589)
\curveto(1237.08787028,149.92545245)(1236.72520347,149.78678573)(1236.36253666,149.78678573)
\curveto(1236.21320327,149.78678573)(1236.04253653,149.7974524)(1235.85053646,149.81878574)
\curveto(1235.65853638,149.86145242)(1235.41320295,150.03211916)(1235.11453616,150.33078594)
\curveto(1234.81586938,150.62945273)(1234.66653599,151.04545289)(1234.66653599,151.57878644)
\curveto(1234.66653599,153.07212037)(1235.23186954,154.40545423)(1236.36253666,155.57878803)
\curveto(1237.49320378,156.77345517)(1238.92253768,157.37078874)(1240.65053837,157.37078874)
\curveto(1242.61320582,157.37078874)(1244.23453979,156.78412184)(1245.5145403,155.61078804)
\curveto(1246.79454081,154.45878758)(1247.43454107,152.97612033)(1247.43454107,151.16278627)
\curveto(1247.43454107,150.52278602)(1247.33854103,149.91478578)(1247.14654095,149.33878555)
\curveto(1246.95454088,148.76278532)(1246.73054079,148.25078511)(1246.47454069,147.80278494)
\curveto(1246.21854058,147.35478476)(1245.74920706,146.76811786)(1245.06654013,146.04278423)
\curveto(1244.38387319,145.33878395)(1243.76520627,144.73078371)(1243.21053939,144.21878351)
\curveto(1242.6558725,143.70678331)(1241.75987214,142.90678299)(1240.52253832,141.81878255)
\lineto(1237.13053697,138.52278124)
\lineto(1242.89053926,138.52278124)
\curveto(1244.76787334,138.52278124)(1245.78120708,138.60811461)(1245.93054047,138.77878134)
\curveto(1246.14387389,139.07744813)(1246.37854065,140.02678184)(1246.63454075,141.62678248)
\lineto(1247.43454107,141.62678248)
\lineto(1246.53854071,136.05878026)
\closepath
}
}
{
\newrgbcolor{curcolor}{0 0 0}
\pscustom[linewidth=1.89354326,linecolor=curcolor]
{
\newpath
\moveto(1046.91464315,167.89709175)
\lineto(1051.91465575,167.89709175)
\lineto(1051.91465575,121.89709301)
\lineto(1046.91464315,121.89709301)
}
}
{
\newrgbcolor{curcolor}{0 0 0}
\pscustom[linewidth=1.89354326,linecolor=curcolor]
{
\newpath
\moveto(1051.91465575,142.89710057)
\lineto(1056.91466835,142.89710057)
}
}
{
\newrgbcolor{curcolor}{0 0 0}
\pscustom[linewidth=2.13543306,linecolor=curcolor]
{
\newpath
\moveto(202.91468598,574.89709301)
\lineto(191.49070488,574.89709301)
}
}
{
\newrgbcolor{curcolor}{0 0 0}
\pscustom[linewidth=2.13543306,linecolor=curcolor]
{
\newpath
\moveto(202.9146822,64.89709301)
\lineto(191.49070148,64.89709301)
}
}
{
\newrgbcolor{curcolor}{0 0 0}
\pscustom[linewidth=2.13543306,linecolor=curcolor]
{
\newpath
\moveto(202.9146822,29.8970842)
\lineto(191.49070148,29.8970842)
}
}
\end{pspicture}

	\caption{An illustration of the sample as described in the simulated model, not to scale. Each bilayer is given an index, dubbed $j$. For illustrative purposes, only a limited amount of bilayers are drawn.}
	\label{modeldescription}
\end{figure}