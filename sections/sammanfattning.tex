\chapter{Populärvetenskaplig sammanfattning}
Neutronspridning är en populär teknik som kan användas för att studera olika materialegenskaper. Tekniken är på många sätt jämförbar med röntgenspridning, men det finns några viktiga skillnader mellan de två teknikerna. Röntgenstrålning interagerar med elektronmolnen runt en atom medan neutroner interagerar med själva atomkärnan. Den här skillnaden ger neutronspridning flera fördelar jämfört med röntgenspridning. Atomkärnan utgör bara en väldigt liten del av atomen som till största del består av tomrum. Det gör det möjligt för neutroner, som enbart interagerar med atomkärnan, att tränga långt in i de flesta material. Detta möjliggör undersökning av tjockare bulk-lager. En annan fördel med att använda neutroner är att de växelverkar med magnetiska material och därför kan mäta magnetiska egenskaper, något som inte är möjligt med röntgenstrålning. Spridningsegenskaperna, som beskriver hur en neutron interagerar med ett material, är tillsynes helt slumpmässiga beroende på isotop. Till exempel så kan olika isotoper av samma grundämne till och med ha helt olika egenskaper. Eftersom röntgen- och neutronspridning ger helt oberoende mätvärden som beskriver samma egenskaper har en kombination av teknikerna använts för att analysera material i det här arbetet. \\
Redan under 90-talet bestämdes det att en ny neutronkälla skulle byggas i Europa. År 2014 började konstruktionen av European Spallation Source (ESS) i Lund, och 2027 förväntas neutrokällan vara i full drift. Byggandet av ESS medför ett större behov av ny kunskap kring neutronspridning i Sverige. Det är därför forskarskolan SwedNess grundades, som det här projektet är en del av. \\
Det här projektet är inriktat på så kallade multilager, som används i många optiska komponenter vid neutronkällor. Multilager består av väldigt tunna lager av olika material. Ett sådant multilager är neutronspeglar som behövs för att transportera neutronerna från källan till platsen där experimentet utförs. För att många neutroner ska reflekteras, behöver man en stor skillnad i spridningsegenskaperna mellan de olika material i multilagret, samt att gränsnittet mellan de olika lager är jämnt och slätt.

Ett av de mest populära materialsystemen för multilager i neutronkällor idag är en kombination av nickel och titan. Denna kombination ger en hög reflektans tack vare den stora skillnaden i spridningsgenskaper. Genom att tillsätta lågabsorberande borkarbid (\BC) kan man skapa amorfiska multilager utan kristaller, vilket möjliggör tillverkning av jämnare lager. Multilagren i detta arbete tillverkas med hjälp av en teknik som kallas sputtring, under sputtringsproccen bombarderas gränsnittet  med joner för att ytterligare jämna ut det under tillväxtprocessen. Multilagren har karaktäriserats med hjälp av olika tekniker, såsom elektronmikroskopi, röntgenspridning och neutronspridning. Genom att anpassa en simulerad modell till röntgen- och neutronmätningar har det kunnat visa att multilagren har en mycket hög kvalitet, där gränsnittets tjocklek endast är några få atomer (0,27 nm). Detta innebär en stor förbättring jämfört med de multilager som används i neutronkällor idag. Eftersom multilager är så viktiga i så många olika komponenter är resultatet av stor betydelse för alla neutronmätningar som utförs.